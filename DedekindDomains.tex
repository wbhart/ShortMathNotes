\documentclass[10pt]{article}
\usepackage{amsfonts}
\usepackage{amssymb, amsmath}
\usepackage{eucal}
\usepackage{amscd}
\usepackage{url}
\usepackage{listings}
\usepackage{algorithmic}
\usepackage{enumerate}
\urlstyle{sf}
\pagestyle{plain}

\newcommand{\Z}{\mathbb{Z}}
\newcommand{\N}{\mathbb{N}}
\newcommand{\Q}{\mathbb{Q}}
\newcommand{\I}{\mathbb{I}}
\newcommand{\C}{\mathbb{C}}
\newcommand{\R}{\mathbb{R}}
\newcommand{\F}{\mathbb{F}}
\newcommand{\Pee}{\mathbb{P}}
\newcommand{\Op}{\mathcal{O}}
\newcommand{\Qbar}{\Opverline{\mathbb{Q}}}
\newcommand{\code}{\lstinline}

\newcommand{\ljk}[2]{\left(\frac{#1}{#2}\right)}
\newcommand{\modulo}[1]{\;\left(\mbox{mod}\;#1\right)}
\newcommand{\fr}{\mathfrak}
\newcommand{\qed}{\square}

\def\notdivides{\mathrel{\kern-3pt\not\!\kern4.5pt\bigm|}}
\def\nmid{\notdivides}
\def\nsubseteq{\mathrel{\kern-3pt\not\!\kern2.5pt\subseteq}}

\newtheorem{theorem}{Theorem}[section]
\newtheorem{lemma}[theorem]{Lemma}
\newtheorem{proposition}[theorem]{Proposition}
\newtheorem{corollary}[theorem]{Corollary}
\newtheorem{property}[theorem]{Property}

\newenvironment{proof}[1][Proof]{\begin{trivlist}
\item[\hskip \labelsep {\bfseries #1}]}{\end{trivlist}}
\newenvironment{definition}[1][Definition]{\begin{trivlist}
\item[\hskip \labelsep {\bfseries #1}]}{\end{trivlist}}
\newenvironment{example}[1][Example]{\begin{trivlist}
\item[\hskip \labelsep {\bfseries #1}]}{\end{trivlist}}
\newenvironment{remark}[1][Remark]{\begin{trivlist}
\item[\hskip \labelsep {\bfseries #1}]}{\end{trivlist}}

\parindent=0pt
\parskip 4pt plus 2pt minus 2pt 

\title{Dedekind domains}

\author{
William B. Hart
}

\begin{document}

\maketitle

\section{Dedekind domains}

In any elementary treatment of algebraic number fields, three important facts about the ring of integers $\Op_K$ of a number field are proved: (i) $\Op_K$ is integrally closed in the number field, (ii) $\Op_K$ is noetherian, (iii) every prime ideal of $\Op_K$ is maximal.

Integral domains that have these properties are called Dedekind domains and these can be studied directly rather than study the special case of the ring of integers of a number field.

As well as the ring of integers of a number field there are other important examples of a Dedekind domain. For example instead of number fields, consider finite extensions $K$ of the function field $\mathbb{F}_p(t)$. If we let $k$ be the algebraic closure of $\mathbb{F}_p$ in $K$ then the integral closure of $k[t]$ in $K$ is a Dedekind domain.

On account of these other important examples, it is useful to study Dedekind domains indepedently of number fields, in full generality.

We begin with the formal definition of a Dedekind domain.

\begin{definition}
An integral domain $R$ is a Dedekind domain if
\begin{itemize}
\item $R$ is noetherian
\item $R$ is integrally closed in its field of fractions
\item every prime ideal of $R$ is maximal
\end{itemize}
\end{definition}

\subsection{Integrality}

We begin by giving an overview of the second condition in the definition of a Dedekind domain, namely integral closure.

We begin by recalling the definition.

\begin{definition}
If $R$ is a subring of a commutative ring $S$ then $\alpha \in S$ is said to be \emph{integral} over $R$ if it is a root of a monic polynomial
$$f(x) = x^n + r_{n-1}x^{n-1} + \cdots + r_0,$$
with coefficients $r_i \in R$.
\end{definition}

There are some convenient alternative formulations of integrality.

\begin{theorem}
Let $R$ be a subring of a commutative ring $S$, with $\alpha \in S$. Then the following are equivalent.
\begin{enumerate}
\item $\alpha$ is integral over $R$.
\item The subring $R[\alpha] \subseteq S$ is a finitely generated $R$-module.
\item There exists a finitely generated $R$-module $M$ with an action of $\alpha$ on it making $M$ into an $R[\alpha]$-module such that the only element $r \in R[\alpha]$ for which $rM = 0$ is $r = 0$.
\end{enumerate}
\end{theorem}

\begin{proof}
(1) $\rightarrow$ (2): Since $\alpha^n$ can be written as an $R$-linear combination of $S = \{1, \alpha, \ldots, \alpha^{n-1}\}$ using the relation $f(\alpha) = 0$, it is clear that $R[\alpha]$ is generated by $S$ as an $R$-module.

(2) $\rightarrow$ (3): Suppose $R[\alpha]$ is finitely generated and let $M = R[\alpha]$. If $\alpha M = 0$ then $\alpha = 0$ and $\alpha R = 0$. But the zero of $R$ is unique and so $\alpha = 0$. Thus $M = R[\alpha]$ satisfies (3).

(3) $\rightarrow$ (1): This is Dedekind's argument. Suppose $M$ is generated over $R$ by $\{m_1, m_2, \ldots, m_n\}$. Then we can write each $\alpha m_i$ as an $R$-linear combination of the $m_i$. Writing this as a matrix equation we note that there is a nontrivial solution iff $\alpha$ satisfies a monic polynomial over $R$ of degree at most $n$. $\qed$
\end{proof}

\begin{definition}
An $R$-module $M$ for $rM = 0$ for $r \in R$ iff $r = 0$ is called a \emph{faithful} $R$-module.
\end{definition}

With this definition we can express the third formulation of the theorem as: there exists a faithful $R[\alpha]$-module which is finitely generated as an $R$-module.

From the first part of the theorem, we have the following result.

\begin{corollary}
If $\alpha \in S$ is integral over $R \subseteq S$ then $\alpha$ is integral over any subring $T$ with $R \subseteq T \subseteq S$.
\end{corollary}

\begin{proof}
The coefficients $r_i$ in the first part of the theorem are in $R$ and hence in $T$. $\qed$
\end{proof}

We can also say something about integrality in subrings, by part (2) of the theorem.

\begin{theorem}
If $R \subseteq T \subseteq S$ is a tower of subrings and $\alpha \in S$ is integral over $T$ and $T$ is finitely generated as an $R$-module, then $\alpha$ is also integral over $R$.
\end{theorem}

\begin{proof}
By part (2) of the theorem above, if $\alpha$ is integral over $T$ then $T[\alpha] \subseteq S$ is a finitely generated $T$-module.

But since $T$ is finitely generated as an $R$-module, $T[\alpha]$ is finitely generated over $R$.

Thus again by part (2), $\alpha$ is integral over $R$. $\qed$
\end{proof}

From the part (3) of the theorem above, we have the following result.

\begin{theorem}
If $R$ is a subring of $S$ with $\alpha \in S$ integral over $R$, then any $\beta \in R[\alpha]$ is also integral over $R$.
\end{theorem}

\begin{proof}
As $R[\beta] \subseteq R[\alpha]$, the action of $R[\alpha]$ on a faithful $R[\alpha]$ module $M$ induces an action of $R[\beta]$ on $M$. This makes $M$ into a faithful $R[\beta]$ module. Thus $\beta$ is also integral over $R$. $\qed$
\end{proof}

We recall the following theorem from algebraic number theory. It is now able to be proved more routinely.

\begin{theorem}
If $R$ is a subring of a ring $S$, then the set of all $\alpha \in S$ which are integral over $R$ forms a ring.
\end{theorem}

\begin{proof}
Let $\alpha, \beta \in S$ be integral over $R$. Then $\alpha$ is integral over $R[\beta]$ by the above. Thus any element of $R[\alpha][\beta] = R[\alpha, \beta]$ is integral over $R[\beta]$, by the previous theorem.

As $R[\beta]$ is finitely generated over $R$, we have that every element of $R[\alpha, \beta]$ is integral over $R$, by the above. In particular, $\alpha + \beta$ and $\alpha\beta$ are integral over $R$. $\qed$
\end{proof}

We recall the following definition.

\begin{definition}
If $R$ is a subring of $S$, the ring of all $\alpha \in S$ which are integral over $R$ is called the \emph{integral closure} of $R$ in $S$.
\end{definition}

We also have the following definition.

\begin{definition}
If $R$ is an integral domain with field of fractions $K$, we say that $R$ is \emph{integrally closed} if $R$ is its own integral closure in $K$.
\end{definition}

\subsection{Unique factorisation into prime ideals}

\begin{theorem}
In a Dedekind domain, every ideal contains a product of prime ideals.
\end{theorem}

\begin{proof}
Suppose to the contrary that there exists an ideal in a Dedekind domain $R$ which doesn't contain a product of prime ideals. Let $S$ be the (nonempty) set of all ideals of $R$ which don't contain a product of prime ideals.

As $R$ is a Dedekind domain, it is Noetherian. Thus $S$ has a maximal element, $I$ say.

We see that $I$ cannot be prime, by the definition of $S$. Thus there exists $r, s \in R$ such that $rs \in I$ but $r, s \notin I$. Let $A = I + (r)$ and $B = I + (s)$. 

We have that the ideals $A$ and $B$ are both strictly bigger than $I$. Thus $A$ and $B$ both contain a product of prime ideals. But this implies that $AB$ contains a product of prime ideals.

However, as $rs \in I$ we have that $AB = (I + (r))(I + (s)) \subseteq I$. This is a contradiction since now $I$ contains a product of prime ideals. $\qed$
\end{proof}

We will also need the following result. Recall that an ideal of a ring $R$ is said to be proper if it is not the whole ring $R$

\begin{theorem}
Let $I$ be a proper ideal of a Dedekind domain $R$ with field of fractions $K$ (e.g. the ring of integers $R = \mathcal{O}_K$ in a number field $K$). Then there exists a non-integral $\alpha \in K$, i.e. $\alpha \notin R$, such that $\alpha I \subseteq R$.
\end{theorem}

\begin{proof}
Let $a \neq 0$ be an arbitrary element of $I$. By the previous theorem, $(a)$ contains a product of prime ideals.

Let $n$ be the smallest positive integer for which there exist prime ideals $P_i$ for which $P_1P_2\ldots P_n \subseteq (a)$.

Since $I$ is a proper ideal, it is contained in a maximal ideal, and as $R$ is a Dedekind domain, this implies $I$ is contained in a prime ideal, $P$ say.

This implies that $P_1P_2\ldots P_n \subseteq P$. We will show that $P$ contains one of the $P_i$.

If not, then for each $i$ let $a_i \in P_i$ but $a_i \notin P$. Let $p = a_1a_2\ldots a_n$.

But $P$ is prime, and so it must contain one of $a_i$, which is a contradiction. Thus $P$ contains one of the $P_i$ as claimed.

Without loss of generality, suppose that $P_i \subseteq P$.

But $P_i$ is a prime ideal and hence maximal. Therefore $P = P_i$.

If $n = 1$, set $b = 1$, in which case $b \notin (a)$, since $(a) \subseteq I$ and $I$ is a proper ideal. 

If $n \neq 1$, by the minimality of $n$, there must exist $b \in (P_2P_3\ldots P_n)$ with $b \notin (a)$.

In either case, if we set $\gamma = b/a$ then $\gamma \in K$ but $\gamma \notin I$ and $bI \subseteq bP = bP_1 \subseteq (a)$. Thus $\gamma I = (b/a)I \subseteq R$. $\qed$
\end{proof}

\begin{theorem}\label{princ}
For any ideal $I$ of a Dedekind domain $R$ there exists a non-zero ideal $J$ such that $IJ$ is principal.
\end{theorem}

\begin{proof}
Let $a \in I$ be nonzero. Let $J = \{b \in R \;|\; bI \subseteq (a)\}$. It is easy to check that the difference of any two elements of $J$ is in $J$, as is the product of any element of $J$ with an element of $R$. Thus $J$ is a nonzero ideal of $R$. Clearly we have $IJ \subseteq (a)$. We will show that in fact $IJ = (a)$.

Let $A = (1/a)IJ$. As $IJ \subseteq (a)$ we have that $A \subseteq R$. As $IJ$ is an ideal of $R$, $A$ must be as well.

If $A = R$ then $IJ = (a)$ and we are done. Otherwise $A$ is a proper ideal of $R$.

By the previous theorem, there exists $\gamma \in K$ with $\gamma \notin R$ such that $\gamma A \subseteq R$. 

As $a \in I$ we have $aJ \subseteq IJ$. Thus $J \subseteq (1/a)IJ = A$. Thus $\gamma J \subseteq \gamma A \subseteq R$. As $J$ is an ideal, this implies $\gamma J \subseteq J$.

Suppose $\alpha_1, \ldots, \alpha_m$ are generators of $J$ as an ideal. Then we can write
$$\gamma \left(\begin{array}{c}\alpha_1 \\ \alpha_2 \\ \vdots \\ \alpha_m\end{array}\right) = M\left(\begin{array}{c}\alpha_1 \\ \alpha_2 \\ \vdots \\ \alpha_m\end{array}\right),$$
for some $m\times m$ matrix.

This is only possible if det$(\gamma I_m - M) = 0$. But this makes $\gamma$ a root of a monic polynomial over $R$. Thus $\gamma \in R$ by the fact that $R$ is integrally closed in its field of fractions.

But this is a contradiction. Thus $IJ = (\alpha)$ after all $\qed$. 
\end{proof}

We can use this to prove the following cancellation law for ideals.

\begin{theorem}\label{abaclab}
If $A, B, C$ are ideals of a number field $K$ and $AB = AC$ with $A \neq (0)$ then $B = C.$
\end{theorem}

\begin{proof}
Choose an ideal $M$ as per the previous theorem, such that $MA = (a)$ is principal. Then multiplying $AB = AC$ through by $M$ we have that $(a)B = (a)C$, and the result follows by division through by $a$. $\qed$
\end{proof}

We are now also in a position to establish that for ideals, `to divide is to contain', as we already mentioned above.

Recall that we say an ideal $B$ divides an ideal $A$ if there is an ideal $C$ such that $A = BC$. We write $B \;|\; C$.

\begin{theorem}
If $A$ and $B$ are ideals in a Dedekind domain $R$ then $B \;|\; A$ iff $B \supseteq A$.
\end{theorem}

\begin{proof}
If $B \;|\; A$ then $A = BC$ for some ideal $C$. It is then clear that $A \subseteq B$ since $B$ is an ideal and $C \subseteq R$.

Conversely, suppose $A \subseteq B$. By the theorem above, we can find an ideal $J$ such that $BJ = (a)$ for some $a \in R$.

Now $AJ \subseteq BJ = (a)$. Thus $C = (1/a)AJ$ is an ideal in $R$ and $BC = A$. Thus $B \;|\; A$. $\qed$
\end{proof}

For ideals, there is a concept of greatest common divisor, just as for rational integers.

We say that an ideal $G$ is a greatest common divisor of ideals $A$ and $B$ if $G$ divides $A$ and $B$, and any other ideal that divides $A$ and $B$ also divides $G$.

\begin{theorem}
Each pair $A = (\alpha_1, \alpha_2, \ldots, \alpha_k), B = (\beta_1, \beta_2, \ldots, \beta_m)$ of ideals in a Dedekind domain, possesses a greatest common divisor $G = \mbox{gcd}(A,B)$.
It has the form
$$G = (\alpha_1, \alpha_2, \ldots, \alpha_k, \beta_1, \beta_2, \ldots, \beta_m).$$
\end{theorem}

\begin{proof}
Clearly $G$ consists of all elements of the form $\alpha + \beta$ with $\alpha \in A, \beta \in B$.

Since every ideal contains 0 then $G \supseteq A$ and $G \supseteq B$. Thus by the previous theorem, $G \mid A$ and $G \mid B$.

The result follows by noting that if $H \mid A$ and $H \mid B$ for some ideal $H$ then $H \supseteq A$ and $H \supseteq B$ and so $H$ must contain every sum of the form $\alpha + \beta$ with $\alpha \in A$ abd $\beta \in B$. In other words, $H \supseteq G$, i.e. $H \mid G$. $\qed$
\end{proof}

From the expression for $G$ in the theorem, and the definition of ideal multiplication, we obtain
immediately

\begin{theorem}
For ideals $A, B$ and $C$
$$C\cdot \mbox{gcd}(A, B) = \mbox{gcd}(CA, CB).$$
\end{theorem}

\begin{definition}
We call an ideal \emph{irreducible} if it is not the product of two nontrivial ideals (we refer to $\Op_K = (1)$ as the trivial ideal).
\end{definition}

\begin{definition}
We say that ideals $A$ and $B$ are \emph{coprime} if their greatest common divisor is the trivial ideal $(1)$.
\end{definition} 

\begin{theorem}
If $P$ is an irreducible ideal and $P \mid AB$ for ideals $A$ and $B$ then $P$ divides $A$ or $B$ (or both), i.e. $P$ is a prime ideal.
\end{theorem}

\begin{proof}
If $P \nmid B$ then gcd$(P, B) = (1)$ since $P$ has no
other factors. Thus
$$A = A(1) = A\cdot\mbox{gcd}(P, B) =
\mbox{gcd}(AP, AB),$$
and since $P \mid AB$ this equation says that $P \mid B$. $\qed$
\end{proof}

\begin{theorem}
Each ideal $I$ of a Dedekind domain can be written as a product of finitely many prime ideals.
\end{theorem}

\begin{proof}
Assume to the contrary that there is an ideal $I$ in a Dedekind domain $R$ which cannot be written as a product of finitely many prime ideals.

Let $S$ be the set of proper ideals of $R$ that cannot be written as a product of prime ideals. As $R$ is a Dedekind domain $S$ must contain a maximal element $M$.

We have that $M$ is contained in some maximal, and hence prime ideal $P$, i.e. $P \mid M$. Thus $M = PA$ for some ideal $A$. But now $A \supseteq M$. 

In fact, this containment is strict, otherwise we would have $A = M$ which would imply $RM = M = PA = PM$, which by cancellation would imply $R = P$, a contradiction. Thus $A \supset M$. 

But since $M$ was maximal amongst the elements of $S$, we must have that $A$ is a product of prime ideals. But then $M = PA$ is also a product of prime ideals, which is a contradiction.

Thus all ideals must be a product of prime ideals after all. $\qed$
\end{proof}

Unique factorisation into prime ideals now follows in precisely the same way as it does for rational integers, from the analogue of Euclid's Lemma above.

\begin{theorem}
Every ideal $A$ of a Dedekind domain has unique factorization into prime ideals, upto order.
\end{theorem}

\begin{proof}
Suppose $P_1P_2\ldots P_r = Q_1Q_2\ldots Q_s$ for prime ideals $P_i$ and $Q_j$.

As $P_1$ divides the left side, it divides the right. By repeated application of Euclid's Lemma we have that it divides one of the $Q_j$. Without loss of generality suppose it divides $Q_1$.

But $P_1$ and $Q_1$ are both prime and hence maximal. Thus $P_1 = Q_1$.

By cancellation, we therefore have
$$P_2P_3\ldots P_r = Q_2Q_3\ldots Q_s.$$

Repeating the argument, we eventually find that each $P_i$ is equal to some $Q_j$ and that $r = s$. Thus the two factorisations are in fact the same up to order of factors. $\qed$
\end{proof} 

\subsection{The Chinese Remainder Theorem}

There is also an analogue of the Chinese Remainder Theorem in Dedekind domains. It depends on the following definition.

\begin{definition}
For a prime ideal $P$ in a Dedekind domain, we write $a \equiv b \pmod{P^i}$ for $a, b \in R$ if $a - b \in P^i$.
\end{definition}

\begin{theorem} (Chinese Remainder Theorem)
Let $R$ be a Dedekind domain, and let $P_1, \ldots, P_m$ be mutually coprime ideals in $R$. If $x_1, \ldots, x_m \in R$ are arbitrary elements, then it is possible to find $a \in R$ such that $a \equiv x_i \pmod{P_i}$ for all $i$.
\end{theorem}

\begin{proof}
Let $Q_i = P_1P_2\ldots P_{i-1}P_{i+1}\ldots P_m$. As all the $P_i$ are mutually coprime, we have that $(P_i, Q_i) = (1)$. Thus there exists $p_i \in P_i$ and $q_i \in Q_i$ such that $rp_i + sq_i = 1$ for some $r, s \in R$.

As $rp_i \in P_i$ and $q_i \in Q_i$, we see that $sq_i \equiv 1 \pmod{P_i}$ but $sq_i \equiv 0 \pmod{Q_i}$.

Thus $a = \sum_i x_isq_i$ has the required properties. $\qed$
\end{proof}

With a little work it is possible to use unique factorisation and the Chinese Remainder Theorem to derive the following technical result.

\begin{theorem}\label{abw}
For any given nonzero ideals $A, B$ of a Dedekind domain $R$, there exists an $a \in R$ such that $(a, AB) = A$. 
\end{theorem}

\begin{proof}
Write $A = P_1^{r_1}\ldots P_m^{r_m}$ and $AB = P_1^{s_1}\ldots P_m^{s_m}$ for $r_i, s_j \geq 0$.

As $A \mid AB$ we have $s_i \geq r_i$ for all $i$.

For each $i$, let $x_i \in A$ be chosen with $x_i \in P_i^{r_i}$ but $x_i \neq P_i^{r_i + 1}$.

We can then use the Chinese Remainder Theorem for Dedekind domains to find an $a \in R$ such that $a \equiv x_i \pmod{P_i^{r_i}}$ for all $i$.

Now as $a - x_i \in P_i^{r_i}$ and $x_i \in P_i^{r_i}$ but $x_i \notin P_i^{r_i + 1}$, we have that $a \in P_i^{r_i}$ but $a \notin P_i^{r_i + 1}$, for all $i$.

In other words, $(a)$ is divisible by $P_i^{r_i}$ but not any higher power of $P_i$, for all $i$.

Thus $(a, AB) = \mbox{gcd}((a), AB) = A$.
$\qed$
\end{proof}

By letting $B$ be the ideal of Theorem \ref{princ} with $AB = (\beta)$ we obtain the following special case of this result.

\begin{corollary}
Every ideal $A$ of a Dedekind domain $R$ has a representation $A = (\omega, \beta)$ for \emph{two} elements $\omega$ and
$\beta$ of $R$.
\end{corollary}

\subsection{Principal ideal domains}

\begin{theorem}
A principal ideal domain $R$ is a Dedekind domain.
\end{theorem}

\textbf{Proof:}
Recall that a ring in which every ideal is finitely generated is noetherian. This applies to a principal ideal domain since every ideal is generated by one element.

To show that $R$ is integrally closed, suppose that $a, b \in R$ with $b \neq 0$. Suppose that $a/b$ is integral over $R$.

As $R$ is a principal ideal domain, there exists a $d \in R$ such that $(a, b) = (d)$. In particular this means that $a = a'd$ and $b = b'd$ for some $a', b' \in R$. There must also exist $s, t \in R$ such that $as + bt = d$, i.e. $a'ds + b'dt = d$ or $a's + b't = 1$. Thus $a/b = a'/b'$ and $(a', b') = 1$.

Replacing $a$ with $a'$ and $b$ with $b'$ we may suppose that $a/b$ is integral in $R$ with $(a, b) = (1)$.

As $a/b$ is integral we have that
$$(a/b)^n + c_{n-1}(a/b)^{n-1} + \cdots + c_1(a/b) + c_0 = 0$$
for some $c_i \in R$.

Multiplying through by $b^n$ yields
$$a^n = -b(c_{n-1}a^{n-1} + \cdots + c_1ab^{n-2} + c_0b^{n-1}).$$

In particular $b \;|\; a^n$. But $(a, b) = (1)$ and so $(a, b)^n = (1)$. But $(a, b)^n$ is generated by $a^ib^j$ for $i + j = n$, all of which are divisible by $b$. Thus $(b) \;|\; (a, b)^n = (1)$.

This implies that $1 = bb'$ for some $b' \in R$, i.e. $b$ is invertible in $R$ and $a/b \in R$. Thus $R$ is integrally closed.

Finally to show that every prime ideal of $R$ is maximal, consider a prime ideal $(p)$ in $R$. Clearly $p$ must be irreducible, otherwise $p = ab$ for two non-invertible elements $a, b \in R$. But this is impossible since $(p) \supseteq (a)(b)$, and yet if $(p) \supseteq (a)$ for example, then $a = pc$ for some $c \in R$, which would imply that $p = pcb$, i.e. $cb = 1$, which contradicts that $b$ is non-invertible.

Thus $p$ is irreducible. Thus $(p)$ must be maximal. For if $R \supseteq (d) \supseteq (p)$ then $p = dp'$ for some $p' \in R$. As $p$ is irreducible this implies that $p'$ is invertible in $R$ and $(d) = (p)$, proving that $(p)$ is maximal. $\qed$

\subsection{Localisation}

Localisation is a generalisation of the construction of the field of fractions of an integral domain. It can be defined for any commutative ring, but is quite straightforward in the case of an integral domain.

First we need the notion of a multiplicatively closed set.

\begin{definition}
Given a commutative ring $R$, a \emph{multiplicative} subset $S$ of $R$ is a subset containing $1$ and such that $s_1s_2 \in S$ for all $s_1, s_2 \in S$.
\end{definition}

\begin{example}
In the case of an integral domain $R$, the set $S$ of all nonzero elements of $S$ is a multiplicative subset of $S$.
\end{example}

\begin{example}
Given a commutative ring $R$ and an element $r \in R$, the set $S = \{1, r, r^2, \ldots\}$ is a multiplicative subset of $R$.
\end{example}

\begin{definition}
Given an integral domain $R$ and a multiplicative set $S \subseteq R$ consider the set $R^\star = \{r/s \;|\; r \in R, s \in S\}$. We define $S^{-1}R$ to be $R^\star$ modulo equivalence, where equivalence is defined as for the fraction field of $R$. We call $S^{-1}R$ the localisation of $R$ at $S$.
\end{definition}

For general commutative rings the definition is similar except that two elements $r_1/s_1, r_2/s_2$of $R^\star$ are taken to be equivalent if $t(r_1s_2 - r_2s_1)$ for some $t \in S$. However this is not the case of interest in what follows.

An example that will be particularly important to us follows from the following result.

\begin{theorem}
Let $R$ be a commutative ring and $\mathfrak{p}$ a prime ideal of $R$. Let $S = R - \mathfrak{p}$. Then $S$ is a multiplicative subset of $R$.
\end{theorem}

\textbf{Proof:} Clearly $1 \in S$, otherwise we would have $1 \in \mathfrak{p}$ which would make $\mathfrak{p} = R$ which is not a prime ideal.

If $s_1, s_2 \in S$ then $s_1$ and $s_2$ are not in the prime ideal $\mathfrak{p}$. Now suppose that $s_1s_2 \in \mathfrak{p}$. In particular $(s_1R)(s_2R) \subseteq \mathfrak{p}$. 

As $\mathfrak{p}$ is a prime ideal we must have that either $s_1R \subseteq \mathfrak{p}$ or $s_2R \subseteq \mathfrak{p}$. But this implies that either $s_1$ or $s_2$ is an element of $\mathfrak{p}$, which is a contradiction.

Thus $s_1s_2 \notin \mathfrak{p}$. $\qed$

\begin{definition}
Let $R$ be a commutative ring and $\mathfrak{p}$ a prime ideal of $R$. If $S = R - \mathfrak{p}$ then $S^{-1}R$ is known as the localisation of $R$ at $\mathfrak{p}$ and is denoted $R_{\mathfrak{p}}$.
\end{definition}

\begin{definition}
A commutative ring $R$ with a unique maximal ideal $\mathfrak{m}$ is called a \emph{local ring}.
\end{definition}

\begin{theorem}
If $R$ is a commutative ring and $\mathfrak{p}$ a prime ideal of $R$ then $R_{\mathfrak{p}}$ is a local ring with maximal ideal $\mathfrak{p}R_{\mathfrak{p}}$.
\end{theorem}

\textbf{Proof:} The units of $R_{\mathfrak{p}}$ are the elements of the form $a/b$ with $a \notin \mathfrak{p}$. Thus the non-units of $R_{\mathfrak{p}}$ are the elements of $\mathfrak{p}R_{\mathfrak{p}}$.

A maximal ideal of $R_{\mathfrak{p}}$ can't contain a unit, so all maximal ideals of $R_{\mathfrak{p}}$ are contained in $\mathfrak{p}R_{\mathfrak{p}}$.

We will show that $\mathfrak{p}R_{\mathfrak{p}}$ is an ideal of $R_{\mathfrak{p}}$. It is certainly an additive subgroup, since if $r_1/s_1, r_2/s_2 \in \mathfrak{p}R_{\mathfrak{p}}$ then $r_1/s_1 - r_2/s_2 = (r_1s_2 - r_2s_1)/(s_1s_2) \in \mathfrak{p}R_{\mathfrak{p}}$ as $r_1, r_2 \in \mathfrak{p}$ and therefore so are $r_1s_2, r_2s_1$ and $r_1s_2 - r_2s_1$. $\mathfrak{p}R_{\mathfrak{p}}$ is also clearly closed under multiplication of elements of $R_{\mathfrak{p}}$.

Thus $\mathfrak{p}R_{\mathfrak{p}}$ must in fact be the unique maximal ideal of $R_{\mathfrak{p}}$. $\qed$

\begin{theorem}
Given a commutative ring $R$ with multiplicative subset $S$ the map $f : R \to S^{-1}R$ which sends $r$ to $r/1$ is a ring homomorphism. We call this map the \emph{canonical homomorphism}. If $R$ is an integral domain then the map is injective.
\end{theorem}

\textbf{Proof:} Clearly $f(1) = 1$, $f(a + b) = (a + b)/1 = a/1 + b/1 = f(a) + f(b)$ and $f(ab) = (ab)/1 = (a/1)(b/1) = f(a)f(b)$.

If $R$ is an integral domain and $f(a) = f(b)$ then $a/1 = b/1$, i.e. $a = b$. $\qed$

\begin{theorem}
If $R$ is a commutative ring and $S$ a multiplicative subset then every ideal of $S^{-1}R$ is of the form $S^{-1}I$ for some ideal $I$ of $R$.
\end{theorem}

\textbf{Proof:}
Let $J$ be an ideal of $S^{-1}R$, $f : R \to S^{-1}R$ be the canonical homomorphism and $I = f^{-1}(J)$.

Since $f$ is a ring homomorphism, $I$ is an ideal of $R$.

Let $r/s \in J$. As $J$ is an ideal, $sr/s = r \in J$. Thus $r \in I$. Thus $r/s \in S^{-1}I$.

As $r/s \in J$ was arbitrary we see that $J \subseteq S^{-1}I$.

But $f(I) = J$ by definition, and thus if $r \in I$ then $r/1 \in J$. But $J$ is an ideal of $S^{-1}R$ and so therefore $r/s = (r/1)(1/s) \in J$ for any $s \in S$.

But $r \in I$ was arbitrary and so $r/s \in J$ for any $r/s \in S^{-1}I$, i.e. $S^{-1}I \subseteq J$.

Combining the two inclusions we have that $J = S^{-1}I$. $\qed$

\begin{theorem}
If $R$ is a commutative ring and $S$ a multiplicative subset then the prime ideals of $S^{-1}R$ are of the form $\mathfrak{q}S^{-1}R$ where $\mathfrak{q}$ is a prime ideal of $R$ that doesn't intersect $S$.
\end{theorem}

\textbf{Proof:}
Suppose $\mathfrak{p}$ is a prime ideal of $R$ not intersecting $S$.

Suppose $(r_1/s_1)(r_2/s_2) \in S^{-1}\mathfrak{p}$ for some $r_1/s_1, r_2/s_2 \in S^{-1}R$. Then $t(r_1r_2s - ps_1s_2)$ for some $p \in \mathfrak{p}$ and $s \in S$.

Then $r_1r_2(st) = rtps_1s_2 \in \mathfrak{p}$.

Since $\mathfrak{p}$ doesn't intersect $S$ this implies that $r_1r_2 \in \mathfrak{p}$. But as $\mathfrak{p}$ is a prime ideal, this implies that either $r_1 \in \mathfrak{p}$ or $r_2 \in \mathfrak{p}$.

In other words, we have shown that $S^{-1}\mathfrak{p}$ is a prime ideal.

Recall that if $f : R \to S^{-1}R$ is the map given by $r \mapsto r/1$ then $f^{-1}(S^{-1}\mathfrak{p}) = \mathfrak{p}$. Thus $\mathfrak{p}$ is uniquely determined by $S^{-1}\mathfrak{p}$.

Now suppose that $\mathfrak{q}$ is an arbitrary prime ideal of $S^{-1}R$. Then $\mathfrak{q} = S^{-1}\mathfrak{p}$ where $\mathfrak{p} = f^{-1}(\mathfrak{q})$.

But $f$ is a ring homomorphism, thus $\mathfrak{p} = f^{-1}(\mathfrak{q})$ is a prime ideal.

We must have $\mathfrak{p} \cap S = \emptyset$, since if $s \in S$ were an element of $\mathfrak{p}$ then $f(s) = s/1$ would be an element of $\mathfrak{q}$, which is impossible, since it is a unit in $S^{-1}R$. $\qed$

\begin{corollary}
If $R$ is an integral domain and $\mathfrak{p}$ is a prime ideal of $R$ then the prime ideals of $R_{\mathfrak{p}}$ are of the form $\mathfrak{q}R_{\mathfrak{p}}$ for $\mathfrak{q}$ a prime ideal of $R$ contained in $\mathfrak{p}$.
\end{corollary}

\textbf{Proof:} In this case $S = R - \mathfrak{p}$, so an ideal that doesn't intersect $S$ is an ideal contained in $\mathfrak{p}$. $\qed$

\begin{theorem} \label{locnoeth}
If $R$ is a noetherian ring and $S$ is a multiplicative subset, then $S^{-1}A$ is noetherian.
\end{theorem}

\textbf{Proof:} Suppose $\{S^{-1}I \;|\; I \in T\}$ is a collection of ideals of $S^{-1}R$ for some collection $T$ of ideals $I$ of $R$.

Since $R$ is noetherian any ascending chain of ideals in $T$ stabilises with some $I_0$, the corresponding chain of ideals in $S^{-1}I$ stabilises with $S^{-1}I_0$. $\qed$

\begin{theorem} \label{locintclosed}
If $R$ is an integral domain which is integrally closed, and $\mathfrak{p}$ is a nonzero prime ideal of $R$ then $R_{\mathfrak{p}}$ is integrally closed.
\end{theorem}

\textbf{Proof:}
Let $K$ be the field of fractions of $R$. We can think of $R_{\mathfrak{p}}$ as embedded in $K$.

Suppose $\alpha \in K$ is integral over $R_{\mathfrak{p}}$. That is, $\alpha$ satisfies a minimum polynomial $x^n + a_{n-1}x^{n-1} + \cdots + a_1x + a_0$ for some $a_i \in R_{\mathfrak{p}}$.

Clearing denominators we get an equation $sx^n + a_{n-1}'x^{n-1} + \cdots + a_1'x + a_0' = 0$ for some $s \in R - \mathfrak{p}$ and $a_i' \in R$.

If we multiply through by $s^{n-1}$ we see that $s\alpha$ is integral over $R$.

As $R$ is integrally closed, $s\alpha \in R$. Thus $\alpha \in R_{\mathfrak{p}}$.

As $\alpha$ was arbitrary, this shows that $R_{\mathfrak{p}}$ is integrally closed. $\qed$

\subsection{Valuation rings}

\begin{definition}
A \emph{valuation ring} is an integral domain $R$ such that for every element $x$ of its field of fractions $K$, either $x$ or $1/x$ belongs to $R$.
\end{definition}

\begin{theorem}
Let $R$ be an integral domain with field of fractions $K$. The following are equivalent.
\begin{enumerate}
\item $R$ is a valuation ring
\item The set of principal ideals of $R$ is totally ordered by inclusion
\item The set of ideals of $R$ is totally ordered by inclusion
\item $R$ is a local ring and every finitely generated ideal of $R$ is principal
\end{enumerate}
\end{theorem}

\textbf{Proof:}
(1) implies (2): Suppose $R$ is a valuation ring.

For nonzero elements $r, s \in R$ we can write $r \geq s$ if $r/s \in R$. Since $r/s \in K$ either $r/s \in R$ or $s/r \in R$. Thus either $r \geq s$ or $s \geq r$.

If $r \geq s$ and $s \geq r$ then $r/s \in R$ and $s/r \in R$ so that $r/s$ is invertible in $R$, i.e. $r = su$ for some $u \in R^\star$, i.e. $(r) = (s)$.

Thus there is an ordering on the set of principal ideals of $R$ induced by the ordering on $R$ and this ordering is total.

This ordering is the same as that given by inclusion, since if $r/s \in R$ then $r = st$ for some $t \in R$ so that $(r) \supseteq (s)$.

(2) implies (3): Let $I, J$ be ideals of $R$ with $I$ not contained in $J$. Choose some $r \in I\\J$. Let $b \in J$.

Since $a \notin J$ we have $a \notin (b)$. By assumption this means that $(b) \subseteq (a) \subseteq I$. 

But $b$ was arbitrary in $J$, thus $J \subseteq I$.

As $I$ and $J$ were arbitrary ideals of $R$, the ideals of $R$ are totally ordered by inclusion.

(3) implies (4): If the set of ideals of $R$ is totally ordered, there can only be one maximal ideal. Thus $R$ is local. 

To show that every finitely generated ideal is principal it is enough to show that any ideal generated by two elements is in fact principal. 

But if $I = (r, s)$ then either $(r) \subseteq (s)$ or $(s) \subseteq (r)$, so $I = (r)$ or $I = (s)$.

(4) implies (1): Suppose $r, s \in R$ and $M$ is the maximal ideal of $R$. Let $I = (r, s)$.

It easy to show that if $I$ is a principal ideal and $M$ is a maximal ideal of $R$ then $I/IM$ is an $R/M$-vector space. 

If $I = (x)$ then $\bar{x} = x + IM$ spans $I/IM$. We claim $\{\bar{x}\}$ is a basis of $I/IM$.

If not then $I/IM$ is dimension $0$ over $R/M$, which means $I \subseteq IM$, i.e. $x \in xM$ which means that $1 \in M$. This is a contradiction since $M$ is maximal.

Thus $\{\bar{x}\}$ is a basis for $I/IM$ over $R/M$, i.e. $I/IM$ is a $1$-dimensional vector space over $R/M$.

This means that the images of $r$ and $s$ in $I/IM$ are linearly dependent over the field $k = R/M$, i.e. if $\bar{r}$ and $\bar{s}$ are the images of $r$ and $s$ in $I/IM$ there exist $\bar{u}$ and $\bar{v}$ in $R/M$ not both zero such that $\bar{u}\bar{r} + \bar{v}\bar{s} = 0$.

Thus there exist elements $u$ and $v$ of $R$ such that $ur + vs \in IM$ with $u$ and $v$ not both in $M$. Therefore there exist $x, y \in M$ such that $ur + vs = xr + ys$.

We then have $r(u - x) = s(y - v)$.

As $u$ and $v$ are not both in $M$, one of them is a unit in $R$. Suppose for example that $u$ is a unit. Then so is $u - x$. Then $r/s \in R$.

As $r$ and $s$ were chosen arbitrarily in $R$, every element $r/s$ of $K$ is either in $R$ or its inverse is. $\qed$

\subsection{Discrete valuation rings}

\begin{definition}
A \emph{discrete valuation ring} is a principal ideal domain that is a local ring but not a field.
\end{definition}

\begin{theorem} \label{locprimemax}
Suppose that the localisation of an integral domain $R$ at a prime ideal $\mathfrak{p}$ is a discrete valuation ring. Then the maximal ideal of $R_{\mathfrak{p}}$ is its only prime ideal.
\end{theorem}

\begin{proof}
Since a discrete valuation ring is principal, all prime ideals are maximal. $\qed$
\end{proof}

\begin{theorem} \label{localded}
Suppose $R$ is an integral domain. Then the following are equivalent.
\begin{enumerate}
\item $R$ is a noetherian valuation ring
\item $R$ is a discrete valuation ring
\item $R$ is a local, noetherian ring and its maximal ideal is generated by a single element
\item $R$ is a local ring and a Dedekind domain
\end{enumerate}
\end{theorem}

\begin{proof}
(1) imples (2): This follows since every finitely generated ideal is principal in a valuation ring, and a valuation ring is local.

(2) implies (1): We have that every ideal is principal and that $R$ is local. Clearly $R$ is noetherian.

But a local ring is a valuation ring iff every finitely generated ideal is principal. Thus $R$ is a valuation ring.

(2) implies (3): This follows since a principal ideal domain is noetherian.

(3) implies (2): Suppose that the maximal ideal $M$ of $R$ is generated by $\pi$. 

Let $I$ be an arbitrary ideal of $R$. 

Suppose that $I \subseteq \cap M^i$. Let $0 \neq a \in I$. As $a \in M^i$ then $a = \pi^i a_i$ for some $a_i \in R$.

As $R$ is an integral domain we have that $a_i = \pi a_{i+1}$ for all $i$. Thus $(a_0) \subset (a_1) \subset (a_2) \subset \cdots$ is an ascending chain of ideals which doesn't stabilise.

But this is impossible as $R$ is noetherian. So $a = \pi^n a_n$ for some unit $a_n$ of $R$.

Let $m$ be the minimum such value of $n$ over all the elements $a$ of $I$. Then $(\pi)^m \subseteq I$. Yet every element of $I$ is in $(\pi)^m$ and so the reverse inclusion also holds.

Thus $I = (\pi)^m$. In other words, an arbitrary ideal of $R$ is principal (a power of the maximal ideal, even).

(2) implies (4): This follows since a principal ideal domain is Dedekind.

(4) implies (3): Let $R$ be a local Dedekind domain with maximal ideal $P$. Find $p \in P$ such that $(p) \subseteq P$ is maximal amongst principal ideals contained in $P$. We will show that $(p) = P$.

Suppose to the contrary that $P \neq (p)$. Since $R$ is a Dedekind domain, the ideal $(p)$ contains a product of prime ideals. And since $R$ is a local ring, it must contain a power of $P$. Let $P^r$ be the smallest power of $P$ contained in $(p)$. Thus there exists $a \in P^{r-1}$ with $a \notin (p)$. 

Clearly in the field of fractions of $R$ we have that $a/p \notin R$.

As $a \in P^{r-1}$, we have that $aP \subseteq P^r \subseteq (p)$. Thus $(a/p)P \subseteq R$. It is easy to see that $(a/p)P$ is an ideal of $R$. 

There are now two cases. Either $(a/p)P = R$ or $(a/p)P \subseteq P$.

If $(a/p)P = R$ then $1 = ax/p$ for some $x \in P$. But then $(p) \subseteq (x)$. As $(p)$ is maximal amongst principal ideals contained in $P$, we must have $(x) = (p)$. Thus $x = cp$ for some $c \in R$, and so $1 = ax/p = ac$.

But this implies $a$ is a unit and since $aP \subseteq (p)$, we have that $P \subseteq (p)$. As $(p) \subseteq P$ we must have $P = (p)$.

On the other hand, if $(a/p)P \subseteq P$ then $P$ is a finitely generated $R[a/p]$-module. Clearly $(a/p)P \neq 0$, otherwise $aP = 0$, which is impossible, since $a \notin (p)$ and $R$ is an integral domain. Thus $P$ is a faithful, finitely generated $R[a/p]$-module and so $a/p$ is integral over $R$.

But as $a/p \notin R$, this contradicts our assumption that $R$ is integrally closed. Thus $(p) = P$ after all. $\qed$
\end{proof}



\begin{theorem}
If $R$ is a Dedekind domain and $\mathfrak{p}$ is a prime ideal of $R$ then $R_{\mathfrak{p}}$ is a discrete valuation ring.
\end{theorem}

\begin{proof}
We have shown that $R_{\mathfrak{p}}$ must be noetherian (\ref{locnoeth}), integrally closed (\ref{locintclosed}) and the only prime ideal is maximal (\ref{locprimemax}). Thus $R_{\mathfrak{p}}$ is also a Dedekind domain.

But a local domain that is Dedekind is a discrete valuation ring (\ref{localded}). $\qed$
\end{proof}

Discrete valuation domains have an especially simple structure, which means that Dedekind domains can be easily understood locally.

\begin{theorem}
If $R$ is a discrete valuation domain then there exists a prime element $\pi \in R$ such that every element of $R$ can be expressed in the form $u\pi^n$ for some unit $u$.
\end{theorem}

\begin{proof}
As $R$ is a local Dedekind domain and a principal ideal domain, it has a unique prime ideal $P$, which is principal. Let it be generated by $\pi \in R$, say.

Suppose that $p$ generates $\pi$ and that $p \mid ab$ for $a, b \in R$. Then $(p) \mid (a)(b)$, and since $(p) = P$ is a prime ideal, either $(p) \mid (a)$ or $(p) \mid (b)$.

Without loss of generality, suppose $(p) \mid (a)$. Then $(p) \supseteq (a)$ and so $a = pa'$ for some $a' \in R$. Thus $p \mid a$. This shows that $p$ is a prime element of $R$.

Now suppose that $(c)$ is an arbitrary ideal of $R$. By unique factorisation, it is a product of prime ideals, of which there is just one, namely $P$. In other words, $(c) = (p)^k$ for some nonnegative integer $k$.

Thus $c \in (p^k)$, i.e. $c = up^k$ for some unit $u \in R$. $\qed$
\end{proof}


\end{document}
