\documentclass[10pt]{article}
\usepackage{amsfonts}
\usepackage{amssymb, amsmath}
\usepackage{eucal}
\usepackage{amscd}
\usepackage{url}
\usepackage{listings}
\urlstyle{sf}
\pagestyle{plain}

\newcommand{\Z}{\mathbb{Z}}
\newcommand{\N}{\mathbb{N}}
\newcommand{\Q}{\mathbb{Q}}
\newcommand{\I}{\mathbb{I}}
\newcommand{\C}{\mathbb{C}}
\newcommand{\R}{\mathbb{R}}
\newcommand{\Pee}{\mathbb{P}}
\newcommand{\EuO}{\mathcal{O}}
\newcommand{\Qbar}{\overline{\mathbb{Q}}}
\newcommand{\code}{\lstinline}

\newcommand{\ljk}[2]{\left(\frac{#1}{#2}\right)}
\newcommand{\modulo}[1]{\;\left(\mbox{mod}\;#1\right)}
\newcommand{\fr}{\mathfrak}

\def\notdivides{\mathrel{\kern-3pt\not\!\kern4.5pt\bigm|}}
\def\nmid{\notdivides}
\def\nsubseteq{\mathrel{\kern-3pt\not\!\kern2.5pt\subseteq}}

\newtheorem{theorem}{Theorem}[section]
\newtheorem{lemma}[theorem]{Lemma}
\newtheorem{proposition}[theorem]{Proposition}
\newtheorem{corollary}[theorem]{Corollary}
\newtheorem{property}[theorem]{Property}

\newenvironment{proof}[1][Proof]{\begin{trivlist}
\item[\hskip \labelsep {\itshape #1}]}{\end{trivlist}}
\newenvironment{definition}[1][Definition]{\begin{trivlist}
\item[\hskip \labelsep {\bfseries #1}]}{\end{trivlist}}
\newenvironment{example}[1][Example]{\begin{trivlist}
\item[\hskip \labelsep {\bfseries #1}]}{\end{trivlist}}
\newenvironment{remark}[1][Remark]{\begin{trivlist}
\item[\hskip \labelsep {\bfseries #1}]}{\end{trivlist}}
\newenvironment{exercise}[1][Exercise]{\begin{trivlist}
\item[\hskip \labelsep {\bfseries #1}]}{\end{trivlist}}

\parindent=0pt
\parskip 4pt plus 2pt minus 2pt 

\title{Group Theory}

\author{William B. Hart}

\begin{document}

\maketitle

\tableofcontents

\section{Introduction to Groups and Subgroups}

\subsection{Groups and their Basic Properties}

\subsubsection{Groups}

\begin{definition}
A set $G$ with a closed binary operation $\star : G \times G \to G$ is a \emph{group} if the following rules hold:
\begin{enumerate}
\item (Associativity) For all $a, b, c \in G$ we have $(a \star b) \star c = a \star (b \star c)$.
\item (Identity) There exists $e \in G$ such that for all $g \in G$ we have $g\star e = g = e \star g$.
\item (Inverses) For every $g \in G$ there exists $g' \in G$ such that $g \star g' = e = g' \star g$.
\end{enumerate}
We sometimes write $(G, \star)$ to denote the group, in order to emphasise the group operation. In other cases where the operation is clear from the context, we just write $G$.
\end{definition}

We call the element $e$ in the definition above the \emph{identity} (element) of the group and $g'$ the \emph{inverse} of $g$.

In the case that the group operation is addition, we usually write $0$ for the identity element and $-g$ for the inverse of $g$.

When the group operation is multiplication, we usually write $1$ for the identity element and $g^{-1}$ for the inverse of $g$.

Implicit in the definition of a group is the closure property:

\begin{enumerate}
\item (Closure) For all $a, b \in G$ we have $a\star b \in G$.
\end{enumerate}

\subsubsection{Basic Results}

To save on notation we will write $g\star h$ as $gh$ where this will not cause any confusion.
\begin{theorem}
The identity element $e$ of a group $G$ is unique.
\end{theorem}

\begin{proof}
Clear. $\square$
\end{proof}

\begin{theorem}
The inverse of an element $g$ of a group $G$ is unique.
\end{theorem}

\begin{proof}
Clearly $gg'' = e$. The result follows after left multiplication by $g'$. $\square$
\end{proof}

\begin{theorem}
If $ab = e$ in a group $G$ then $b = a^{-1}$.
\end{theorem}

\begin{proof}
The result follows from $ab = e$ after left multiplication by $a^{-1}$. $\square$
\end{proof}

\begin{theorem}\label{idempotent_is_identity}
Suppose $G$ is a group with identity $e$. If $a \in G$ is idempotent i.e. $aa = a$, then $a = e$.
\end{theorem}

\begin{proof}
The result follows after right multiplication of $a^2 = a$ by $a^{-1}$. $\square$
\end{proof}

\begin{theorem} \label{product_inverse}
Suppose $G$ is a group and $a, b \in G$. Then $(ab)^{-1} = b^{-1}a^{-1}$.
\end{theorem}

\begin{proof}
Clear. $\square$
\end{proof}

\subsubsection{Right Identity and Inverses Suffice for a Group}

In the definition of a group, it isn't necessary to assume that $e$ is both a left and right identity or that $g'$ is both a left and right inverse of an element $g$. The following theorem shows that a right identity and right inverses suffice in order to have a group.

\begin{theorem}
Let $G$ be a set closed under a binary, associative operation. Suppose that:
\begin{enumerate}
\item (Right identity) There exists $e \in G$ such that $g\star e = g$ for all $g \in G$.
\item (Right inverse) For every $g \in G$ there exists $g' \in G$ such that $g\star g' = e$.
\end{enumerate}
Then $e\star g = g$ and $g'\star g = e$ and $(G, \star)$ is a group.
\end{theorem}

\begin{proof}
Assume $e$ is a right identity and every element $g$ has a right inverse $g'$.

Theorem \ref{idempotent_is_identity} only relies on right inverses and a right identity, so it holds in the context of the theorem.

It is easy to check that $g'g$ is idempotent. Thus $g'g = e$, i.e. $g'$ is a left inverse.

That $eg = g$ follows from $e = gg^{-1}$. $\square$
\end{proof}

\subsubsection{Cancellation in a Group}

\begin{theorem}
Let $G$ be a group and $a, b, c \in G$. If $ab = ac$ then $b = c$. Similarly if $ac = bc$ then $a = b$.
\end{theorem}

\begin{proof}
These follow by left (resp. right) multiplication by $a^{-1}$. $\square$
\end{proof}

\subsubsection{The Laws of Exponents in a Group}

Just as we do in arithmetic, it is convenient to define integer powers of elements of a group $G$.

\begin{definition}
Let $G$ be a group, $g\in G$ and $n$ a positive integer. We define:

\begin{align*}
   g^n &= \underbrace{g g \cdots g}_{n \:\mathrm{factors}}\\
g^{-n} &= \underbrace{g^{-1} g^{-1} \cdots g^{-1}}_{n \:\mathrm{factors}}\\
   g^0 &= e
\end{align*}
\end{definition}

All of the usual laws of exponents hold.

\begin{theorem}
If $G$ is a group and $g \in G$ then for all $m, n \in \mathbb{Z}$ we have
$$g^m g^n = g^{m + n}.$$
\end{theorem}

\begin{proof}
Clear. $\square$
\end{proof}

\begin{theorem}
Let $G$ be a group and $g \in G$. Then for any $m, n \in \mathbb{Z}$ we have
$$(g^{m})^n = g^{mn}.$$
\end{theorem}

\begin{proof}
Clear. $\square$
\end{proof}

When the group operation is addition, we write $ng$ for the sum of $n$ copies of $g$. Similarly $-ng$ corresponds to the sum of $n$ copies of $-g$. Similar results to the above hold, i.e. $mg + ng = (m + n)g$ and $m(ng) = (mn)g$.

\subsubsection{The Order of a Group}

\begin{definition}
The cardinality of a group is called its \emph{order}. We denote the order of a group $G$ by $|G|$.
\end{definition}

\begin{definition}
If a group has finite order we say that it is a \emph{finite group}. Otherwise we say that it is an \emph{infinite group}.
\end{definition}

The smallest number of elements a group can have is $1$, since a group must contain an identity element.

\begin{theorem}
A set $G$ containing just the identity element is a group under any closed binary operation on $G$.
\end{theorem}

\begin{proof}
Clear. $\square$
\end{proof}

\subsection{The Order of Elements of a Group}

\subsubsection{The Order of Elements of a Group}

\begin{definition}
Let $G$ be a group and $g \in G$. If there is a positive integer $n$ such that $g^n = e$, the least such $n$ is called the \emph{order} of $g$, denoted ord$(g)$. If no such $n$ exists, the element $g$ is said to have \emph{infinite order}.
\end{definition}

If the group is written additively the order of an element $g$ is the least positive integer $n$ such that $ng = 0$.

\begin{theorem}
The order of an element $g$ in a group $G$ is $1$ iff it is the identity.
\end{theorem}

\begin{proof}
Clear. $\square$
\end{proof}

\subsubsection{Basic Results about the Order Elements in a Group}

\begin{theorem}
Let $G$ be a group and $g \in G$. Then ord$(g) =$ ord$(g^{-1})$.
\end{theorem}

\begin{proof}
The result follows easily from $(g^{-1})^n = (g^n)^{-1}$. $\square$.
\end{proof}

\begin{theorem}
Let $G$ be a group and $a, b \in G$. Then ord$(ab) =$ ord$(ba)$. 
\end{theorem}

\begin{proof}
The result follows easily from $(ab)^n = a(ba)^{n-1}b$. $\square$
\end{proof}

\begin{theorem}
If $G$ is a group and $g \in G$ has finite order $n$, then $g^0$, $g^1$, $\ldots$, $g^{n - 1}$ are all of the distinct powers of $g$.
\end{theorem}

\begin{proof}
First show that all powers $g^a$ are equivalent to one of the given powers. By the division algorithm
$$a = nq + r$$
for some $0 \leq r < n$. It follows easily that $g^a = g^r.$

To show they are distinct, suppose to the contrary that $g^i = g^j$ for $0\leq i < j < n$. Multiplication by $g^{-i}$ contradicts the minimality of $n$. $\square$
\end{proof}

\begin{theorem}
In a finite group $G$, all elements have finite order.
\end{theorem}

\begin{proof}
If $g \in G$ is of infinite order then it is easy to see that all powers of $g$ are distinct. $\square$
\end{proof}

\subsubsection{Groups with every Element of Finite Order can be Infinite}

\begin{theorem}
There are infinite groups in which all elements have finite order.
\end{theorem}

\begin{proof}
Consider an alphabet $L$ of infinitely many distinct letters $\{g_1, g_2, \ldots\}$. Now consider the set $W$ of all finite words (including the empty word) made from these letters in which
\begin{enumerate}
\item Each letter occurs only once.
\item The letters are in alphabetical order, i.e. $g_i$ appears before $g_j$ if $i < j$.
\end{enumerate}

Define a binary operation on $W$ as follows. For $w_1, w_2 \in W$, form the word that is obtained by taking all the letters that occur in either $w_1$ or $w_2$, but not both, and putting them in alphabetical order.

It is easy to show that $W$ is a group under this operation and $w^2 = e$ for all $w \in W$. But $g_1, g_2, \ldots \in W$ are infinitely many distinct elements of $W$. $\square$
\end{proof}

\subsection{Subgroups}

\subsubsection{Subgroups}

\begin{definition}
A subset $H$ of a group $G$ which is a group under the same operation as $G$ when restricted to the elements of $H$ is called a \emph{subgroup} of $G$. We write $H \leq G$.
\end{definition}

\begin{theorem}
If $e$ is the identity of a group $G$ and $H \leq G$ then $e$ is also the identity of $H$.
\end{theorem}

\begin{proof}
The identity of $H$ is idempotent in $H$ and therefore also in $G$. $\square$
\end{proof}

\begin{theorem}
If $H \leq G$ and $g^{-1}$ is the inverse of $g$ in $H$ then it is also the inverse of $g$ in $G$.
\end{theorem}

\begin{proof}
This follows immediately from the uniqueness of inverses and the previous theorem. $\square$
\end{proof}

\begin{theorem}
A non-empty subset $H$ of a group $G$ is a subgroup iff it is closed under the group operation of $G$ and $h^{-1} \in H$ for all $h \in H$.
\end{theorem}

\begin{proof}
Clear. $\square$
\end{proof}

\begin{corollary} \label{subgroup}
A non-empty subset $H$ of a group $G$ is a subgroup iff $ab^{-1} \in H$ for every pair of elements $a, b \in H$.
\end{corollary}

\begin{proof}
Let $H$ satisfy the given conditions. As $H$ is non-empty it contains an element $a$. Therefore by assumption $aa^{-1} = e \in H$ and $ea^{-1} = a^{-1} \in H$.

Closure follows by taking $a = g$ and $b = h^{-1}$ for arbitrary $g, h \in H$. Thus $H$ is a subgroup by the previous theorem.

The converse is clear. $\square$
\end{proof}

\subsubsection{The Intersection and Union of two Subgroups}

\begin{theorem}
If $G$ is a group and $H, K \leq G$ then $H\cap K$ is a subgroup of $G$.
\end{theorem}

\begin{proof}
Clear. $\square$
\end{proof}

\begin{theorem}
If $G$ is a group and $H, K \leq G$ then $H\cup K$ is a subgroup of $G$ iff $H \leq K$ or $K \leq H$.
\end{theorem}

\begin{proof}
Let $H \cup K$ be a subgroup of $G$. Suppose $H \not\subseteq K$ and $K \not\subseteq H$.

This implies that there exists $h \in H$ with $h \notin K$ and $k \in K$ with $k \notin H$. We have that $h, k \in H\cup K$ and it is easy to see that $hk \in H$ or $hk \in K$.

In the first case, $k = h^{-1}(hk) \in H$ which is a contradiction. The second case also gives a contradiction. Thus our assumption is false and either $H \leq K$ or $K \leq H$.

The converse is clear. $\square$
\end{proof}

\subsubsection{The Product of Two Subgroups}

\begin{definition}
If $G$ is a group and $H, K \leq G$ we define:
$$HK = \{hk \;|\; h \in H, K \in K\}.$$
\end{definition}

Note that $HK$ is just a set at this point. In fact we will show the following.

\begin{theorem}\label{HKKH}
If $G$ is a group and $H, K \leq G$ then $HK \leq G$ iff $HK = KH$.
\end{theorem}

\begin{proof}
First suppose that $HK = KH$. Clearly $HK$ is non-empty as $1 \in HK$.

Suppose $h_1, h_2 \in H$ and $k_1, k_2 \in K$. Then as $HK = KH$ we have that $k_1h_2 = h'k'$ for some $h' \in H$ and $k' \in K$. It follows that
$(h_1k_1)(h_2k_2) = (h_1h')(k'k_2) \in HK$. Thus $HK$ is closed under the group operation.

That inverses exist in $HK$ follows from
$$(hk)^{-1} = k^{-1}h^{-1} \in KH = HK.$$
Thus $HK$ is a subgroup of $G$ by Theorem \ref{subgroup}.

Conversely, suppose that $HK \leq G$ and let $hk$ be an arbitrary element of $HK$ for some $h \in H$ and $k \in K$.

It is easy to see that $(hk)^{-1} \in KH$. But $x \to x^{-1}$ is a bijection on the elements of $HK$, thus $HK \subseteq KH$.

The reverse inclusion is easy to show since $k, h \in HK$ for all $k \in K$ and $h \in H$. $\square$
\end{proof}

\subsubsection{The Order of the Product of Two Subgroups}

\begin{theorem}
If $G$ is a group and $H, K \leq G$ both of finite order then:
$$|HK| = \frac{|H||K|}{|H\cap K|}.$$
\end{theorem}

\begin{proof}
There are $|H||K|$ not necessarily distinct products $hk$ with $h \in H$ and $k \in K$. We claim there are precisely $|H\cap K|$ ways of writing a given element $hk$ as a product of an element of $H$ by an element of $K$.

For each $z \in H\cap K,$
$$hk = (hz)(z^{-1}k) \in HK.$$

Thus there are at least $|H\cap K|$ distinct such ways of writing $hk$.

Conversely, note that if $hk = h'k'$ for $h' \in H$ and $k' \in K$ then
$$h^{-1}h' = k(k')^{-1} \in H\cap K.$$

Writing $h^{-1}h' = z = k(k')^{-1}$ we have that $h'k' = (hz)(z^{-1}k)$.

Therefore every product $h'k'$ is of this form which settles our claim.

As $H\cap K$ is non-empty, $|H\cap K| \neq 0$ and the result follows. $\square$
\end{proof}

\subsubsection{The Subgroup Generated by a Subset}

\begin{definition}
Let $G$ be a group and $S \subseteq G$. The \emph{subgroup generated} by $S$ is defined to be the intersection of all subgroups of $G$ containing $S$. We denote it $\langle S \rangle$. If $S = \{g_1, g_2, \ldots, g_n\}$ we usually write $\langle g_1, g_2, \ldots, g_n \rangle$.
\end{definition}

The subgroup generated by $S$ is the smallest subgroup of $G$ containing $S$ in the sense that it is contained in any subgroup of $G$ containing $S$.

It is clearly also the smallest subset of $G$ containing $S$ which is closed under taking inverses and closed under the group operation. In fact, we have the following theorem.

\begin{theorem}
Let $G$ be a group and $S$ a subset of $G$. Let $S' = \{s^{-1} \;|\; s \in S\}$. Then $\langle S \rangle$ is the set of all finite products of elements of $S \cup S'$.
\end{theorem}

\begin{proof}
Let the given set be denoted $T$. Then $T$ is clearly closed under the group operation. But if $a \in T$, it has the form $a = s_1s_2\ldots s_n$ for some $s_i \in S\cup S'$. The inverse of $a$ is also clearly in $T$. Thus $T$ is a subgroup of $G$.

But all of the elements of $G$ are contained in any subgroup of $G$ containing $S$. Thus $T = \langle S \rangle$. $\square$
\end{proof}

\subsubsection{The Commutator Subgroup of a Group}

\begin{definition}
Let $G$ be a group and $g, h \in G$. We define the \emph{commutator} of $g$ and $h$ to be $[g, h] = g^{-1}h^{-1}gh$. The subgroup of $G$ generated by the commutators of $G$ is called the \emph{commutator subgroup} of $G$ and is denoted $[G, G]$.
\end{definition}

Note that the commutator subgroup may contain elements that are not commutators.

\begin{theorem}
If $G$ is a group and $g, h \in G$ then $[g, h] = 1$ iff $g$ and $h$ commute.
\end{theorem}

\begin{proof}
Clear. $\square$
\end{proof}

\begin{theorem}
If $G$ is a group and $g, h \in G$ then $[g, h] = hg[h, g]$.
\end{theorem}

\begin{proof}
Clear. $\square$
\end{proof}

\begin{corollary}
If $G$ is a group then every $h \in [G, G]$ can be written as a finite product of commutators.
\end{corollary}

\begin{proof}
Clear.
\end{proof}

\begin{theorem}
A group $G$ is abelian iff it has only one distinct commutator.
\end{theorem}

\begin{proof}
We always have $1 = [1, 1]$. Thus if a group contains only one distinct commutator, $[g, h] = 1$ for all $g, h \in G$ which implies it is abelian. The converse is also clear. $square$
\end{proof}

\begin{theorem}
If $G$ is a group and $g, h \in G$ then $[g, h]^{-1} = [h, g]$.
\end{theorem}

\begin{proof}
Clear.
\end{proof}

\begin{theorem}
If $G$ is a group and $g, h, s \in G$ then $[g, h]^s = [g^s, h^s]$ where $g^s = s^{-1}gs$.
\end{theorem}

\begin{proof}
We have
$$[g, h]^s = s^{-1}g^{-1}h^{-1}ghs = s^{-1}g^{-1}ss^{-1}h^{-1}ss^{-1}gss^{-1}hs = [g^s, h^s].$$
$\square$
\end{proof}

\subsection{Abelian Groups}

\subsubsection{Abelian Groups}

\begin{definition}
A group $G$ with operation $*$ is said to be \emph{abelian} if the following rule holds
\begin{itemize}
\item (Commutativity) $ab = ba$ for all $a, b \in G$.
\end{itemize}
\end{definition}

We often write the group operation in an abelian group as $+$ as the number systems $\Z$, $\Q$, $\R$ and $\C$ all form abelian groups under $+$.

A familiar example of a group which is not abelian is the group of invertible $n\times n$ matrices under multiplication. 

\subsubsection{Basic Theorems about Abelian Groups}

We give some criteria for a group to be abelian.

\begin{theorem}
Let $G$ be a group. If $(ab)^2 = a^2b^2$ for all $a, b \in G$ then $G$ is abelian.
\end{theorem}

\begin{proof}
If $(ab)^2 = a^2b^2$, then $a(ba)b = a(ab)b$ from which the result follows by cancellation. $\square$
\end{proof}

\begin{theorem}
A group $G$ is abelian iff $(ab)^{-1} = a^{-1}b^{-1}$ for all $a, b \in G$.
\end{theorem}

\begin{proof}
We have $(ab)^{-1} = b^{-1}a^{-1}$ by Theorem \ref{product_inverse}. If $G$ is abelian this implies $(ab)^{-1} = a^{-1}b^{-1}$.

Assuming the latter, the converse follows from $b^{-1}a^{-1} = (ab)^{-1} = a^{-1}b^{-1}$ and the bijection $z \leftrightarrow z^{-1}$ between elements of $G$. $\square$
\end{proof}

\begin{theorem}
If $G$ is a group in which $a = a^{-1}$ for all $a \in G$ then $G$ is abelian.
\end{theorem}

\begin{proof}
Suppose $a = a^{-1}$ for all $a \in G$. Then $ab = a^{-1}b^{-1} = (ba)^{-1} = ba$ for all $b \in G$. $\square$
\end{proof}

\subsubsection{The Order of a Nonabelian Group is at least 6}

\begin{theorem}
If $G$ is a non-abelian group then $|G| \geq 6$.
\end{theorem}

\begin{proof}
Let $G$ be a non-abelian group and suppose that $ab \neq ba$ for distinct $a, b \in G$.

Clearly $a = 1$ and $b = 1$ lead to a contradiction. Thus $a, b, 1$ are distinct.

By application of cancellation we have that $ab \neq a$, $ab \neq b$. If $ab = 1$ then $a$ and $b$ are inverse which contradicts $ab \neq ba$. 

Similar arguments apply for $ba$ and we have that $1$, $a$, $b$, $ab$, $ba$ are distinct. Suppose that these are the only elements in $G$.

Firstly, note $a^2$ cannot equal $ab$, $ba$ or $a$, by cancellation. We can't have $a^2 = b$ otherwise $ab = ba$. Therefore $a^2 = 1$.

The element $aba$ cannot equal $a$, $ab$ or $ba$ by cancellation. We also can't have $aba = 1$ since multiplication by $a$ on the left leads to a contradiction. Similarly we can't have $aba = b$.

Thus $aba$ is distinct from all five elements in the group and the group must have order at least $6$. $\square$
\end{proof}

\subsubsection{The Quaternion Group $Q_8$}

We give an example of a finite non-abelian group.

\begin{definition}
The quaternion group $Q_8$ is the multipicative group containing eight elements $\{1, -1, i, -i, j, -j, k, -k\}$ which satisfy the relations $i^2 = j^2 = k^2 = -1$, $(-1)^2 = 1$ and $ijk = -1$.
\end{definition}

The remaining entries in the multiplication table for $Q_8$ can be worked out from the given relations:
$$jk = -i^2jk = -i(ijk) = i.$$
and
$$ij = -ijk^2 = -(ijk)k = k,$$
and
$$ki = -kij^2 = -k(ij)j = -k^2j = j.$$

We also have:
$$kj = k(ki) = k^2i = -i,$$
and
$$ji = j(jk) = j^2k = -k,$$
and
$$ik = i(ij) = i^2j = -j.$$

Note that $ij \neq ji$, $jk \neq kj$ and $ik \neq ki$, so $Q_8$ is nonabelian.

\subsection{Cyclic Groups}

\subsubsection{Cyclic Groups}

\begin{definition}
A group $G$ is said to be \emph{cyclic} if there exists an element $g \in G$ such that every element of the group is of the form $g^n$ for some $n \in \Z$. We call $g$ a \emph{generator} of $G$ and write $G = <g>$.
\end{definition}

The identity in a cyclic group is $g^0$ and the inverse of $g^n$ is $g^{-n}$.

\subsubsection{Basic Results Regarding Cyclic Groups}

\begin{theorem}
Cyclic groups are abelian.
\end{theorem}

\begin{proof}
Clear. $\square$
\end{proof}

\begin{theorem}
The order of a finite cyclic group $G = <g>$ is the smallest positive integer $n$ for which $g^n = 1$.
\end{theorem}

\begin{proof}
Let $n$ be as in the theorem. Clearly $g^a = g^b$ for all $a \equiv b \pmod{n}$. Therefore all elements of $G$ lie in $\{g^i \;|\; 0 \leq i < n\}$.

It is easy to see that no two such elements are equal. $\square$
\end{proof}

\begin{theorem}
The generator of an infinite cyclic group is unique.
\end{theorem}

\begin{proof}
Let $G = <g>$ be infinite cyclic and suppose that $g^a$ is also a generator for some $1 \neq a \in \Z$.

Then we must have $g = (g^a)^n = g^{an}$ for some $n \in Z$.

Multiplying by $g^{-1}$ we get $g^{an - 1} = 1$ which contradicts the infinite order of $g$. $\square$
\end{proof}

A generator of a finite cyclic group is not generally unique.

\begin{theorem}
Let $G = <g>$ be a cyclic group of order $n$. Then $g^a$ is a generator of $G$ iff $a$ coprime to $n$.
\end{theorem}

\begin{proof}
If gcd$(a, n) = d$ then clearly all powers of $g^a$ lie in $\{1, g^d, g^{2d}, \ldots, g^{n - d}\}$. Thus if $d \neq 1$ then $g^a$ doesn't generate $G$.

Conversely, if gcd$(a, n) = 1$ there exist $s, t \in Z$ such that $as + nt = 1$ by the extended Euclidean algorithm. As $g^n = 1$, this implies that $(g^a)^s = g$. Thus every power of $g$ is also a power of $g^a$. $\square$
\end{proof}

\begin{theorem}
A subgroup $H$ of a cyclic group $G = <g>$ is cyclic and is generated by the smallest positive power of $g$ in $H$.
\end{theorem}

\begin{proof}
Suppose that $g^a$ is the smallest positive power of $g$ in $H$. Clearly all powers of $g^a$ are in $H$.

No other power of $g$ is in $H$ by the division algorithm and the minimality of $a$. $\square$
\end{proof}

\subsection{The Centre and Centralizers}

\subsubsection{The Centre of a Group}

\begin{definition}
The \emph{centre} of a group $G$ is defined to be the subset
$$Z_G = \{g \in G \;|\; ga = ag \;\;\mbox{for all}\;\; a \in G\}.$$
\end{definition}

In other words, the centre is the set of all elements that commute with every element of $G$.

\begin{theorem}
The centre of a group $G$ is an abelian subgroup of $G$.
\end{theorem}

\begin{proof}
Clearly the identity of $G$ is in the centre of $G$. Therefore $Z_G$ is not empty.

If $g \in Z_G$ and $a \in G$ then $ga = ag$ for all $a \in G$. Multiplying by $g^{-1}$ on the left and right we see that $g^{-1} \in Z_G$.

It is easy to see that if $g, h \in Z_G$ then $gh \in Z_G$ and so $Z_G \leq G$. It is clear that $Z_G$ is abelian. $\square$
\end{proof}

\begin{theorem}
A group $G$ is abelian iff $Z_G = G$.
\end{theorem}

\begin{proof}
Clear.
\end{proof}

\subsubsection{The Centralizer of a Subset of a Group}

\begin{definition}
Let $G$ be a group and $S$ a nonempty subset of $G$. The \emph{centralizer} of $S$ in $G$ is the set
$$C_G(S) = \{g \in G \;|\; gs = sg \;\;\mbox{for all}\;\; s \in S\}.$$
When $S = \{g\}$ we write $C_G(S) = C_G(g)$.
\end{definition}

In other words, the centralizer of $S$ is the set of all elements that commute with $S$. The centralizer of $g \in G$ is the set of all elements that commute with $g$.

\begin{theorem}
Let $S$ be a subset of a group $G$. Then $C_G(S) \leq G$.
\end{theorem}

\begin{proof}
Clear.
\end{proof}

Note that $C_G(S)$ need not be abelian. For example, if $G$ is nonabelian, $C_G(1) = G$ which is not abelian.

\subsubsection{Basic Results Regarding Centralizers}

\begin{theorem}
Let $G$ be a group. Then $Z_G = \bigcap_{g \in G} C_G(g)$.
\end{theorem}

\begin{proof}
Clearly $Z_G \subseteq \bigcap_{g \in G} C_G(g)$.

For the reverse inclusion, if $x \in \bigcap_{g \in G} C_G(g)$ then $x$ commutes with every $g \in G$. Thus $g \in Z_G$. $\square$
\end{proof}

\begin{corollary}
If $G$ is a group and $S \subseteq G$ then $Z_G \subseteq C_G(S)$.
\end{corollary}

\begin{proof}
Clearly $C_G(S) = \bigcap_{s \in S} C_G(s)$. The result therefore follows from the theorem. $\square$
\end{proof}

\begin{theorem}
Let $G$ be a group. Then $C_G(Z_G) = G$.
\end{theorem}

\begin{proof}
Clear.
\end{proof}

\begin{theorem}
Let $G$ be a group and $S \subseteq T \subseteq G$, then $C_G(T) \leq C_G(S)$.
\end{theorem}

\begin{proof}
It suffices to show that $C_G(T) \subseteq C_G(S)$. But this is clear. $\square$
\end{proof}

\begin{theorem}
Let $G$ be a group and $H \leq G$. Then $H \subseteq C_G(H)$ iff $H$ is abelian.
\end{theorem}

\begin{proof}
Clear.
\end{proof}

\subsection{Conjugacy}

\subsubsection{Conjugate elements}

\begin{definition}
Let $G$ be a group. Two elements $a, b \in G$ are said to be \emph{conjugate} if $a = g^{-1}bg$ for some $g \in G$.
\end{definition}

\begin{theorem}
If $G$ is an abelian group then $a, b \in G$ are conjugate iff $a = b$.
\end{theorem}

\begin{proof}
Clear. $\square$
\end{proof}

\begin{theorem}
Let $G$ be a group. If $a, b \in G$ are conjugate then they have the same order.
\end{theorem}

\begin{proof}
Clear. $\square$
\end{proof}

\begin{theorem}
Let $G$ be a group. Conjugacy of elements is an equivalence relation on $G$.
\end{theorem}

\begin{proof}
Clear. $\square$
\end{proof}

\begin{definition}
Let $G$ be a group and $g \in G$. The class of all elements of $G$ that are conjugate to $g$ is called the \emph{conjugacy class} of $g$.
\end{definition}

Clearly conjugacy classes partition $G$.

\subsubsection{Conjugacy and Centralizers}

\begin{theorem}\label{conjugacy}
Let $G$ be a group. The elements of the conjugacy class of $g$ are in bijection with the left cosets of $C_G(g)$.
\end{theorem}

\begin{proof}
Suppose that $h$ and $h'$ are in the conjugacy class of $g$, i.e. $h = aga^{-1}$ and $h' = bgb^{-1}$ for some $a, b \in G$.

We have that $h = h'$ iff $(b^{-1}a)g(b^{-1}a)^{-1} = g$. But this holds iff $b^{-1}a \in C_G(g)$.

But $b^{-1}a \in C_G(a)$ iff $aC_G(g) = bC_G(g)$.

Writing $[g]$ for the conjugacy class of $g$ and $\mathcal{C}$ for the set of left cosets of $C_G(g)$, we can define a map $\phi : [g] \to \mathcal{C}$ by $h \mapsto aC_G(g)$ where $h = aga^{-1}$.

By what we have just shown, this is a well-defined map and it is clearly a bijection. $\square$
\end{proof}

\subsection{Cosets and Lagrange's Theorem}

\subsubsection{Cosets}

\begin{definition}
Let $G$ be a group and $H \leq G$. A \emph{left coset} is a set of the form
$$aH = \{ah \;|\; h \in H\},$$
for some $a \in G$. Similarly a \emph{right coset} is a set of the form
$$Ha = \{ha \;|\; h \in H\}.$$
\end{definition}

Clearly if $G$ is abelian, the left coset $aH$ coincides with the right coset $Ha$ for all $a \in G$.

\begin{theorem}
Let $G$ be a group and $H \leq G$. Two left (right) cosets of $H$ either coincide or are disjoint.
\end{theorem}

\begin{proof}
If $g \in aH$ then clearly $gH \in aH$. Similarly if $gH = aH$ then $g.1 \in aH$. A similar argument applies for right cosets. $\square$
\end{proof}

\begin{theorem}\label{cosets}
Let $G$ be a group and $H \leq G$. If $g \in aH$ then $aH = gH$. Similarly if $g \in Ha$ then $Ha = Hg$.
\end{theorem}

\begin{proof}
Clear, as $1 \in H$. $\square$
\end{proof}

Note that these theorems imply that the cosets of $H$ in $G$ partition $G$.

\begin{theorem}
Let $G$ be a group and $H \leq G$. There is a bijection between the left and right cosets of $H$ in $G$.
\end{theorem}

\begin{proof}
Define a map $\phi : gH \to Hg^{-1}$ from left to right cosets.

The map $\phi$ is well-defined, since $gH = g'H$ iff $g \in g'H$ iff $g = g'h$ for some $h \in H$ iff $g'^{-1} = hg^{-1}$ iff $Hg^{-1} = Hg'^{-1}$.

The map is clearly bijective. $\square$
\end{proof}

\begin{definition}
The number of cosets of $H$ in $G$, if it is finite, is called the \emph{index} of $H$ in $G$, written $[G:H]$. Otherwise we say that $H$ has infinite index in $G$ and write $[G:H] = \infty$.
\end{definition}

\begin{theorem}
Let $G$ be a group and $H \leq G$. There is a bijection between any two left (right) cosets of $H$ in $G$.
\end{theorem}

\begin{proof}
Define the map $\psi : gH \to hH$ which sends $a \mapsto hg^{-1}a$.

As the map is invertible, it is bijective. A similar argument holds for right cosets. $\square$
\end{proof}

\begin{corollary} \label{cosetsize}
If $G$ is a finite group and $H \leq G$ then $|gH| = |hH|$ for all $g, h \in G$.
\end{corollary}

\begin{proof}
Clear.
\end{proof}

\subsubsection{Lagrange's Theorem}

\begin{theorem} (Lagrange)
If $G$ is a group and $H \leq G$ then $|H| \;|\; |G|$.
\end{theorem}

\begin{proof}
Follows from Corollary \ref{cosetsize} and the fact that cosets partition a group. $\square$
\end{proof}

\begin{corollary}
If $G$ is a finite group and $H \leq G$ then $[G:H] = |G|/|H|$.
\end{corollary}

\begin{proof}
Clear.
\end{proof}

\begin{corollary}
If $G$ is a finite group, the order of $g$ must divide $G$.
\end{corollary}

\begin{proof}
The subgroup generated by $g$, $\langle g \rangle = \{g^i \;|\; i \in \Z\}$ is a subgroup of $G$. We have already shown that if the order of $g$ is $n$, then there are $n$ distinct powers of $g$, which it is easy to see are the elements of $\langle g \rangle$.

The result then follows from Lagrange's theorem. $\square$
\end{proof}

\begin{theorem}
If a group $G$ has prime order $p$, then it is cyclic.
\end{theorem}

\begin{proof}
Let $g$ be any element of $G$ other than the identity. The result follows from the fact that the order of $g$ must divide $p$. Thus $\langle g \rangle = G$. $\square$
\end{proof}

\begin{theorem}
Let $G$ be a finite group and $K \leq H \leq G$, then
$$[G:K] = [G:H][H:I].$$
\end{theorem}

\begin{proof}
$[G:K] = |G|/|K| = (|G|/|H|)(|H|/|K|) = [G:H][G:K]$. $\square$
\end{proof}

\begin{theorem} \label{gtoorder}
If $G$ is a finite group of order $n$ and $g \in G$ then $g^n = e$.
\end{theorem}

\begin{proof}
We have that the order of $g$ divides $n$. $\square$
\end{proof}

\begin{theorem} (Euler)
If gcd$(a, n) = 1$ then $a^{\varphi(n)} = 1$.
\end{theorem}

\begin{proof}
The function $\varphi(n)$ counts the number of integers $i$ in $0 \leq i < n$ such that $gcd(i, n) = 1$.

It is easy to see that this set of integers is a group under multiplication modulo $n$. Euler's theorem is an application of Theorem \ref{gtoorder}. $\square$
\end{proof}

\begin{corollary} (Fermat's Little Theorem)
If $p$ is a prime then $a^{p-1} \equiv 1 \pmod{p}$ for all $1 \leq a < p$.
\end{corollary}

\begin{proof}
This is Euler's theorem where $n = p$, as $\varphi(p) = p - 1$. $\square$
\end{proof}

\subsection{Normal Subgroups and Normalizers}

\subsubsection{Normal Subgroups}

\begin{definition}
Let $G$ be a group. A subgroup $N$ is said to be \emph{normal} if $gN = Ng$ for all $g \in G$. We write $N \mathrel{\unlhd} G$.
\end{definition}

\begin{theorem}\label{equivnormal}
Let $G$ be a group and $N \leq G$. The following are equivalent:
\begin{enumerate}
\item $N \mathrel{\unlhd} G$
\item $g^{-1}ng \in N$ for all $n \in N$ and $g \in G$
\item For all $g, h \in G$, $gh \in N$ iff $hg \in N$
\item For all $g \in G$, $g^{-1}Ng \subseteq N$
\item For all $g \in G$, $g^{-1}Ng = N$
\item All left cosets are right cosets and conversely
\item For all $x, y, g, h \in G$, $x \in gN$ and $y \in hN$ implies $xy \in (gh)N$
\end{enumerate}
\end{theorem}

\begin{proof}
(1)$\iff$ (2) We have that $gN = Ng$ iff for all $n \in N$, $ng \in gN$. Multiplication on the left by $g^{-1}$ gives the result.

(1)$\iff$ (3) We have that $gh \in N$ iff $h \in g^{-1}N$. This is equal to $Ng^{-1}$ for all $g \in G$ iff (1) holds. But $h \in Ng^{-1}$ iff $hg \in N$.

(1)$\iff$ (5) Clear.

(5)$\implies$ (4) Clear.

(4)$\implies$ (1) Clearly (4) implies $Ng \subseteq gN$. But replacing $g$ with $g^{-1}$ in (4) also yields $gN \subseteq Ng$. Thus we have (1).

(1)$\iff$ (6) Clear.

(1)$\implies$ (7) If $x \in gN$ and $y \in hN$, then (1) implies $xy \in gNhN = (gh)NN = (gh)N$.

(7)$\implies$ (1) Another way of writing (7) is that for $g, h \in G$ and $n, n' \in N$, $(gn)(hn') = (gh)n''$ for some $n'' \in N$. Multiplication on the left by $g^{-1}$ and on the right by the inverse of $n'$ yields a statement equivalent to (1). $\square$
\end{proof}

\begin{theorem}
All subgroups of an abelian group are normal.
\end{theorem}

\begin{proof}
Clear. $\square$
\end{proof}

\begin{theorem}
If $G$ is a group then the trivial subgroup $\{e\}$ and $G$ itself are normal subgroups of $G$.
\end{theorem}

\begin{proof}
Clear. $\square$
\end{proof}

\begin{theorem}
If $G$ is a group then $Z_G \mathrel{\unlhd} G$.
\end{theorem}

\begin{proof}
Clearly $g^{-1}zg = z$ for all $z \in Z_G$. $\square$
\end{proof}

\begin{theorem}
If $G$ is a group and $H \mathrel{\unlhd} G$ and $H \leq K \leq G$ then $H \mathrel{\unlhd} K$.
\end{theorem}

\begin{proof}
We have that $kH = Hk$ for all $k \in K \subseteq G$. $\square$
\end{proof}

\subsubsection{A Subgroup of Order $2$ is Normal}

\begin{theorem}
If $G$ is a group and $H \leq G$ with $[G:H] = 2$ then $H \mathrel{\unlhd} G$.
\end{theorem}

\begin{proof}
Let $H$ and $aH$ be the two distinct left cosets. This implies that $a \notin H$, which in turn implies that $Ha \neq H$.

As the number of left and right cosets is equal, the right cosets must be $H$ and $Ha$. But then it is easy to see every left coset of is a right coset and vice versa and so $H$ is normal in $G$. $\square$
\end{proof}

\subsubsection{Intersection and Products of Normal Subgroups}

\begin{theorem}
If $G$ is a group and $M, N \mathrel{\unlhd} G$ then $M \cap N \mathrel{\unlhd} G$.
\end{theorem}

\begin{proof}
We already know $M \cap N \leq G$. 

If $a \in M \cap N$ then $g{-1}ag \in M, N$ for all $g \in G$. Thus $M \cap N$ is normal in $G$. $\square$
\end{proof}

\begin{theorem}
If $G$ is a group and $M \leq G$ and $N \mathrel{\unlhd} G$ then $MN, NM \leq G$.
\end{theorem}

\begin{proof}
If $N \mathrel{\unlhd} G$ then $g^{-1}Ng \in N$ for all $g \in G$. In particular, this holds for all $g \in M$ and $Ng = gN$ for all $g \in M$, i.e. $NM = MN$. Thus by Theorem \ref{HKKH} $MN, NM \leq G$. $\square$
\end{proof}

\begin{theorem}
If $G$ is a group and $M, N \mathrel{\unlhd} G$ then $MN, NM \mathrel{\unlhd} G$.
\end{theorem}

\begin{proof}
By the previous theorem $MN, NM \leq G$.

If $a = nm \in NM$ for some $n \in N$ and $m \in M$ and $g \in G$ then $g^{-1}ag = (g^{-1}ng)(g^{-1}mg) \in NM$. Thus $NM \mathrel{\unlhd} G$. A similar argument shows that $MN \mathrel{\unlhd} G$. $\square$
\end{proof}

\begin{theorem}
If $G$ is a group and $H \leq G$ and $N \mathrel{\unlhd} G$ then $H\cap N \mathrel{\unlhd} H$.
\end{theorem}

\begin{proof}
Let $x \in H\cap N$ and $h \in H$. Then $h^{-1}xh \in H$.

We also have $h^{-1}xh \in N$ as $x \in N \mathrel{\unlhd} G$.

Thus $h^{-1}xh \in H\cap N$ for all $h \in H$ and $x \in H\cap N$. Thus $H\cap N \mathrel{\unlhd} H$. $\square$
\end{proof}

\subsubsection{Congugate subgroups}

\begin{definition}
Let $G$ be a group and $H \leq G$. If $g \in G$ then $g^{-1}Hg$ is called a \emph{conjugate} of $H$.
\end{definition}

\begin{theorem}
If $G$ is a group with $H \leq G$ and $g \in G$ then $g^{-1}Hg \leq G$.
\end{theorem}

\begin{proof}
Let $K = g^{-1}Hg$. It is easy to show that $1 \in K$, $g^{-1} \in K$ for any $g \in K$ and $ab \in K$ for any $a, b \in K$. Thus $K$ is a subgroup of $G$. $\square$
\end{proof}

\begin{theorem}
If $G$ is a group and $H \leq G$ then $H$ is in bijection with any conjugate subgroup $g^{-1}Hg$.
\end{theorem}

\begin{proof}
The map $h \mapsto g^{-1}hg$ is invertible and hence a bijection. $\square$
\end{proof}

\begin{definition}
Let $G$ be a group and $H \leq G$. The intersection, $\cap_{g \in G} g^{-1}Hg$, of the conjugates of $H$ in $G$ is called the \emph{normal core} of $H$ in $G$.
\end{definition}

\begin{theorem}
The normal core of a subgroup $H$ of a group $G$ is a normal subgroup of $G$.
\end{theorem}

\begin{proof}
Let $K = \cap_{g \in G} g^{-1}Hg$. Clearly $K$ is a subgroup of $G$ as it is an intersection of subgroups of $G$.

As $M \to g^{-1}Mg$ is a bijection of conjugates of $H$ we have that $k \in K$ implies $g^{-1}kg \in K$. Thus $K$ is normal in $G$. $\square$
\end{proof}

\subsubsection{The Normalizer of a Subset}

\begin{definition}
Let $G$ be a group and $S \subseteq G$, then the \emph{normalizer} of $S$ in $G$ is the set
$$N_G(S) = \{g \in G \;|\; g^{-1}Sg = S\}.$$
If $S = \{s\}$ we usually write $N_G(s) = N_G(S)$.
\end{definition}

Note that $N_G(S)$ need not be a normal subgroup of $G$.

\begin{theorem}
If $G$ is a group and $\emptyset \neq S \subseteq G$ then $N_G(S) \leq G$.
\end{theorem}

\begin{proof}
Clear. $\square$
\end{proof}

\begin{theorem}
If $G$ is a group and $S \subseteq G$ then $Z_G \in N_G(S)$.
\end{theorem}

\begin{proof}
Clear. $\square$
\end{proof}

\subsubsection{Basic Theorems Regarding the Normalizer of a Subset}

\begin{theorem}
Let $G$ be a group and $g \in G$. Then $N_G(g) = C_G(g)$.
\end{theorem}

\begin{proof}
Clear. $\square$
\end{proof}

\begin{theorem}
Let $G$ be a group and $H \leq G$, then
$$C_G(H) \leq N_G(H) \leq G.$$
\end{theorem}

\begin{proof}
Clear. $\square$
\end{proof}

\begin{theorem}
Let $G$ be a group, then $N_G(Z_G) = G$.
\end{theorem}

\begin{proof}
We have that $G = C_G(Z_G) \leq N_G(Z_G) = G$. $\square$
\end{proof}

\begin{theorem}
Let $G$ be a group and $H \leq G$. Then $H \leq N_G(H)$.
\end{theorem}

\begin{proof}
It is clear that $H \subseteq N_G(H)$. As both are groups, the result follows. $\square$
\end{proof}

\subsubsection{The Commutator Subgroup of a Group is Normal}

\begin{theorem}
Let $G$ be a group. Then $[G, G] \mathrel{\unlhd} G$.
\end{theorem}

\begin{proof}
Letting $g^s = s^{-1}gs$. Recall that any element of $[G, G]$ is a finite product of commutators.

The result follows from
\begin{align*}
([g_1, h_1][g_2, h_2]\ldots [g_n, h_n])^s &= [g_1, h_1]^s[g_2, h_2]^s\ldots [g_n, h_n]^s\\
                                          &= [g_1^s, h_1^s][g_2^s, h_2^s]\ldots [g_n^s, h_n^s].
\end{align*} $\square$
\end{proof}

\subsubsection{Unique Subgroups of a Given Order}

\begin{theorem}
Let $G$ be a group and suppose that $H$ is the only subgroup of $G$ of (finite) order $n$. Then $H$ is a normal subgroup of $G$.
\end{theorem}

\begin{proof}
As $H$ is in bijection with all its conjugates $g^{-1}Hg$, all its conjugates also have order $n$ and are thus equal to $H$. Thus $H$ is a normal subgroup of $G$. $\square$
\end{proof}

\subsection{Quotient Groups}

\subsubsection{Quotient Groups}

\begin{theorem}
Let $G$ be a group and $N \mathrel{\unlhd} G$. If $aH = a'N$ and $bN = b'N$ for $a, a', b, b' \in G$ then $abN = a'b'N$.
\end{theorem}

\begin{proof}
This follows immediately from Theorem \ref{equivnormal} (7) and Theorem \ref{cosets}. $\square$
\end{proof}

This allows us to define a well-defined operation on cosets of a normal subgroup.

\begin{theorem}
Let $G$ be a group an $N \mathrel{\unlhd} G$. The cosets of $N$ form a group under the operation
$$(aN).(bN) = abN.$$
\end{theorem}

\begin{proof}
The identity is the coset $N$. The inverse of $gN$ is $g^{-1}N$ and the associative law clearly holds. $\square$
\end{proof}

\begin{definition}
The group of cosets of a normal subgroup $N$ of a group $G$ is called the \emph{quotient group} of $G$ by $N$, which we denote $G/N$.
\end{definition}

Note that we have used left cosets in this definition. As $N$ is normal in $G$ we have $gN = Ng$ for all $g \in G$. Therefore we can equally well give the definition in terms of right cosets. The quotient group in that case is usually denoted $N\backslash G$.

\begin{theorem}
If $N \mathrel{\unlhd} G$ then
$$|G/N| = [G:N].$$
\end{theorem}

\begin{proof}
Clear. $\square$
\end{proof}

\begin{theorem}
Let $G$ be a group and $N \mathrel{\unlhd} G$. Then:
\begin{enumerate}
\item If $G$ is abelian then so is $G/N$
\item If $G$ is finite then so is $G/N$
\item If $G$ is cyclic then so is $G/N$
\item If $G$ is finitely generated then so is $G/N$
\end{enumerate}
\end{theorem}

\begin{proof}
(1) $(aN)(bN) = abN = baN = (bN)(aN)$.

(2) Clear.

(3) Let $G = \langle g \rangle$. The elements of $G/N$ are precisely the elements $g^iN = (gN)^i$ for $i \in \mathbb{N}$.

(4) If $G = \langle g_1, g_2, \ldots, g_n \rangle$ then $G/N = \langle g_1N, g_2N, \ldots, g_nN \rangle$. $\square$
\end{proof}

\begin{theorem}
If $G$ is a group and $G/Z_G$ is cyclic then $G$ is abelian.
\end{theorem}

\begin{proof}
As $G/Z_G$ is cyclic $G/Z_G = \langle aZ_G \rangle$ for some $a \in G$. Then $g = a^rk_1$ and $h = a^sk_2$ for some $r, s \in \Z$ and $k_1, k_2 \in Z_G$.

Thus
$$gh = a^rk_1a^sk_2 = a^ra^sk_1k_2 = a^sa^rk_1k_2 = a^sk_2a^rk_1 = hg.$$ $\square$
\end{proof}

\subsubsection{The Abelianisation of a Group}

\begin{theorem}
If $G$ is a group then $G/[G, G]$ is abelian. It is the largest quotient of $G$ which is abelian.
\end{theorem}

\begin{proof}
The first part follows from $ab[G, G] = ab[b, a][G, G] = ba[G, G]$.

Suppose that $N$ is a normal subgroup of $G$, but $[g, h] \notin N$ for some $g, h \in G$.

Clearly $[g, h] \neq 1$, as $1 \in N$. Thus $gh \neq hg$.

But we also have that $ghN \neq hgN$, as $g^{-1}h^{-1}gh \notin N$. Thus $(gN)(hN) \neq (hN)(gN)$ and $G/N$ is not abelian.

Thus if $G/N$ is abelian, $[G, G] \subseteq N$ which shows the result. $\square$
\end{proof}

\begin{definition}
The group $G/[G, G]$ is called the \emph{abelianisation} of $G$.
\end{definition}

\subsubsection{The Correspondence Theorem}

\begin{theorem}
If $G$ is a group and $N \mathrel{\unlhd} G$ then there is an order preserving bijection
$$\{\mbox{subgroups of}\;\;G\;\mbox{containing}\;N \} \leftrightarrow \{\mbox{subgroups of}\;\;G/N\}.$$ 
\end{theorem}

\begin{proof}
If $N \leq A \leq G$ then $N \mathrel{\unlhd} A$. Thus we can define the map $\phi : A \mapsto A/N$. The group Clearly $A/N \subseteq G/N$. It is easy to show that $A/N \leq G/N$.

Conversely, suppose $H' \leq G/N$. Let $H = \{h \in G \;|\; hN \in H'\}$. Clearly $N \subseteq H$. It is also easy to show that $H \leq G$. We see that $H' = \phi(H)$ and so $\phi$ is a surjection.

If $A/N = B/N$ for $A, B \leq G$ containing $N$, then $A$ and $B$ consist of precisely the same cosets of $N$ in $G$. Thus $A = B$ and $\phi$ is injective.

It is easy to see that $\phi$ is order preserving. $\square$
\end{proof}

\begin{definition}
If $G$ is a group and $A, B \leq G$ we write $\langle A, B \rangle = \langle A\cup B \rangle$.
\end{definition}

\begin{theorem} \label{normalcontaining}
If $G$ is a group, $A, B \leq G$ and $N \mathrel{\unlhd} G$ then
\begin{enumerate}
\item If $A \subseteq B$ for a finite group $A$ then $|B:A| = |B/N:A/N|$
\item $\langle A, B \rangle/N = \langle A/N, B/N \rangle$
\item $(A\cap B)/N = (A/N)\cap (B/N)$
\item $A \mathrel{\unlhd} G$ iff $A/N \mathrel{\unlhd} G/N$
\end{enumerate}
\end{theorem}

\begin{proof}
(1) Follows from $|B:A| = |B|/|A|$ and $|B/N| = |B|/|N|$.

(2) $\langle A, B \rangle/N$ consists of the cosets $gN$ for $g \in \langle A\cup B \rangle$. But as $(aN)(bN) = abN$ and $(aN)^{-1} = a^{-1}N$ a set of generators of this group is given by $(A/N)\cup (B/N)$, which gives the result.

(3) Note that $N \mathrel{\unlhd} A\cap B$ and so the result makes sense. It is clear that both groups consist of the cosets $gN$ for $g \in A\cap B$.

(4) We have that $A \mathrel{\unlhd} G$ iff $g^{-1}ag \in A$ for all $a \in A$ and $g \in G$. But this holds iff $g^{-1}agN \in A/N$ which is true iff $(g^{-1}N)(aN)(gN) \in A/N$ for all $gN \in G/N$, which is precisely the statement of the result. $\square$
\end{proof}

\subsection{Homomorphisms}

\subsubsection{Homomorphisms}

\begin{definition}
If $G$ and $H$ are groups a \emph{homomorphism} from $G$ to $H$ is a map $\phi : G \to H$ such that $\phi(g_1g_2) = \phi(g_1)\phi(g_2)$ for all $g_1, g_2 \in G$.
\end{definition}

Of course this is the multiplicative version of the definition. There is a similar definition with addition instead of multiplication in the additive case.

\begin{theorem}
If $\phi : G \to H$ is a homomorphism then $\phi(g^{-1}) = \phi(g)^{-1}$ and $\phi(1) = 1$.
\end{theorem}

\begin{proof}
From $\phi(1) = \phi(1.1) = \phi(1)\phi(1)$ we see that $\phi(1) = 1$.

From $1 = \phi(1) = \phi(gg^{-1}) = \phi(g)\phi(g^{-1})$ we see that $\phi(g^{-1}) = \phi(g)^{-1}$. $\square$
\end{proof}

\begin{theorem}
If $H \leq G$ then the map $\iota : H \to G$ which sends $h \mapsto h$ for all $h \in H$ is a homomorphism, called the canonical injection.
\end{theorem}

\begin{proof}
Clear. $\square$
\end{proof}

\begin{theorem}
If $G$ is a group and $N \mathrel{\unlhd} G$ then the map $\rho : G \to G/N$ which maps $g \mapsto gN$ for all $g \in G$ is a homomorphism, called the canonical projection.
\end{theorem}

\begin{proof}
We have that $\rho(gh) = ghN = (gN)(hN) = \rho(g)\rho(h)$. $\square$
\end{proof}

\subsubsection{The Kernel and Image of a Homomorphism}

\begin{definition}
Let $\phi : G \to H$ be a homomorphism. The \emph{kernel} of $\phi$ is the set
$$\mbox{ker}(\phi) = \{g \in G \;|\; \phi(g) = 1\}.$$
\end{definition}

\begin{theorem}
If $\phi : G \to H$ is a homomorphism then ker$(\phi) \mathrel{\unlhd} G$. Moreover, if $N \mathrel{\unlhd} G$ is any normal subgroup, it is the kernel of some homomorphism from $G$.
\end{theorem}

\begin{proof}
It is easy to see that ker$(\phi)$ is a subgroup of $G$.

To see that it is normal, let $g \in G$ and $k \in \mbox{ker}(\phi)$, then $\phi(g^{-1}kg) = 1$. Thus $g^{-1}kg \in \mbox{ker}(\phi)$.

To show the second part of the theorem it is enough to note that the kernel of the canonical projection $\rho : G \to G/N$ is $N$. $\square$
\end{proof}

\begin{definition}
If $\phi : G \to H$ is a homomorphism, the \emph{image} of $\phi$ is the set
$$\mbox{im}(\phi) = \{h \in H \;|\; \mbox{there exists}\;\; g \in G \;\;\mbox{such that}\;\; \phi(g) = h\}.$$
\end{definition}

\begin{theorem}
If $\phi : G \to H$ is a homomorphism then im$(\phi) \leq H$.
\end{theorem}

\begin{proof}
Clear.
\end{proof}

\subsubsection{Endomorphisms of a Group}

\begin{definition}
Let $G$ be a group. An \emph{endomorphism} of $G$ is a homomorphism $\phi : G \to G$. The set of all endomorphisms of $G$ is denoted End$(G)$.
\end{definition}

\begin{theorem}
If $G$ is an abelian group under addition, then End$(G)$ has the structure of an additive abelian group.
\end{theorem}

\begin{proof}
If $\phi_1$ and $\phi_2$ are elements of End$(G)$ we define $(\phi_1 + \phi_2)(g) = \phi_1(g) + \phi_2(g)$ for all $g \in G$. If it easy to check that this gives the structure of an abelian group on End$(G)$, with the identity map serving as the identity of End$(G)$. $\square$
\end{proof}

In fact, the group End$(G)$ can even be given the structure of a ring, where the multiplication in the ring is given by composition of homomorphisms. However, we will not prove this here.

\subsubsection{Isomorphisms}

\begin{definition}
A homomorphism of groups $\phi : G \to H$ is said to be an \emph{isomorphism} if there exists a homomorphism $\phi^{-1} : H \to G$ such that $\phi^{-1}\circ \phi =$ id$_G$. We say that $G$ and $H$ are isomorphic, which we denote by $G \cong H$.
\end{definition}

\begin{theorem}
A homomorphism $\phi : G \to H$ is an isomorphism iff it is a bijection.
\end{theorem}

\begin{proof}
Let $\phi^{-1} : H \to G$ be the inverse map. Clearly it is a homomorphism, which also composes with $\phi$ to give the identity. $\square$
\end{proof}

\subsubsection{Automorphisms}

\begin{definition}
If $G$ is a group, an \emph{automorphism} of $G$ is an isomorphism $\phi : G \to G$. We denote the set of automorphisms of $G$ by Aut$(G)$.
\end{definition}

\begin{theorem}
Let $G$ be a group and $a \in G$. The map $\sigma_a : g \mapsto a^{-1}ga$, called \emph{conjugation} by $a$ is an automorphism.
\end{theorem}

\begin{proof}
The map is clearly a bijection. We have that
$$\sigma_a(g_1g_2) = a^{-1}g_1g_2a = (a^{-1}g_1a)(a^{-1}g_2a) = \sigma_a(g_1)\sigma_a(g_2).$$
Thus $\sigma_a$ is a homomorphism. $\square$
\end{proof}

\begin{definition}
Let $G$ be a group. An automorphism of $G$ that is conjugation $\sigma_a$ for some $a \in G$ is said to be an \emph{inner autormorphism} of $G$. We write Inn$(G)$ for the set of inner autormorphisms of $G$.
\end{definition}

\begin{theorem}
For a group $G$, the set Aut$(G)$ is a group under composition.
\end{theorem}

\begin{proof}
Clear. $\square$
\end{proof}

\begin{theorem}
For a group $G$, Inn$(G) \mathrel{\unlhd}$ Aut$(G)$.
\end{theorem}

\begin{proof}
Conjugation by $1 \in G$ is an inner automorphism. It is the identity of Aut$(G)$.

It is clear that $\sigma_{g^{-1}}$ is the inverse of $\sigma_g$ in Aut$(G)$ and is an inner automorphism. It is easy to see that $\sigma_{ab}$ is the composition of $\sigma_a$ and $\sigma_b$ and so Inn$(G)$ is a subgroup of Aut$(G)$.

To show that it is normal, let $\sigma_g \in$ Inn$(G)$ and $\phi \in$ Aut$(G)$. Then for $h \in G$:
\begin{align*}
\phi^{-1}\cdot\sigma_g\cdot\phi(h) &= \phi\circ \sigma_g \circ \phi^{-1}(h)\\
                                   &= \phi(g^{-1}\phi^{-1}(h)g)\\
                                   &= \phi(g^{-1})h\phi(g)\\
                                   &= \sigma_{\phi(g)}(h) \in \mbox{Inn}(G)
\end{align*} $\square$
\end{proof}

\begin{theorem}
Let $G$ be a group and let $\phi : G \to$ Aut$(G)$ be the map $g \mapsto \sigma_g$. Then $\phi$ is a homomorphism. We refer to this as the natural homomorphism from $G$ to Aut$(G)$.
\end{theorem}

\begin{proof}
Clear. $\square$
\end{proof}

\begin{theorem}
Let $\phi : G \to$ Aut$(G)$ be the natural homomorphism. Then ker$(\phi) = Z_G$.
\end{theorem}

\begin{proof}
$\phi(g) =$ id$_G$ iff $g^{-1}ag = a$ for all $a \in G$. $\square$
\end{proof}

\subsubsection{The Automorphism Group of a Cyclic Group}

\begin{theorem}
The automorphism group of $\mathbb{Z}$ is isomorphic to $C_2$.
\end{theorem}

\begin{proof}
An automorphism of $\mathbb{Z}$ must send $1$ to a generator of $\mathbb{Z}$. As the only generators are $\pm 1$, the result follows. $\square$
\end{proof}

\begin{theorem}
Let $C_n = \langle g \rangle$ be a cyclic group of order $n$. The automorphisms of $C_n$ consist of the maps $a \mapsto a^k$ for $1 \leq k < n$ with gcd$(k, n) = 1$.
\end{theorem}

\begin{proof}
Let $\sigma \in$ Aut$(C_n)$. Then $\sigma(g) = g^k$ for some $k$. In order to be an automorphism $g^k$ must generate $C_n$. Thus gcd$(k, n) = 1$.

Consider any element $a = g^r \in C_n$. Then $\sigma(a) = \sigma(g)^r = (g^k)^r = a^k$. $\square$
\end{proof}

\begin{corollary}
We have $|\mbox{Aut}(C_n)| = \varphi(n)$.
\end{corollary}

\begin{proof}
Clear. $\square$
\end{proof}

Note that Aut$(C_n)$ is an abelian group since $(a^r)^s = (a^s)^r$.

\subsection{Characteristic Subgroups}

\subsubsection{Characteristic Subgroups}

\begin{definition}
Let $G$ be a group. A subgroup $H \leq G$ is said to be a \emph{characteristic subgroup} if $\phi(H) = H$ for all $\phi \in$ Aut$(G)$.
\end{definition}

\begin{theorem}
A characteristic subgroup $H$ of a group $G$ is normal in $G$.
\end{theorem}

\begin{proof}
$H$ is preserved under all inner automorphisms, i.e. under conjugation by elements of $G$. $\square$
\end{proof}

\begin{theorem}
If $H$ is a characteristic subgroup of a group $G$ then every $\phi \in$ Aut$(G)$ induces an automorphism of $G/H$ which gives a map Aut$(G) \to$ Aut$(G/H)$.
\end{theorem}

\begin{proof}
As $\phi$ preserves $H$, the induced map $gH \mapsto \phi(g)H$ is well defined and an injection. It is clearly a surjection because $\phi$ is invertible. It is clearly a homomorphism. $\square$
\end{proof}

\begin{theorem}
Let $H$ be a subgroup of $G$ of index $n$. If it is the only subgroup of $G$ of index $n$ then it is a characteristic subgroup.
\end{theorem}

\begin{proof}
Any automorphism $\phi$ of $G$ must send $H$ to a subgroup $H'$ of $G$. The induced map $H \to H'$ must be an isomorphism since $\phi$ is injective.

Let $g_1H, g_2H, \ldots, g_nH$ be the set of left cosets of $H$ in $G$, for some fixed set of representatives $g_i$.

The map $\phi$ sends each coset $g_iH$ to a coset $\phi(g_i)H'$ of $H'$ in $G$. As every element of $G$ has a unique representation as $g_ih$ for some $i$ and $h \in H$ and $\phi$ is an automorphism, the cosets $\phi(g_i)H'$ must partition $G$.

Thus the index of $H'$ in $G$ is $[G:H]$. But $H$ is the only subgroup with this index, thus $H = H'$. $\square$
\end{proof}

Let us write $H$ char $G$ to denote that $H$ is a characteristic subgroup of $G$.

\begin{theorem}
If $K$ char $H$ and $H$ char $G$ then $K$ char $G$.
\end{theorem}

\begin{proof}
Clearly any automorphism of $G$ induces an automorphism of $H$ which in turn induces an automorphism of $K$. $\square$
\end{proof}

\subsubsection{A Characteristic Subgroup of a Normal Subgroup is Normal}

\begin{theorem}
If $K$ char $H \mathrel{\unlhd} G$ then $K \mathrel{\unlhd} G$.
\end{theorem}

\begin{proof}
Any inner automorphism of $G$ preserves $H$ because it is normal. Thus it induces an automorphism of $H$, which preserves $K$ because it is a characteristic subgroup of $H$. $\square$
\end{proof}

\subsubsection{The Centre of a Group is a Characteristic Subgroup}

\begin{theorem}
Let $G$ be a group. Then $Z_G$ char $G$.
\end{theorem}

\begin{proof}
An automorphism $\phi$ of $G$ maps $Z_G$ isomorphically to a subgroup $Z$ of $G$. But if $z \in Z_G$ and $g \in G$ we have $\phi(z)\phi(g) = \phi(g)\phi(z)$.

Because $\phi$ is an automorphism of $G$, this shows that every element of $Z$ commutes with every element of $G$. Thus $Z \leq Z_G$. By inverting $\phi$ we see that $Z_G \leq Z$. $\square$
\end{proof}

\subsubsection{Subgroups of Cyclic Groups are Characteristic}

\begin{theorem}
A subgroup $H$ of a cyclic group $G$ is characteristic.
\end{theorem}

\begin{proof}
Let $G = \langle g \rangle$. Any automorphism $\phi$ of $G$ sends $g$ to $g^n$ for some integer $n$ such that $\langle g \rangle = \langle g^n \rangle$.

If $H \leq G$ then $H = \langle g^k \rangle$ for some integer $k$.

Now $phi(H) = \langle \phi(g^k) \rangle = \langle g^{nk} \rangle$.

Clearly $\langle g^{nk} \rangle \subseteq \langle g^k \rangle$. Therefore it remains to show that $\langle g^k \rangle \subseteq \langle g^{nk}$.

As $\langle g^n \rangle = G$ there exists an integer $s$ such that $g = g^{ns}$.

Then $g^k = (g^{ns})^k = (g^{nk})^s \in \langle g^{nk} \rangle$.

Thus $\phi(H) = H$. $\square$
\end{proof}

\subsubsection{The Commutator Subgroup is Characteristic}

\begin{theorem}
If $G$ is a group then $[G, G]$ char $G$.
\end{theorem}

\begin{proof}
Let $S$ be the set of commutators of $G$. If $[x, y] \in S$ and $\phi \in$ Aut$(G)$ then $\phi([x, y] = [\phi(x), \phi(y)]$.

This is clearly a bijective map on $S$. As $[G, G]$ is the smallest subgroup of $G$ containing $S$, we have that $\phi([G, G]) = [G, G]$. $\square$
\end{proof}

\subsection{The Group Isomorphism Theorems}

\subsubsection{The First Group Isomorphism Theorem}

\begin{theorem}
Let $\phi : G \to H$ be a group homomorphism. Then $G/\mbox{ker}(\phi) \cong \mbox{im}(\phi)$.
\end{theorem}

\begin{proof}
Let $K = \mbox{ker}(\phi)$. Define $\psi : G/K \to \phi(G)$ by $\psi(aK) = \phi(a)$.

The map $\psi$ is well defined, for if $aK = bK$ then $a = bk$ for some $k \in K$ and $\phi(a) = \phi(b)\phi(k) = \phi(b)$.

It is easy to see that $\psi$ is a homomorphism on $G/K$.

We can check that $\psi$ is injective, since if $\phi(a) = \phi(b)$ then $ab^{-1} \in K$ and so $aK = bK$.

Clearly $\psi$ is surjective and thus an isomorphism. $\square$
\end{proof}

\subsubsection{Quotients of quotient groups}

\begin{theorem}
Let $H$ and $K$ be normal subgroups of a group $G$ and let $K \leq H$. Then $(G/K)/(H/K) \cong G/H$.
\end{theorem}

\begin{proof}
As $K \mathrel{\unlhd} G$ we have that $K \mathrel{\unlhd} H$. As $H \mathrel{\unlhd} G$ and both groups contain $K$, we have that $H/K \mathrel{\unlhd} G/K$. Thus the theorem makes sense.

We define the map $\phi : G/K \to G/H$ by $\phi(gK) = gH$. It is easy to check that $\phi$ is well-defined and a homomorphism.

We have that ker$(\phi) = \{gK \in G/K \;|\; g \in H\} = H/K$. Then by the first isomorphism theorem $(G/K)/(H/K) \cong G/H$. $\square$
\end{proof}

\subsubsection{The Second Isomorphism Theorem}

\begin{theorem}
Let $H, N$ be subgroups of a group $G$ and let $N \mathrel{\unlhd} G$. Then
$$H/(H\cap N) \cong HN/N.$$
\end{theorem}

\begin{proof}
We have previously shown that $H\cap N \mathrel{\unlhd} H$ and it is easy to see that $N \mathrel{\unlhd} HN$, so the theorem makes sens.

Define a map $\phi : H \to HN/N$ by $\phi(h) = hN$. It is easy to check that this map is a homomorphism.

We have that ker$(\phi) = H\cap N$ and $\phi$ is surjective. Thus by the first isomorphism theorem the result follows. $\square$
\end{proof}

\subsubsection{The Third Isomorphism Theorem}

\begin{theorem}
Let $G$ be a group and $N$ a normal subgroup. If $K$ is a normal subgroup of $G$ containing $N$ then
$$(G/N)/(K/N) \cong G/K.$$
\end{theorem}

\begin{proof}
We have already shown in Theorem \ref{normalcontaining} that $K/N$ is a normal subgroup of $G/N$. Therefore the quotient $(G/N)(K/N)$ makes sense.

Define a map $\Phi : G/N \to G/K$ by $\Phi : gN \mapsto gK$ for all $g \in G$. It is easy to see that this is well-defined as $N \subseteq K$. It is also easy to see that it is a surjective homomorphism.

The kernel is precisely the set of cosets $gN$ with $g \in K$, i.e. $K/N$. 

Therefore, the result follows by the first isomorphism theorem. $\square$
\end{proof}

\subsection{Direct Products of Groups}

\subsubsection{The Direct Product}

\begin{theorem}
Let $G$ and $H$ be groups. Define a binary operation on $G\times H$ by
$$(g_1, h_1)\cdot (g_2, h_2) = (g_1g_2, h_1h_2),$$
for $g_1, g_2 in G$ and $h_1, h_2 \in H$.
Then $G\times H$ with this operation is a group.
\end{theorem}

\begin{proof}
The operation is clearly closed.

Associativity follows from associativity of the groups $G$ and $H$. The element $(1_G, 1_H)$ is an identity, and $(g^{-1}, h^{-1})$ is an inverse of $(g, h)$. $\square$
\end{proof}

\begin{definition}
The group $G\times H$ under the operation given in the theorem is called the \emph{(external) direct product} of $G$ and $H$ and is also denoted $G\times H$.
\end{definition}

\subsubsection{The Order of a Direct Product}

\begin{theorem}
Let $G$ and $H$ be groups and let $g \in G$ have order $m$ and $h \in H$ have order $n$. Then $(g, h) \in G\times H$ has order $lcm(m, n)$.
\end{theorem}

\begin{proof}
The order of $(g, h)$ will be the smallest positive integer which is a multiple of $m$ and $n$. $\square$
\end{proof}

\begin{corollary}
If $G$ and $H$ be finite groups. Then $|G\times H| = |G|\times |H|$.
\end{corollary}

\begin{proof}
Follows from the order of a cartesian product of finite sets. $\square$
\end{proof}

\subsubsection{The Direct Product of a Family of Groups}

\begin{definition}
Let $\{G_i : i \in I\}$ be a family of groups indexed by a set $I$. The (external) \emph{direct product} of the family is
$$\prod_{i \in I} G_i = \{f : I \to \cup_{i \in I} G_i \;|\; f(i) \in G_i \forall i \in I\}.$$
\end{definition}

If $I$ is a two element set, the definition isn't exactly the same as the definition we already gave, but note that $(g, h) = (f(1), f(2))$ for some $f$ from the second definition.

Also note that if $I$ is a countably infinite set then we can think of the direct product as sequences of elements, one from each group $G_i$, i.e. the elements are of the form $(g_i)_{i \in \mathbb{N}}$ where $g_i \in G_i$.

\begin{theorem}
Let $\{G_i : i \in I\}$ be a family of groups. The direct product $\prod_{i \in I} G_i$ is a group under the operation $(f_1\cdot f_2)(i) = f_1(i)f_2(i)$.
\end{theorem}

\begin{proof}
The definition makes sense because $f_1(i)f_2(i) \in G_i$ since $G_i$ is a group. It is clear that the direct product is closed under its operation.

It is easy to check associativity and that $e(i) = 1_{G_i}$ is an identity for the group. The inverse of $f$ is given by $f^{-1}(i) = f(i)^{-1}$ for all $i \in I$. $\square$
\end{proof}

\subsubsection{The Canonical Projections of a Direct Product}

\begin{definition}
Let $\{G_i : i \in I\}$ be a family of groups. The \emph{canonical projections} of $\prod_{i \in I} G_i$ are the maps $\pi_j : \prod_{i \in I} G_i \to G_j$ given by $\pi_j : f \mapsto f(i)$ for all $f \in \prod_{i \in I} G_i$.
\end{definition}

\begin{theorem}
The canonical projections $\pi_j : \prod_{i \in I} G_i \to G_j$ are surjective homomorphisms.
\end{theorem}

\begin{proof}
It is easy to see that the $\pi_j$ are homomorphisms.

If $g \in G_j$ and $f(i) = e_{G_i}$ for all $i \neq j$ and $f(j) = g$ then we see that $\pi_j(f) = g$, which shows that $\pi_j$ is surjective. $\square$
\end{proof}

\subsubsection{The Restricted Direct Product}

\begin{definition}
Let $\{G_i : i \in I\}$ be a family of groups. The \emph{restricted direct product} of the $G_i$ is
$$\prod'_{i \in I} G_i = \{f \in \prod_{i \in I} G_i \;|\; f(i) = e_{G_i} \;\;\mbox{for all but finitely many} i\}.$$
\end{definition}

\begin{theorem}
Let $\{G_i : i \in I\}$ be a family of groups. The restricted direct product $H = \prod'_{i \in I} G_i$ is a normal subgroup of the direct product $G = \prod_{i \in I} G_i$.
\end{theorem}

\begin{proof}
It is easy to check that the restricted direct product is a subgroup of the direct product.

Note that if $g \in G$ and $h \in H$ then $(g^{-1}hg)(i) = g^{-1}(i)h(i)g(i)$ for all $i \in I$.

But $h(i) = e_{G_i}$ for all but finitely many $i \in I$. Thus $(g^{-1}hg)(i) = g^{-1}(i)g(i) = e_{G_i}$ for all but finitely many $i$. This means that $g^{-1}hg \in H$ and $H \mathrel{\unlhd} G$. $\square$
\end{proof}

\subsubsection{The Canonical Injections of the Restricted Direct Product}

\begin{definition}
Let $\{G_i : i \in I\}$ be a family of groups. The \emph{canonical injections} of the restricted product are the maps $\iota_j : G_j \to \prod'_{i \in I} G_i$ given by
$$(\iota_j(g))(i) = \begin{cases}g, & i = j\\e_{G_i}, & i \neq j\end{cases}.$$
\end{definition}

\begin{theorem}
Let $\{G_i : i \in I\}$ be a family of groups. The canonical injections $\iota_j : G_j \to \prod'_{i\in I} G_i$ are injective homomorphisms.
\end{theorem}

\begin{proof}
Clear. $\square$
\end{proof}

\begin{theorem}
Let $\{G_i : i \in I\}$ be a family of groups. The images $\iota_j(G_j)$ of the canonical injections of the restricted product are normal subgroups of the product $\prod_{i\in G} G_i$.
\end{theorem}

\begin{proof}
As $\iota_j$ is a homomorphism, its image is a subgroup of the product.

Let $G = \prod_{i \in I} G_i$ and $H = \iota_j(G_j)$ be the image of $\iota_j$. If $h \in H$ then there exists an $a \in G_j$ such that $\iota_j(a) = h$. But this means that $h(j) = a$ and $h(i) = e_{G_i}$ for all $i \neq j$.

Let $g \in G$. Then for $i = j$ we have $(g^{-1}hg)(j) = g^{-1}(j)ag(j) \in G_j$.

If $i \neq j$ then $(g^{-1}hg)(i) = g^{-1}(i)g(i) = e_{G_i}$.

Thus we see that $g^{-1}hg = \iota_j(g^{-1}(j)ag(j)) \in \iota_j(G_j) = H$. Thus $H \mathrel{\unlhd} G$. $\square$
\end{proof}

\subsubsection{Internal Direct Products}

\begin{definition}
Let $G$ be a group and $H, K \leq G$. The group $G$ is said to be the \emph{internal direct product} of $H$ and $K$ if
\begin{enumerate}
\item $G = \{hk \;|\; h \in H, k \in K\}$
\item $G\cap K = \{1_G\}$
\item $hk = kh$ for all $h \in H, k \in K$
\end{enumerate}
\end{definition}

\begin{theorem}
Let $G$ be a group and $H, K \leq G$. Then $G$ is the internal direct product of $H$ and $K$ iff the following hold:
\begin{enumerate}
\item $G = \{hk \;|\; h \in H, k \in K\}$
\item $G\cap K = \{1_G\}$
\item $H, K \mathrel{\unlhd} G$
\end{enumerate}
\end{theorem}

\begin{proof}
Suppose $G$ is the direct product of $H$ and $K$. Let $g \in G$ and $k \in K$. Then by the first condition above, $g = h'k'$ for some $h' \in H$, $k' \in K$.

We have that $g^{-1}kg = k'^{-1}h'^{-1}kh'k'$ which by the third condition above is equal to $k'^{-1}kk' \in K$. Thus $K$ is a normal subgroup of $G$. A similar argument holds for $H$.

Conversely, suppose that the given conditions hold. Let $h \in H$ and $k \in K$. As $K$ is normal in $G$ we have $h^{-1}kh = k'$ for some $k' \in K$. Similarly $k'^{-1}hk' = h'$ for some $h' \in H$.

Rewriting these, we have $h^{-1}k = k'h^{-1}$ and $k'^{-1}h = h'k'^{-1}$. Multiplying these relations we have that $k'^{-1}k = h'h^{-1}$. But the left side is an element of $K$ and the right side an element of $H$. Thus both equal $1$ by the second condition above.

In other words, $k' = k$ and $h' = h$, which implies that $hk = kh$. $\square$
\end{proof}

\begin{definition}
Let $\{N_i : i \in I\}$ be a family of subgroups of a group $G$. The group $G$ is said to be the internal direct product of the family if:
\begin{enumerate}
\item $N_i \mathrel{\unlhd} G$ for all $i$
\item $G = \langle \cup_{i\in I} N_i \rangle$
\item $N_i \cap \langle \cup_{j \neq i} N_i \rangle = 1_G$
\end{enumerate}
\end{definition}

\begin{theorem}
Let $\{N_i : i \in I\}$ be a family of subgroups of a group $G$. Then $G$ is the internal direct product of the family if:
\begin{enumerate}
\item $N_i \mathrel{\unlhd} G$ for all $i \in I$
\item $g_ig_j = g_jg_i$ for all $g_i \in N_i$ and $g_j \in N_j$ for $i \neq j$
\item Every $g \in G$ has a unique expression as a product of elements from distinct $N_i$'s
\end{enumerate}
\end{theorem}

\begin{proof}
The proof of the second condition is essentially the same as for the theorem above.

Assume $G$ is the direct product of the $N_i$'s. Let $g \in G$. As the $N_i$'s generate $G$, we can write $g$ as a product $g_{\lambda_1}g_{\lambda_2}\ldots g_{\lambda_k}$ where $1 \neq g_{\lambda_i} \in G_{\lambda_i}$, and the $\lambda_i$ are distinct. 

Let $g = h_{\mu_1}h_{\mu_2}\ldots h_{\mu_m}$ be another such expression and suppose that $\mu_1 \neq \lambda_i$ for all $i$. Then $h_{\mu_1} \in N_{\mu_1} \cap \langle \cup_{i \neq \mu_1} N_i \rangle$, which is trivial.

This shows that the third condition holds.

Conversely, if the conditions in the theorem hold then clearly the third condition of the definition holds, due to uniqueness of expression as a product. The other conditions follow just as for the proof of the previous theorem. $\square$
\end{proof}

\subsubsection{Equivalence of Internal and External Direct Products}

\begin{theorem}
Let $G$ be a group and $H, K \leq G$. If $G$ is the internal direct product of $H$ and $K$ then $G \cong H\times K$.
\end{theorem}

\begin{proof}
Define a map $f : H\times K \to G$ by $(h, k) = hk$.

By the commutativity in the internal direct product, the map $f$ is a homomorphism.

To show it is injective, let $(h_1, k_1), (h_2, k_2) \in H\times K$. Suppose that $f(h_1, k_1) = f(h_2, k_2)$, i.e. $h_1k_1 = h_2k_2$. We have that $h_2^{-1}h_1 = k_2k_1^{-1}$. As the intersection of $H$ and $K$ is trivial, we have that $h_1 = h_2$ and $k_1 = k_2$.

Surjectivity of $f$ follows from the fact that $G = \{hk \;|\; h \in H, k \in K\}$. $\square$
\end{proof}

A similar proof shows that internal and external direct products of families of subgroups are isomorphic.

\subsubsection{The Automorphism Group of a Direct Product}

\begin{theorem}
If $A$ and $B$ are finite groups with coprime orders then Aut$(A\times B) \cong$ Aut$(A)\times$Aut$(B)$.
\end{theorem}

\begin{proof}
Let $\phi : A\times B \to A\times B$ be an automorphism. Composing with the projections onto $A$ and $B$ we get homomorphisms $\phi_A : A\times B \to A$ and $\phi_B : A\times B \to B$.

By the first isomorphism theorem, the order of ker$(\phi_A)$ must be $|B|$. As the order of any nontrivial element $a \in A$ is coprime to $|B|$ we see that no element of $A\times \{1\}$ can be in the kernel of $\phi_A$.

This means that $\phi_A$ induces an automorphism on $A$, and similarly $\phi_B$ induces an automorphism on $B$.

It is now clear that there is a homomorphism from Aut$(A\times B) \to$ Aut$(A)\times$ Aut$(B)$. But any pair of automorphisms $\rho : A \to A$, $\sigma : B \to B$ induces an automorphism of $A\times B$, and there is a homomorphism Aut$(A)\times$ Aut$(B) \to$ Aut$(A\times B)$, which is thus an isomorphism. $\square$
\end{proof}

\begin{theorem}
The automorphism group of $C_p\times C_p$ for a prime $p$ is isomorphic to GL$_2(\mathbb{Z}/p\mathbb{Z})$.
\end{theorem}

\begin{proof}
An endomorhism of $C_p\times C_p$ is entirely characterised by where it sends the generators $x$ and $y$ of the summands. Suppose it maps $(x, 1) \to (x^a, y^b)$ and $(1, y) \to (x^c, y^d)$, where $a, b, c, d$ can be taken modulo $p$.

This is an automorphism iff it is invertible, which happens iff $ad - bc = \pm 1$. Thus an automorphism corresponds to a matrix in GL$_2(\mathbb{Z}/p\mathbb{Z})$ and vice versa.

An easy computation shows that composition of two automorphisms corresponds to matrix multiplication of matrices in GL$_2(\mathbb{Z}/p\mathbb{Z})$. $\square$
\end{proof}

\subsection{The Semidirect Product}

\subsubsection{The Semidirect Product}

\begin{definition}
Let $H$ and $Q$ be groups and let $\theta : Q \to$ Aut$(H)$ be a group homomorphism. The semidirect product $H\rtimes_{\theta} Q$ is the set $\{(h, q)\;|\; h \in H, q \in Q\}$ with operation $(h_1, q_1)(h_2, q_2) = (h_1\theta(q_1)h_2, q_1q_2)$.
\end{definition}

\begin{theorem}
The semidirect product $H\rtimes_{\theta} Q$ is a group.
\end{theorem}

\begin{proof}
It is clear that $(1, 1)$ is the identity.

We have
$$(h_1, q_1)[(h_2, q_2)(h_3, q_3)] = (h_1\theta(q_1)[h_2\theta(q_2)h_3], q_1q_2q_3)] = (h_1\theta(q_1)h_2\theta(q_1q_2)h_3).$$
Similarly
$$[(h_1, q_1)(h_2, q_2)](h_3, q_3) = (h_1\theta(q_1)h_2\theta(q_1q_2)h_3, q_1q_2q_3).$$
Thus the associative law holds.

Finally, $(\theta(q^{-1})h^-1, q^{-1})$ is the inverse of $(h, q)$.
$\square$
\end{proof}

\begin{theorem}
Given groups $H$ and $Q$ and a group homomorphism $\theta : Q \to$ Aut$(H)$ there exist canonical injections $H \to H\rtimes_{\theta} Q$ and $Q \to H\rtimes_{\theta} Q$ given by $h \mapsto (h, 1)$ and $q \mapsto (1, q)$. These are group homomorphisms.
\end{theorem}

\begin{proof}
It is clear that the first is a homomorphism.

For the second, it follows from $\theta(q)1 = 1$. $\square$
\end{proof}

We will identify $H$ with its image under the first injection, so that it is a subgroup of $H\rtimes_{\theta} Q$ and similarly we will identify $Q$ with its image under the second injection, so that it is also a subgroup of $H\rtimes_{\theta} Q$.

\begin{theorem}
Let $H$ and $Q$ be groups and $\theta : Q \to$ Aut$(H)$ be a homomorphism. Then
\begin{enumerate}
\item $H \mathrel{\unlhd} H\rtimes_{\theta} Q$
\item $HQ = H\rtimes_{\theta} Q$
\item $H\cap Q = (1, 1) \in H\rtimes_{\theta} Q$
\end{enumerate}
\end{theorem}

\begin{proof}
Note that $(h, q) = (h, 1)(1, q)$. Therefore $HQ = H\rtimes_{\theta} Q$.

Let $\rho : H\rtimes_{\theta} Q \to Q$ be defined by $\rho(h, q) = q$. This is clearly a homomorphism with kernel $H$. Thus $H$ is a normal subgroup of $H\rtimes_{\theta} Q$.

Clearly $(1, 1)$ is the only element in (the images of) $H$ and $Q$ (in $H\rtimes_{\theta} Q$). $\square$
\end{proof}

\subsubsection{Internal Semidirect Products}

\begin{definition}
If $G$ is a group with subgroups $H$ and $Q$, we say that $G$ is the \emph{internal direct product} of $H$ and $Q$ if
\begin{enumerate}
\item $H \mathrel{\unlhd} G$
\item $HQ = G$
\item $H\cap Q = 1 \in G$
\end{enumerate}
\end{definition}

\begin{lemma}
Let $G$ be a group with subgroups $H$ and $Q$. Suppose $G = HQ$ and $H\cap Q = \{1\}$. Then every element $g \in G$ can be written uniquely in the form $hq$ with $h \in H$ and $q \in Q$.
\end{lemma}

\begin{proof}
Suppose $hq = h'q'$. Then $h'^{-1}h = q'q^{-1} \in H\cap Q$. Thus $h = h'$ and $q = q'$. $\square$
\end{proof}

It is clear that any external semidirect product is an internal semidirect product. But we also have the converse.

\begin{theorem}
Let $G$ be a group which is the internal semidirect product of subgroups $H$ and $Q$. Then $G \cong H\rtimes_{\theta} Q$ where $\theta : Q \to$ Aut$(H)$ is given by
$$\theta(q)(h) = qhq^{-1}.$$
\end{theorem}

\begin{proof}
By the lemma, every $g \in G$ can be written uniquely as $hq$ for $h \in H$ and $q \in Q$. Thus the map $\phi : (h, q) \mapsto hq$ is a bijection from $H\rtimes_{\theta} Q$ to $G$.

It suffices to show $\phi$ is a homomorphism:
\begin{multline*}
\phi((h_1, q_1)(h_2, q_2)) = \phi((h_1\theta(q_1)h_2, q_1q_2)) =\\
\phi(h_1q_1h_2q^{-1}, q_1q_2) = h_1q_1h_2q_2 = \phi((h_1, q_1))\phi((h_2, q_2)).
\end{multline*} $\square$
\end{proof}

Note that if $G$ is the internal direct product of subgroups $H$ and $Q$ with $Q$ also normal in $G$ then the semidirect product is the direct product.

Also note that whilst any two direct products of groups $H$ and $Q$ are isomorphic, this is not true of the semidirect product as the semidirect product still depends on the action of $Q$ on $H$.

In the case of the internal semidirect product, one should not confuse the action of $Q$ on $H$ by conjugation in $G$ with inner automorphisms of $H$ (which would correspond to conjugation by elements of $H$). As $H$ is normal in $G$, the action of $Q$ on $H$ must map $H$ to itself, but this map can be any automorphism of $H$, not just an inner automorphism.

\subsubsection{Dihedral Groups}

\begin{definition}
Let $D_n$ be the group generated by $r$ and $s$ with $r^n = 1$, $s^2 = 1$ and $(sr)^2 = 1$. We call $D_n$ the $n$-th \emph{dihedral group}.
\end{definition}

\begin{theorem}
The dihedral group $D_n$ has $2n$ elements, namely $1, r, \ldots, r^{n-1}$ and $s, sr, \ldots sr^{n-1}$.
\end{theorem}

\begin{proof}
From $(sr)^2 = 1$ we have $rs = s^{-1}r^{-1} = sr^{n-1}$. Repeated application of this identity allows us to replace any product of $s$ and $r$ with a product having at most one $s$ on the left. $\square$
\end{proof}

\begin{theorem}
The dihedral group $D_n$ is isomorphic to the semidirect product $C_n\times C_2$ where the the non-identity element of $C_2$ acts on $C_n$ by inverting elements, i.e. if $C_n = \langle r \rangle$ and $C_2 = \langle s \rangle$ then $srs^{-1} = r^{-1}$.
\end{theorem}

\begin{proof}
Note that we have identified $C_n$ and $C_2$ with subgroups of $D_n$ generated by $r$ and $s$ respectively.

We have $C_n \mathrel{\unlhd} D_n$ as it has index 2.

Clearly $C_n\cap C_2 = 1$ and $C_nC_2 = D_n$.

The given identity follows from $(sr)^2 = 1$ and $s^2 = 1$. $\square$
\end{proof}

\subsection{Exact Sequences}

\subsubsection{Exact Sequences}

\begin{definition}
An exact sequence of groups is a sequence of homomorphisms $f_i$ between groups $G_i$:
$$G_0 \overset{f_1}{\longrightarrow} G_1 \overset{f_2}{\longrightarrow} G_{2} \overset{f_{3}}{\longrightarrow} \cdots \overset{f_n}{\to} G_n$$
such that im$(f_k) =$ ker$(f_{k+1})$ for every $n$.
\end{definition}

\begin{theorem}
The sequence $0 \longrightarrow A \overset{f}{\longrightarrow} B$ is exact at $A$ iff $f$ is an injective homomorphism.
\end{theorem}

\begin{proof}
Clear. $\square$
\end{proof}

\begin{theorem}
The sequence $A \overset{f}{\longrightarrow} B \longrightarrow 0$ is exact at $B$ iff $f$ is an surjective homomorphism.
\end{theorem}

\begin{proof}
Clear. $\square$
\end{proof}

\begin{corollary}
The sequence $0 \longrightarrow A \overset{f}{\longrightarrow} B \longrightarrow 0$ is exact at $B$ iff $f$ is an isomorphism.
\end{corollary}

\subsubsection{Short Exact Sequences}

\begin{definition}
A \emph{short exact sequence} is an exact sequence of the form
$$0 \longrightarrow A \overset{f}{\longrightarrow} B \overset{g}{\longrightarrow} C \longrightarrow 0.$$
\end{definition}

\begin{theorem}
If $0 \longrightarrow A \overset{f}{\longrightarrow} B \overset{g}{\longrightarrow} C \longrightarrow 0$ is a short exact sequence of groups then $C \cong B/\mbox{im}(f)$.
\end{theorem}

\begin{proof}
Follows from the first isomorphism theorem and the definition of exactness at $B$. $\square$
\end{proof}

\subsubsection{Split Short Exact Sequences}

\begin{definition}
A short exact sequence $0 \longrightarrow A \overset{f}{\longrightarrow} B \overset{g}{\longrightarrow} C \longrightarrow 0$ is said to be \emph{split} if there exists a homomorphism $h : C \to B$ such that the composition $g\circ h$ on $C$ is the identity map.
\end{definition}

\begin{theorem}
Let $0 \longrightarrow A \overset{f}{\longrightarrow} B \overset{g}{\longrightarrow} C \longrightarrow 0$ be a short exact sequence of groups. If there exists a homomorphism $t : B \to A$ such that $t\circ f$ is the identity map on $A$ then $B$ is a direct sum of $A$ and $C$.
\end{theorem}

\begin{proof}
Define $\theta : B \to A\times C$ by
$$\theta(b) = (t(b), g(b)).$$

This is a homomorphism since $t$ and $g$ are homomorphisms.

We will show that the kernel of $\theta$ is trivial, so that it is injective. If $b \in$ ker$(\theta)$ then $t(b) = 1$ and $g(b) = 1$. From exactness at $B$ we see that $g(b) = 1$ implies $b = f(a)$ for some $a \in A$. If $t(b) = 1$ then $1 = t(f(a)) = a$. But this also means $b = f(1) = 1$.

To show that $\theta$ is surjective, let $(a, c) \in A\times C$. As $g$ is surjective, $c = g(b)$ for some $b \in B$.

Let $x = t(b)^{-1}a \in A$. We have $t(bf(x)) = t(b)t(f(x)) = t(b)x = a$. By exactness at $B$ we have that $g(bf(x)) = g(b)g(f(x)) = g(b) = c$. Thus $\theta(bf(x)) = (a, c)$ and $\theta$ is surjective. $\square$
\end{proof}

\begin{corollary}
Let $0 \longrightarrow A \overset{f}{\longrightarrow} B \overset{g}{\longrightarrow} C \longrightarrow 0$ be a short exact sequence of groups. If there exists a homomorphism $t : B \to A$ such that $t\circ f$ is the identity map on $A$ then the sequence is split.
\end{corollary}

\begin{proof}
Define $h : C \to B$ as the composition of the canonical injection of $C$ into $A\times C$ and the inverse of the map $\theta$ from the proof of the theorem.

Let $c \in C$ and $b = h(c) = \theta^{-1}((1, c))$. Then $g(b) = \rho_2(\theta(b))$, by definition of $\theta$, where $\rho_2$ is the projection onto the second component. But this is clearly $c$. Thus $h$ splits the sequence. $\square$
\end{proof}

The converse is not true, but we have the following.

\begin{theorem}
Let $0 \longrightarrow A \overset{f}{\longrightarrow} B \overset{g}{\longrightarrow} C \longrightarrow 0$ be a short exact sequence of groups. If the sequence is split then $B \cong A\rtimes_{\theta} C$ for some homomorphism $\theta : C \to$ Aut$(A)$.
\end{theorem}

\begin{proof}
Let $h : C \to B$ split the sequence. It suffices to show that $B$ is the internal direct product im$(f)\rtimes_{\theta}$ im$(h)$.

As im$(f) =$ ker$(g)$, it is a normal subgroup of $B$.

Firstly, we show that im$(f)$im$(h) = B$. Let $b \in B$. Then $b = yz$ where $z = h(g(b))$ and $y = bz^{-1}$. We have $g(z) = g(h(g(b))) = g(b)$. Thus $g(y) = g(b)g(z)^{-1} = 1$. Thus $y \in$ ker$(g) =$ im$(f)$. But $z \in$ im$(h)$ by construction, and so $b \in$ im$(f)$im$(h)$.

Now we show that if $x \in$ im$(f)\cap$ im$(h)$ then $x = 1$. For if $x = f(a) = h(c)$ for $a \in A$ and $c \in C$ then $c = g(h(c)) = g(f(a)) = 1$ due to exactness of the sequence. $\square$
\end{proof}

\begin{corollary}
Let $0 \longrightarrow A \overset{f}{\longrightarrow} B \overset{g}{\longrightarrow} C \longrightarrow 0$ be a short exact sequence of groups. If $A$, $B$ and $C$ are abelian and the sequence is split, then $C \cong A\times B$.
\end{corollary}

\begin{proof}
Clear. $\square$
\end{proof}

\section{Free Groups and Presentations}

\subsection{Free Groups}

\subsubsection{Free Groups}

\begin{definition}
Let $S$ be a set of symbols and let $S^{-1}$ be the set of symbols $s^{-1}$ for each symbol in $S$. Let $T = S\cup S^{-1}$ and let $W$ be the set of words constructed from symbols in $T$, including the empty word. A word can be \emph{reduced} by removing any pair of adjacent symbols of the form $ss^{-1}$ or $s^{-1}s$ from the word, for any $s \in S$. A word is said to be \emph{reduced} if no such adjacent pairs exist in the word. The \emph{free group} on $S$ is the set $F(S)$ of reduced words with concatenation followed by reduction as the operation.
\end{definition}

\begin{theorem}
The free group $F(S)$ on a set $S$ is a group.
\end{theorem}

\begin{proof}
The empty word is the identity, and the operation is clearly associative. The inverse of a word is defined to be the reverse of the word with each symbol $s$ replaced with $s^{-1}$ and vice versa. $\square$
\end{proof}

\begin{theorem}
Every group $G$ is a quotient of a free group.
\end{theorem}

\begin{proof}
Let $S$ be the underlying set of $G$.

Define a map $F(S) \to G$ by sending a reduced word in $F(S)$ to the corresponding product of elements and inverses in $G$.

This map is clearly well-defined and a surjective homomorphism. The result follows from the first isomorphism theorem. $\square$
\end{proof}

\subsection{Group Presentations}

\subsubsection{Group Presentations}

\begin{definition}
Let $S$ be a set and $R$ be a set of (reduced) words in $F(S)$. Let $N(R)$ be the smallest normal subgroup of $F(S)$ containing $R$ (the \emph{normal closure} of $R$), then the group $G = F(S)/N(R)$ is said to have \emph{presentation} $\langle S \;|\; R \rangle$.
\end{definition}

\begin{theorem}
Let $R$ be the kernel of the map $F(S) \to G$ in the previous theorem. Then $\langle S \;|\; R \rangle$ is a presentation of $G$.
\end{theorem}

\begin{proof}
The kernel is normal and so $N(R) = R$. $\square$
\end{proof}

\begin{definition}
A presentation $\langle S \;|\; K \rangle$ of a group $G$ is said to be finitely generated if $S$ is finite. It is said to be \emph{finitely related} if $K$ is finite, and a \emph{finite presentation} if both $S$ and $K$ are finite. The group $G$ is also said to be finitely generated or finitely related or finitely presented, respectively, if it has a presentation with the given properties.
\end{definition}

\begin{theorem}
Let $G$ be a group. There is a unique homomorphism $\phi : F(G) \to G$ which is the identity on elements of $G$.
\end{theorem}

\begin{proof}
The definition of $\phi$ on a word in $F(G)$ is fixed in view of $\phi$ being a homomorphism and from the fact that $gg^{-1}$ and $g^{-1}g$ are equal to the empty word after reduction, for any $g \in G$, so that $\phi(g)\phi(g^{-1}) = \phi(g^{-1})\phi(g) = 1$. $\square$
\end{proof}

\begin{corollary}
Every group has a presentation and every finite group has a finite presentation.
\end{corollary}

\begin{proof}
Clear. $\square$
\end{proof}

\subsubsection{Fundamental Theorem of Presentations of a Group}

\begin{theorem}
Let $G = \langle S \;|\; R \rangle$ be a presentation of a group. Let $\theta : S\cup S^{-1} \to H$ be a map to any group $H$ such that $\theta(s^{-1}) = \theta(s)^{-1}$ for any $s \in S$. Suppose that for all relations $r = t_1t_2\ldots t_k \in R$ with $t_i \in S\cup S^{-1}$ we have $\theta(t_1)\theta(t_2)\ldots \theta(t_k) = 1$. Then $\theta$ extends uniquely to a homomorphism $\theta' : G \to H$.
\end{theorem}

\begin{proof}
If such a homomorphism exists, it is clearly unique.

Define $\psi : F(X) \to H$ by $\psi(x) = \phi(x)$ for $x \in X$ and $x \in X^{-1}$ and by $\psi(x_1x_2) = \phi(x_1)\phi(x_2)$ for $x_1, x_2 \in X\cup X^{-1}$. This is clearly a well-defined homomorphism.

Now $\psi(r) = 1$ for all $r \in R$ and so $R \subseteq$ ker$(\psi)$.

Let $N =$ ker$(\psi)$ and $R'$ be the normal closure of $R$ in $F(X)$. Then $N \mathrel{\unlhd} F(X)$ and $R' \leq N$. Thus there is a well-defined homomorphism $\theta' : F(X)/R' \to H$. $\square$
\end{proof}

\subsubsection{Presentation of a Direct Product}

\begin{theorem}
Let $G_1 = \langle X_1 \;|\; R_1 \rangle$ and $G_2 = \langle X_2 \;|\; R_2 \rangle$ be groups with presentations such that $X_1$ and $X_2$ are disjoint. Then $G_1\times G_2 = \langle X_1\cup X_2 \;|\; R_1\cup R_2\cup [X_1, X_2] \rangle$.
\end{theorem}

\begin{proof}
Define $\theta : X_1\cup X_2 \to G_1\times G_2$ by $\theta(x_1) = (x_1, 1)$ and $\theta(x_2) = (1, x_2)$ for $x_i \in X_i$ and extend $\theta$ to $F(X_1\cup X_2)$ by multiplicativity.

Let $H = \langle X_1\cup X_2 \;|\; R_1\cup R_2\cup [X_1, X_2] \rangle$. Let $r = t_1t_2\ldots t_k$ be a relation in $H$ with $t_i \in X_1\cup X_2\cup X_1^{-1}\cup X_2^{-1}$.

If $r \in R_1$ then $\theta(r) = (r, 1) = (1, 1)$. Similarly if $r \in R_2$ then $\theta(r) = (1, 1)$. Finally, if $r = [x_1, x_2]$ for $x_i \in X_i$ then
$$\theta(r) = \theta(x_1)^{-1}\theta(x_2)^{-1}\theta(x_1)\theta(x_2) = (x_1, 1)^{-1}(x_2, 1)^{-1}(x_1, 1)(x_2, 1) = (1, 1).$$

Therefore by the Fundamental Theorem of Presentations, $\theta$ extends to a map $H \to G_1\times G_2$. It is clearly surjective.

Let $V = \langle X_1\cup X_2 \;|\; H \rangle$.

We will show that ker$(\theta) = 1$. Because elements of $X_1$ and $X_2$ commute in $V$, any element of $V$ can be written $v = s_1s_2\ldots s_mt_1t_2\ldots t_n$ with $s_i \in X_1\cup X_1^{-1}$ and $t_i \in X_2\cup X_2^{-1}$. Then $\theta(v) = (s_1s_2\ldots s_m, t_1t_2\ldots t_n)$.

Thus if $v \in$ ker$(\theta)$ then $s_1s_2\ldots s_m = 1$ and $t_1t_2\ldots t_n = 1$. Thus $\alpha = s_1s_2\ldots s_m$ is in the normal closure of $R_1$ and $\beta = t_1t_2\ldots t_n$ in the normal closure of $R_2$.

Thus both $\alpha$ and $\beta$ are in the normal closure of $H$. Thus both are $1$ in $V$. Thus $v = 1$. $\square$
\end{proof}

\section{Permutations}

\subsection{Permutations}

\subsubsection{Permutations}

\begin{definition}
A \emph{permutation} on a set $S$ of $n$ elements is a bijection $\phi : S \to S$. Such a permutation can be defined by its action on the $i$-th element of $S$, which we can denote
$$\left(\begin{array}{cccc}1 & 2 & \cdots & n\\ \phi(1) & \phi(2) & \cdots & \phi(n)\end{array}\right).$$
\end{definition}

\subsubsection{Cycles}

\begin{definition}
A \emph{cycle} $(a_1, a_2, \cdots, a_k)$ is a permutation $\phi$ that maps $a_i \mapsto a_{i+1}$ for $1 \leq i < k$ and $a_k \mapsto a_1$.
\end{definition}

\begin{theorem}
A permution $\phi$ on the set $S$ of $n$ elements can be written as a product of disjoint cycles.
\end{theorem}

\begin{proof}
Begin with $1 \in S$. Then apply $\phi$ repeatedly until we reach $1$ again (as $\phi$ is a bijection and $S$ is finite, this must eventually happen): $1 \mapsto a_2 \mapsto \cdots \mapsto 1$. This is a cycle.

If any elements of $S$ did not appear in this cycle, let $b_1$ be the first such element. Apply the same trick to get a cycle beginning with $b_1$. As $\phi$ is a bijection, the elements of this cycle must be distinct from the elements of the first cycle.

Apply the same trick until all elements of $S$ appear in a cycle. $\square$
\end{proof}

\begin{theorem}
Let $\alpha = (a_1, a_2, \ldots, a_s)$ be a cycle. Let $\sigma$ be a permutation. Then
$$\sigma\alpha\sigma^{-1} = (\sigma(a_1), \sigma(a_2), \ldots, \sigma(a_s)).$$
\end{theorem}

\begin{proof}
Let $b_i = \sigma(a_i)$ for $i \in \{1, 2, \ldots, s\}$.

For $i \in \{1, 2, \ldots, s-1\}$ we have
$$(\sigma\alpha\sigma^{-1})(b_i) = (\sigma\alpha)(a_i) = \sigma(a_{i+1}) = b_{i+1}.$$

We also have
$$(\sigma\alpha\sigma^{-1})(b_s) = \sigma\alpha(a_s) = \sigma(a_1) = b_1.$$

As $\alpha$ leaves any $x \not\in \{a_1, a_2, \ldots, a_s\}$ fixed, we have that $(\sigma\alpha\sigma^{-1})(y) = y$ for all $y \not\in \{b_1, b_2, \ldots, b_s\}$. $\square$
\end{proof}

\subsubsection{Transpositions}

\begin{definition}
A \emph{transposition} is a cycle of length $2$.
\end{definition}

\begin{theorem}
A cycle is a product of transpositions.
\end{theorem}

\begin{proof}
$(a_1, a_2, \ldots, a_s) = (a_s, a_{s-1})(a_s, a_{s - 2})\cdots(a_s, a_1)$. $\square$
\end{proof}

\subsubsection{The Symmetric Group}

\begin{definition}
The \emph{symmetric group} on the set $S$ of $n$ elements is the set of permutations of $S$. We denote it $S_n$.
\end{definition}

\begin{theorem}
The set $S_n$ is a group under composition of permutations.
\end{theorem}

\begin{proof}
The trivial permutation is the identity. As a permutation is a bijection, it is invertible. Thus inverses exist. Composition is clearly associative. $\square$
\end{proof}

\begin{theorem}
The symmetric group $S_n$ is generated by transpositions of the form $t_i = (i, i + 1)$.
\end{theorem}

\begin{proof}
We have already seen that any permutation can be written as a product of disjoint cycles, and each cycle can be written as a product of transpositions.

But $(a_i, a_j)$ for $j > i$ can be expressed as $(a_{j-1}, a_j)\ldots (a_{i+1}, a_{i+2})(a_i, a_{i+1})$. $\square$
\end{proof}

\begin{theorem}
$|S_n| = n!$.
\end{theorem}

\begin{proof}
Follows immediately by counting the number of possible permutation. $\square$
\end{proof}

\begin{theorem}
Two elements of $S_n$ are conjugate iff they have disjoint cycle representations with cycles of the same length.
\end{theorem}

\begin{proof}
Let $\alpha = c_1c_2\ldots c_k$ be a product of disjoint cycles in $S_n$. Let $\sigma$ be any permutation. Then
$$\sigma c_1c_2\ldots c_k\sigma^{-1} = (\sigma c_1\sigma^{-1})(\sigma c_2 \sigma^{-1})\ldots (\sigma c_k\sigma^{-1}).$$

But by the theorem on conjugation of cycles, this will have the same cycle structure as $\alpha$.

Conversely, let $\alpha = c_1c_2\ldots c_k$ and $\beta = d_1d_2\ldots d_k$ be permutations with the same cycle structure.

Let $a_1, a_2, \ldots a_s$ be the elements of the cycles of $\alpha$ in order (including $1$-cycles), and let $b_1, b_2, \ldots, b_s$ be the elements of the cycles of $\beta$ in order (including $1$-cycles). Then the permutation that sends $a_i$ to $b_i$ for all $1 \leq i \leq s$ will conjugate $\alpha$ to $\beta$. $\square$
\end{proof}

\subsubsection{The Alternating Group}

\begin{definition}
A permutation is even if it is a product of an even number of transpositions.
\end{definition}

\begin{theorem}
Let $\epsilon$ be the identity permutation. Then $\epsilon$ is an even permutation.
\end{theorem}

\begin{proof}
We will show that if $\epsilon$ is a product of $k \geq 2$ transpositions, then it is the product of $k - 2$ transpositions, which will prove the result, as the identity is not a transposition.

Let $\epsilon = s_1s_2\ldots s_k$ and suppose that $s_j = (x, a)$ but $x$ does not appear in $s_{j+1}, s_{j+2}, \ldots, s_k$.

Let $s_{j-1} = (y, b)$. There are four cases:

(i) If $s_{j-1} = (x, a)$ we are done.

(ii) If $s_{j-1} = (x, b)$ with $b \neq a$ and $b \neq x$ then $s_{j-1}s_j = (x, a)(a, b)$.

(iii) If $s_{j-1} = (y, a)$ with $y \neq x$, $y \neq a$ then $s_{j-1}s_j = (x, y)(y, a)$.

(iv) If $s_{j-1} = (y, b)$ with $y \neq x$, $y \neq a$, $b \neq x$ and $b\neq a$ then $s_{j-1}s_j = (x, a)(y, b)$.

In all cases, we may move the first appearance of $x$ back or cancel a pair of transpositions.

As $x$ cannot appear just once in the identity, we must always eventually end up in case (i). $\square$
\end{proof}

\begin{corollary}
Every permutation is either odd or even, but not both.
\end{corollary}

\begin{proof}
If a permutation were both, the identity could be written as a product of an even permutation and the inverse of an odd permutation, which would be odd. $\square$
\end{proof}

\begin{definition}
The \emph{alternating group} on $n$ elements is the subset of $S_n$ consisting of even permutations. We denote it $A_n$.
\end{definition}

\begin{theorem}
The set $A_n$ is a group under composition of permutations.
\end{theorem}

\begin{proof}
It is clearly a subgroup of $S_n$ because it is closed under composition and taking of inverses. $\square$
\end{proof}

\begin{theorem}
$|A_n| = n!/2$.
\end{theorem}

\begin{proof}
We show that half of the elements of $S_n$ are in $A_n$.

Let $\tau$ be a fixed transposition. Define $f : S_n \to S_n$ by $f(\sigma) = \tau\sigma$. This is a bijection between the sets of odd and even permutations. Note $f(\tau\sigma) = \tau\tau\sigma = \sigma$, which implies that $f$ is surjective, and it is clearly injective. $\square$
\end{proof}

\begin{theorem}
The group $A_n$ is generated by its $3$-cycles.
\end{theorem}

\begin{proof}
We note:

(i) $(a, b)(c, d) = (a, c, b)(a, c, d)$.

(ii) $(a, b)(b, c) = (a, b, c)$.

(iii) $(a, b)(a, b) = (a, b, c)^2$. $\square$
\end{proof}

\subsubsection{A Presentation of $S_n$}

\begin{theorem}
The symmetric group $S_n$ has the presentation
$$\langle T \;|\; t_i^2 = 1, (t_it_{i+1})^3 = 1, (t_it_j)^2 = 1 \;\;\mbox{for}\;\; |i - j| > 1 \rangle,$$
where $T = \{t_1, t_2, \ldots, t_{n-1}\}$.
\end{theorem}

\begin{proof}
Let the group in the theorem be denoted $G_n$.

We already know that $S_n$ is generated by transpositions $t_i = (i, i+1)$. It is also easy to check that the given relations hold in $S_n$.

Thus there is a surjective group homomorphism $\phi : G_n \to S_n$ which maps the symbol $t_i$ to the transposition $t_i$.

As $S_n$ is finite, it only remains to show that $|G_n| \leq n!$. We will prove this by induction, the basecase $n = 2$ being clear.

Suppose the result is true for $G_n$ and let $H$ be the subgroup of $G_{n + 1}$ generated by $\{t_1, t_2, \ldots, t_n\}$. By induction $|H| \leq n!$. We will investigate the cosets of $H$ in $G_{n+1}$.

Let $H_n = H$, $H_{n-1} = t_nH$, $H_{n-2} = t_{n-1}t_nH$, $\ldots$, $H_0 = t_1t_2\ldots t_nH$. We will show that these are precisely the cosest of $H$ in $G_{n+1}$ which will prove the result.

To prove this, it suffices to show that $\{H_0, H_1, \ldots, H_n\}$ is permuted by left multiplication by elements of $G_{n+1}$, since all such cosets include the cosets of $H = H_0$.

As $G_n$ is generated by $t_1, t_2, \ldots, t_n$ it is enough to show this for left multiplication by these elements. We will compute $t_iH_j$ for all $1 \leq i < n$ and $0 \leq j \leq n$.

It's clear that $t_iH_i = H_{i-1}$ and $t_iH_{i-1} = H_i$. Therefore, suppose that $j \neq i, i - 1$. We will show that $t_iH_j = H_j$ in these cases.

Let us denote $t_kt_{k_1}\ldots t_{l}$ by $P(k, l)$.

Firstly, if $j \geq i + 1$ we have
$$t_iH_j = t_iP(j+1, n)H = P(j+1, n)t_iH = P(j+1, n)H = H_j.$$

In the case $j \leq i - 2$ we have
\begin{multline*}
y_iH_j = y_iP(j+1, n)H = P(j+1, i-2)t_it_{i-1}t_iP(i+1, n)H =\\
 P(j+1, i-2)t_{i-1}t_it_{i-1}P(i+1, n)H = P(j+1, n)t_{i-1}H = P(j+1, n)H = H_j.
\end{multline*}
$\square$
\end{proof}

\section{Groups Acting on a Set}

\subsection{Groups Actions}

\subsubsection{Group Actions}

\begin{definition}
Let $A$ be a set and $G$ a group. A \emph{left group action} of $G$ on $A$ is a map $G\times A \to A$ denoted by $(g, a) \mapsto ga$ for all $g \in G$ and $a \in A$ such that the following hold.
\begin{enumerate}
\item $g_1(g_2a) = (g_1g_2)a$ for all $g_1, g_2 \in G$ and $a \in A$
\item $1a = a$ for all $a \in A$
\end{enumerate}
Similarly, a \emph{right group action} is a map $A\times G \to A$ such that
\begin{enumerate}
\item $(ag_1)g_2 = a(g_1g_2)$ for all $g_1, g_2 \in G$ and $a \in A$
\item $a1 = a$ for all $a \in A$
\end{enumerate}
\end{definition}

\begin{theorem}
Let $G$ be a group acting on a set $A$ on the left. For each $g \in G$ we let $\sigma_g : A \to A$ be defined by $\sigma_g(a) = ga$ for all $a \in A$. Then the following hold.
\begin{enumerate}
\item Each $\sigma_g$ is a permutation of $A$, i.e. $\sigma_g \in S_A$
\item The map $\varphi : G \to S_A$ defined by $\varphi(g) = \sigma_g$ is a group homomorphism
\item If $\psi : G \to S_A$ is any homomorphism then the map $G\times A \to A$ defined by $(g, a) \mapsto \psi(g)a$ is a left group action of $G$ on $A$
\end{enumerate}
\end{theorem}

\begin{proof}
1. It is easy to see that $\sigma_g\circ \sigma_{g^{-1}}$ is the identity on $A$ and similarly for $\sigma_{g^{-1}}\circ \sigma_g$. Thus $\sigma_g$ is a bijection and hence a permutation of $A$.

2. It is easy to show that $\sigma_{g_1g_2} = \sigma_{g_1}\circ\sigma_{g_2}$. Therefore $\phi(g_1g_2) = \varphi(g_1)\circ\varphi_{g_2}$.

3. Let $g_1, g_2 \in G$ and $a \in A$. Then it is easy to check that $g_1(g_2(a)) = \psi(g_1)(\psi(g_2)(a))$. Now on account of $\psi$ being a homomorphism, this is equal to $\psi(g_1g_2) = (g_1g_2)(a)$. But this is precisely the first requirement for a group representation.

For $a \in A$ we have $1a = \psi(1)(a) = a$. Thus we have a group action of $G$ on $A$. $\square$
\end{proof}

\begin{definition}
The map $\varphi : G \to S_A$ or more generally any homomorphism $\psi$ as in the theorem is called a \emph{permutation representation} of $G$ on the set $A$.
\end{definition}

\subsubsection{The Trivial Group Action}

\begin{theorem}
Let $G$ be a group and $A$ a set. The map $G\times A \to A$ defined by $(g, a) = a$ for all $g \in G$ and $a \in A$ is a group action.
\end{theorem}

\begin{proof}
Clear. $\square$
\end{proof}

\begin{definition}
The group action of the theorem is called the trivial group action of $G$ on $A$.
\end{definition}

\begin{example}
Let $\varphi : G \to S_A$ be the permutation representation given by $\varphi(g) = \;\;\mbox{id}_A$. Clearly $\varphi : G \to S_A$ is the trivial homomorphism of the group $G$ to the trivial subgroup $\{\mbox{id}_A\}$ of $S_A$.
\end{example}

\subsubsection{Regular Group Actions}

Groups can act on themselves by a left or right regular action.

\begin{definition}
Let $G$ be a group. The \emph{left regular action} of $G$ on itself is the group action defined by $(g, a) \mapsto ga$ for all $g, a \in G$, where $ga$ on the right hand side here stands for the group operation. The right regular action of $G$ on itself is defined similarly.
\end{definition}

\begin{theorem}
The left and right regular actions of $G$ on itself are left and right group actions, respectively.
\end{theorem}

\begin{proof}
Clear. $\square$
\end{proof}

\subsubsection{The Action of the Symmetric Group}

\begin{theorem}
Let $X$ be a nonempty set and $S_X$ be the symmetric group on $X$. Then the action defined by $(\sigma, x) = \sigma(x)$ for $\sigma \in S_X$ and $x \in X$ is a group action.
\end{theorem}

\begin{proof}
Clear. $\square$
\end{proof}

Let $\varphi : S_X \to S_X$ be the permutation representation defined by $\varphi(\delta) = \sigma_{\delta}$ for all $\delta \in S_X$ where $\sigma_\delta$ is defined by $\sigma_\delta(x) = \delta_x$.

Clearly $\varphi(\delta) = \delta$ for all $\delta \in S_X$ and so $\varphi$ is simply the identity homomorphism on $S_X$.

\subsubsection{The Group Action of Conjugation by a Subgroup}

\begin{theorem}
Let $G$ be a group and $H \leq G$. The map $(h, g) \mapsto hgh^{-1}$ is a group action of $H$ on $G$. We say that $H$ acts by conjugation on $G$.
\end{theorem}

\begin{proof}
Clear. $\square$
\end{proof}

\subsubsection{Faithful Group Actions}

\begin{definition}
Let $A$ be a set and $G$ a group acting on $A$. We say that the action is \emph{faithful} if the associated representation homomorphism $\varphi : G \to S_A$ defined by $\varphi(g) = \sigma_g$ where $\sigma_g(x) = gx$, is injective.
\end{definition}

\begin{theorem}
Let $A$ be a set and $G$ a group acting on $A$. Then $G$ acts faithfully on $A$ iff for every pair of distinct elements $g_1, g_2 \in G$ there exists an element $a \in A$ such that $g_1a \neq g_2a$.
\end{theorem}

\begin{proof}
This just says $\sigma_{g_1} \neq \sigma_{g_2}$. $\square$
\end{proof}

\begin{theorem}
Let $A$ be a set and $G$ a group acting on $A$. Then $G$ acts faithfully on $A$ iff for every $g \neq e$ in $G$ there exists $a \in A$ such that $ga \neq a$.
\end{theorem}

\begin{proof}
This just says that the kernel of $\varphi$ is trivial. $\square$
\end{proof}

\subsubsection{Transitive Group Actions}

\begin{definition}
Let $A$ be a set and $G$ a group acting on $A$. Then $G$ is said to act \emph{transitively} on $A$ if for every $a, b \in A$ there exists $g \in G$ such that $ga = b$.
\end{definition}

\subsubsection{Regular Group Actions}

\begin{definition}
The action of a group $G$ on a set $A$ is said to be \emph{regular} if it is faithful and transitive. We say that $A$ is a \emph{principal homogeneous space} or $G$-torsor.
\end{definition}

\begin{theorem}
The action of a group $G$ on itself by left multiplication is regular.
\end{theorem}

\begin{proof}
It is clearly transitive and free. $\square$
\end{proof}

\begin{corollary} (Cayley)
Every group $G$ can be embedded in $S_G$.
\end{corollary}

\begin{proof}
The associated permutation representation $\varphi : G \to S_G$ is injective as the action of a group on itself by left multiplication is faithful. $\square$
\end{proof}

\subsection{Orbits and Stabilisers}

\subsubsection{The Orbit and Stabiliser of a Point}

\begin{definition}
Let $G$ be a group acting on a nonempty set $A$ and let $a \in A$. The \emph{orbit} of $a$ under the action of $G$ is the set
$$Ga = \{b \in A \;|\; b = ga \;\;\mbox{for some}\;\; g \in G\}.$$
\end{definition}

\begin{definition}
Let $G$ be a group acting on a nonempty set $A$ and let $a \in A$. The \emph{stabiliser} of $a$ in $G$ is the set
$$G_a = \{g \in G \;|\; ga = a\}.$$
\end{definition}

\begin{theorem}
Let $G$ be a group acting on a set $A$. Then $G_a \leq G$.
\end{theorem}

\begin{proof}
$G_a$ is clearly closed under the group operation of $G$.

Clearly $e \in G_a$.

If $g \in G_a$ then
$$g^{-1}a = g^{-1}(ga) = (g^{-1}g)a = ea = a.$$
Thus $g^{-1} \in G_a$ and $G_a$ is a subgroup of $G$. $\square$
\end{proof}

\subsubsection{The Class Equation for Group Actions}

\begin{theorem}
Let $G$ be a group acting on a nonempty set $A$ and let $a \in A$. Then the orbit $Ga$ is in bijection with the cosets of $G_a$, i.e. $|Ga| = [G:G_a]$.
\end{theorem}

\begin{proof}
Let $f : G/G_a \to Ga$ be a map on sets ($G_a$ need not be normal in $G$) defined by $f(gG_a) = ga$.

We will show for $g_1, g_2 \in G$ that $g_1G_a = g_2G_a$ iff $g_1a = g_2a$ so that $f$ is well-defined and injective.

Clearly we have $g_1a = g_2a$ iff $(g_2^{-1}g_1)a = a$ iff $g_2^{-1}g_1 \in G_a$ iff $g_1G_a = g_2G_a$.

The map $f$ is clearly surjective. $\square$
\end{proof}

\begin{theorem}
Let $G$ be a group acting on a nonempty set $A$. The set of orbits of the action of $G$ on $A$ partition $A$.
\end{theorem}

\begin{proof}
It is clear that two elements $a, b \in A$ are in the same orbit iff there exists $g \in G$ such that $a = gb$.

But this is clearly an equivalence relation. $\square$
\end{proof}

\begin{definition}
Let $G$ be a group acting on a nonempty set $A$. The subset of $A$ \emph{fixed} by $G$ is the set
$$A^G = \{a \in A \;|\; ga = a \;\;\mbox{for all}\;\; g \in G\}.$$
\end{definition}

\begin{theorem} (Class Equation)
Let $G$ be a group acting on a finite, nonempty set $A$. Then
$$|A| = |A^G| + \sum [G:G_a],$$
where the sum runs over a set of representatives $a$ of the orbits $Ga$ with $|Ga| > 1$.
\end{theorem}

\begin{proof}
Clear from the preceding two theorems. $\square$
\end{proof}

In particular, note that the size of any orbit divides the order of the group $G$.

\subsubsection{The Class Equation for Conjugation}

\begin{theorem}
Let $G$ be a finite group. Then
$$|G| = |Z_G| + \sum [G:C_G(g)],$$
where the sum is over all nontrivial (more than one element) conjugacy classes in $G$.
\end{theorem}

\begin{proof}
Each element of the centre has trivial conjugacy class and the converse is also true.

The rest of the sum follows immediately from Theorem \ref{conjugacy}. $\square$
\end{proof}

\section{The Sylow Theorems}

\subsection{$p$-groups}

\subsubsection{$p$-groups}

\begin{definition}
A $p$-group for a prime $p$ is a group in which every element has order a power of $p$.
\end{definition}

\begin{theorem} (Cauchy)
If $G$ is a finite group whose order is divisible by a prime $p$ then it has an element of order $p$.
\end{theorem}

\begin{proof}
We prove the result first for abelian groups, using induction. If $|G| = p$ then any nontrivial element will do.

Let $a$ be any nontrivial element and let $H = \langle a \rangle$. If $p \;|\; |H|$ then $a^{|H|/p}$ is of order $p$. If not, then $p$ divides $[G:H]$ and by induction $G/H$ has an element of order $p$. Let that element be $xH$ and let the order of $x$ in $G$ be $m$. As $(xH)^m = H$ we have that $p \;|\; m$ and $x^{m/p}$ has order $p$.

When $G$ is not abelian, if $p$ divides $|Z_G|$ then by the first part, $Z_G$ contains an element of order $p$ and we are done. If not, as $p \;|\; |G|$ we have by the class equation there is some $a \in G$ such that $[G:C_G(a)]$ is not divisible by $p$. This means that $|C_G(a)|$ is divisible by $p$. As $C_G(a)$ is a proper subgroup it has an element of order $p$ by induction. $\square$
\end{proof}

\begin{theorem}
Let $p$ be a prime number. A finite group is a $p$-group iff it has order a power of $p$.
\end{theorem}

\begin{proof}
If a group $G$ has order $p^k$, every element of the group must have order dividing $p^k$ and so 
$G$ must be a $p$-group.

If $G$ does not have order $p^k$ for some $k \in \N$ then its order is divisible by some prime $q$ or it is the trivial group. In either case it is not a $p$-group. $\square$
\end{proof}

\subsubsection{Burnside's theorem for $p$-groups}

\begin{theorem} (Burnside)
The centre $Z_G$ of a nontrivial $p$-group $G$ is nontrivial.
\end{theorem}

\begin{proof}
If the centre is trivial, from the class equation, we must have that $[G:C_G(a)]$ is not a power of $p$ for some non-central $a \in G$. But this means that either $|G|$ is divisible by a prime $q \neq p$ in which case $G$ is not a $p$-group, or $[G:C_G(a)] = 1$, which means $|C_G| = |G|$. But the latter means that $a$ is central, which is a contradiction. $\square$
\end{proof}

\begin{theorem} \label{pgroupsubs}
If $G$ is a $p$-group of order $p^k$ it has a normal subgroup of order $p^m$ for all $1 \leq m \leq n$.
\end{theorem}

\begin{proof}
We proceed by induction. The result is trivial for $k = 0, 1$.

By the previous theorem, $|Z_G| = p^m$ for some $1 \leq m \leq k$. Thus $Z_G$ has an element of order $p$ by Cauchy's theorem. Let $H$ be the subgroup it generates. Note that $H$ is a normal subgroup since it is central. By the induction hypothesis, the result is true for $G/H$.

Therefore by the correspondence theorem $G$ has normal subgroups of order $p^m$ for $2 \leq m \leq k$. As $H$ has order $p^1$ the result is proved. $\square$
\end{proof}

\begin{theorem}
Let $G$ be a finite $p$-group and $H$ a proper subgroup of $G$. Then $H$ is a proper subgroup of $N_G(H)$.
\end{theorem}

\begin{proof}
Recall that $H \leq N_G(H)$. Suppose to the contrary of the theorem that $H = N_G(H)$ and that $G$ is the smallest group with this property. As $Z_G \leq N_G(H)$ we have that $Z_G \leq H$. But $Z_G$ is normal in $H$.

But $Z_G$ is nontrivial by the theorem above and so $G/Z_G$ furnishes a smaller group where the normalizer of the subgroup $H/Z_G$ is $N/Z_G$ and $N/Z_G = N/Z_G$, which is a contradiction of the minimality of $G$. $\square$
\end{proof}

\subsubsection{Basic results regarding $p$-groups}

\begin{theorem}
Let $G$ be a finite $p$-group. Every nontrivial normal subgroup $N$ of $G$ intersects $Z_G$ nontrivially.
\end{theorem}

\begin{proof}
As $N$ is a normal subgroup, it is a union of conjugacy classes. Each conjugacy class has $[G:C_G(a)]$ elements for $a$ in the conjugacy class. But this is a power of $p$ divisible by $p$ unless $a$ is in the centre of $G$.

As $N$ has order a power of $p$, the number of elements of $N$ in the centre of $G$ must also be divisible by $p$. $\square$
\end{proof}

\begin{theorem}
A group $G$ of order $p^2$ is abelian.
\end{theorem}

\begin{proof}
If $G$ is not abelian then $Z_G$ is a proper subgroup of $G$. But it is not trivial by Burnside's theorem and so we must have $|Z_G| = p$.

Let $g \in G$ but $g \notin Z_G$. Note that $g \in C_G(g)$ and $Z_G \leq C_G(g)$. But this implies $|C_G(g)| = p^2$, which implies $C_G(g) = G$ and thus $C_G(g) = G$ which is a contradiction as this would imply $g \in Z_G$. $\square$
\end{proof}

\begin{theorem}
The centre of a non-abelian group $G$ of order $p^3$ has order $p$.
\end{theorem}

\begin{proof}
We have that $|Z_G| \in \{1, p, p^2, p^3\}$. As $G$ is non-abelian, $|Z_G| \neq p^3$. As $G$ is a $p$-group $|Z_G| \neq 1$.

Assume $|Z_G| = p^2$. As $Z_G \mathrel{\unlhd} G$ the group $G/Z_G$ has order $p$ and is therefore cyclic. Thus $G$ is abelian.

The only remaining case is $|Z_G| = p$. $\square$
\end{proof}

\subsection{Sylow $p$-subgroups and the Sylow Theorems}

\subsubsection{Sylow $p$-subgroups}

\begin{definition}
Let $G$ be a finite group of order $n$ and let $p$ be a prime dividing $n$. A \emph{Sylow $p$-subgroup of $G$} is a subgroup of order $p^k$ where $p^i \;|\; n$ for $1 \leq i \leq k$ but not for $i = k + 1$.
\end{definition}

\begin{definition}
The set of Sylow $p$-subgroups of a finite group $G$ for which $p \;|\; n$ is denoted Syl$_p(G)$. We will denote the number of such groups by $n_p$.
\end{definition}

\subsubsection{Intersection of Sylow Subgroups}

\begin{theorem}
Let $G$ be a finite group of order $n$ and let $p$ and $q$ be distinct primes that divide $n$. A Sylow $p$-subgroup $H$ of $G$ and a Sylow $q$-subgroup $K$ of $G$ intersect trivially.
\end{theorem}

\begin{proof}
The order of $H$ is a power of $p$ and the order of $K$ is a power of $q$.

But $H \cap K$ is a subgroup of both $H$ and $K$, thus it has order dividing a power of $p$ and a power of $q$. $\square$
\end{proof}

\subsubsection{The Sylow Theorems}

\begin{theorem} (First Sylow)
Let $G$ be a finite group of order $n$ and let $p$ be a prime. If $p^k \;|\; n$ then $G$ has a subgroup of order $p^k$.
\end{theorem}

\begin{proof}
First we consider the case where $p$ divides the order of $Z_G$. By Cauchy's Theorem, $Z_G$ has an element $a$ of order $p$. As $a \in Z_G$ the subgroup $\langle a \rangle$ is normal in $G$.

By induction the quotient group $G/\langle a \rangle$ has a subgroup $M$ of order $p^{k-1}$. It is easy to see that the preimage of $M$ under the quotient homomorphism is a subgroup of $G$ of order $p^k$.

Now consider the case where $p$ does not divide the order of $Z_G$. From the class equation we see that there is an element $g \in G$ for which $p$ does not divide $[G:C_G(g)]$. We see that $p^k$ must divide the order of $C_G(g)$ and by induction this must have a subgroup of order $p^k$. $\square$
\end{proof}

\begin{corollary}
If $G$ is a finite group of order $n$ and $p$ is a prime for which $p^k \;|\; n$ then $G$ has a subgroup of order $p^k$.
\end{corollary}

\begin{proof}
Follows immediately from the First Sylow Theorem and Theorem \ref{pgroupsubs}. $\square$
\end{proof}

\begin{corollary}
A group $G$ is a $p$-group iff the order of every element of $G$ is a power of $p$.
\end{corollary}

\begin{proof}
If $G$ has an element of order divisible by $q$ for some prime $q \neq p$ then $q$ divides the order of $G$ and it is not a $p$-group.

Conversely, if $G$ is not a $p$-group it has order divisible by a prime $q \neq p$ which means that it has an element of order $q$. $\square$
\end{proof}

\begin{theorem} (Second Sylow)
Let $G$ be a finite group, then the number $n_p$ of Sylow $p$-subgroups of $G$ satisfies $n_p \equiv 1 \pmod{p}$.
\end{theorem}

\begin{proof}
Let $P$ be a Sylow $p$-subgroup. Let $g \in G$ have order a power of $p$ and $gPg^{-1} = P$. We claim $g \in P$. 

Consider the subgroup $R$ of $G$ generated by $P$ and $g$. We have $g \in N_G(P)$ and so $R \leq N_G(P)$. Thus $P \mathrel{\unlhd} R$.

We have that $|R| = |R/P||P|$. But $R/P$ is generated by $gP$ which has order a power of $p$. Thus $|R|$ is a power of $p$.

We note that $G$ acts on Syl$_p(G)$ by conjugation. Let $Q$ be a Sylow $p$-subgroup of $G$ distinct from $P$. Then $Q$ cannot be fixed under conjugation by elements of $P$ because of the claim above.

Let $O$ be the orbit of $Q$ under conjugation by elements of $P$. By the orbit-stabilizer theorem $|O| = [P:P_Q]$ where $P_Q$ is the stabiliser of $Q$ under the action of $P$ by conjugation.

As $|P|$ is a power of $p$, $|O|$ must be a power of $p$. We cannot have $|O| = 1$ since $Q$ is not fixed by the whole of $P$.

The set of all Sylow $p$-subgroups is a union of $P$-orbits. The only orbit of order $1$ is $\{P\}$. Thus $n_p \equiv 1 \pmod{p}$. $\square$
\end{proof}

\begin{theorem} (Third Sylow)
Any two Sylow $p$-subgroups of a finite group $G$ are conjugate.
\end{theorem}

\begin{proof}
Let $P$ be a Sylow $p$-subgroup of $G$. Let $S$ be the set of all $G$-conjugates of $P$. We see that $S$ is invariant under the action of $P$ by conjugation and that $P \in S$.

As per the proof of the second Sylow theorem, $|S| \equiv 1 \pmod{p}$.

If $S$ does not cover the whole of $G$, choose $Q$ a Sylow $p$-subgroup of $G$ not in $S$. Let $T$ be the set of $G$-conjugates of $Q$. We also have $|T| \equiv 1 \pmod{p}$ by a similar argument to that for $S$ and $P$.

But $T$ is invariant under the action of $P$ by conjugation and $P \notin T$. Thus $|T| \equiv 0 \pmod{p}$, which is a contradiction. Thus $S$ must cover the whole of $G$. $\square$
\end{proof}

\begin{theorem} (Fourth Sylow)
If $G$ is a finite group of order $n$ and $n = p^km$ for a prime $p$ such that $p$ does not divide $m$ then $n_p$ divides $m$.
\end{theorem}

\begin{proof}
Let $P$ be a Sylow $p$-subgroup of $G$. The number of conjugates of $P$ is given by $[G:N_G(P)]$ by the orbit-stabiliser theorem.

As $P \leq N_G(P)$ we have that $|N_G(P)|$ is divisibly by $p^k$. Thus the number of conjugates of $P$ must divide $m$.

But there are $n_p$ conjugates of $P$, since all the Sylow $p$-subgroups are conjugate. $\square$
\end{proof}

\subsubsection{Theorems Regarding Sylow Subgroups}

\begin{theorem}
Let $H$ be a $p$-group which is a subgroup of a finite group $G$. Then $H$ is a contained in a Sylow $p$-subgroup of $G$.
\end{theorem}

\begin{proof}
Let $H$ act on Syl$_p(G)$ by conjugation. The size of the orbit of $P \in$ Syl$_p(G)$ under this action is $[H:H_P]$ where $H_P$ is the stabiliser of $P$ under the action of $H$. But this is a power of $p$.

As the size of Syl$_p(G)$ is not a multiple of $p$, there must be an orbit with a single element, $P$ say. Consider the subgroup $HP$ of $G$ generated by $P$ and $H$. We have that $P$ is a normal subgroup of $HP$.

By the second isomorphism theorem $HP/P \cong H/(H\cap P)$. The order of the group on the left is coprime to $p$ because $P$ is a Sylow $p$-subgroup of $G$. The order of the group on the right is a power of $p$ as it divides the order of $H$. 

Thus both groups are trivial and $H \subseteq P$. $\square$
\end{proof}

\begin{theorem}
An abelian group $G$ is the inner direct product of its Sylow subgroups.
\end{theorem}

\begin{proof}
There is a unique Sylow $p$-subgroup for each prime $p$ dividing the order of $G$, as all the Sylow $p$-subgroups are conjugate and $G$ is abelian.

We can proceed by induction on the number $n$ of primes dividing the order of $G$. Clearly if $n = 1$ the result holds.

Let $P_1, P_2, \ldots, P_n$ be the Sylow $p$-subgroups of $G$ of order $p_1^{a_1}, \ldots, p_n^{a_n}$ respectively. We have by induction that $H = P_1P_2\ldots P_{n-1}$ is the direct product of $P_1, P_2, \ldots, P_{n-1}$. By induction it has order $m = p_1^{a_1}\ldots p_{n-1}^{a_{n-1}}$.

We see that the subgroup of $G$ generated by $H$ and $P_n$ has order divisible by $mp_n^{a_n}$ which is the order of $G$. Thus $HP_n = G$.

We must have that $H\cap P_n = \emptyset$ since every element of $H$ has order dividing $m$ and every element of $P_n$ has order dividing $p_n$.

A similar argument will apply if we replace $P_n$ in the argument by any of the $P_i$. As $G$ is abelian, all the $P_i$ are normal in $G$. Thus $G$ is the direct product of the $P_i$. $\square$
\end{proof}

\subsection{Groups of Prescribed Order}

\subsubsection{Groups of Order $pq$}

\begin{theorem}
Let $G$ be a group of order $pq$ where $p < q$ are primes. Then there are two possibilities:
\begin{enumerate}
\item $G$ is cyclic of order $pq$
\item $G$ is non-abelian with generators $a, b$ such that $a^p = b^q = 1$ and $a^{-1}ba = b^r$ for some $r \not\equiv 1 \pmod{q}$, $r^p \equiv 1 \pmod{p}$ and $p \;|\; q - 1$.
\end{enumerate}
\end{theorem}

\begin{proof}
Let $n_p, n_q$ be the number of Sylow $p$-subgroups and $q$-subgroups respectively, of $G$.

By the Sylow theorems, $n_q = 1 + kq$ for some $k \in \Z$ and $n_q$ divides $p$. Thus $n_q = 1$. Thus there is one Sylow $q$-subgroup and it is therefore normal in $G$. Denote it by $Q$.

Let $P$ be any $p$-Sylow subgroup. Then $Q\cap P = 1$ and $QP = G$, so $G$ is a semidirect product of $Q$ and $P$.

We will determine the action of $P$ on $Q$ by conjugation.

If $p$ does not divide $q - 1$ then $n_p = 1 + mp$ does not equal $q$ but divides $q$, so it must equal $1$. Then $G$ is a direct product of $Q$ and $P$ and therefore cyclic of order $pq$.

If $p$ does divide $q - 1$ then Aut$(Q) \cong C_{q-1}$, which has a unique subgroup $P'$ of order $p$. The map $\varphi : P \to$ Aut$(Q)$ must be either trivial, which is the case above, or it must map to the unique subgroup of order $p$.

Let $a$ and $b$ be generators of $P$ and $Q$ respectively. Let the action of $a$ on $Q$ by conjugation be $x \mapsto x^r$ where $r \neq 1 \pmod{q}$. By the above argument we must have $r^p \equiv 1 \pmod{p}$.

Choosing a different such $r$ is equivalent to choosing a different generator for $P$ and gives an isomorphic group $G$. Thus there is a unique group $G$ up to isomorphism in this case, and $G$ is clearly non-abelian. $\square$
\end{proof}

\subsubsection{Groups of Order $p^2q$}

\begin{theorem}
Let $G$ be a group of order 12. Then $G$ has one Sylow $3$-subgroup or it is isomorphic to $A_4$ which has one Sylow $2$-subgroup.
\end{theorem}

\begin{proof}
By the Sylow theorems, $n_3 \;|\; 4$ and $n_3 \equiv 1 \pmod{3}$. Thus $n_3 = 1$, in which case we are done, or $n_3 = 4$.

We will first show that $A_4$ has four Sylow $3$-subgroups.

We have $|A_4| = 12$. There are eight elements that are $3$-cycles, three elements that are a pair of disjoint transpositions and the identity element.

It is easy to see there are four Sylow $3$-subgroups, each consisting of the identity and two $3$-cycles.

Similarly, there is only one Sylow $2$-subgroup consisting of the identity and three pairs of disjoint $2$-cycles.

Returning to the general case, suppose that $G$ has four Sylow $3$-subgroups, i.e. Syl$_3(G) = \{H_1, H_2, H_3, H_4\}$. We have $H_i\cap H_j = 1$ for $i \neq j$. $G$ acts on Syl$_3(G)$ by conjugation.

We claim $N_G(H_i) = H_i$. This follows because all Sylow $3$-subgroups are conjugate. Thus there is only one orbit of the action of $G$, from which the theorem follows by the orbit-stabiliser theorem and the fact that $H_i \leq N_G(H_i)$.

Let $\varphi : G \to S_4$ be a permutation representation associated to the action of $G$ on Syl$_3(G)$. We will show the kernel of the map is trivial, i.e. the representation is faithful.

But if $g \in$ ker$\varphi$ it must be in the normaliser of each of the $H_i$ and thus in each of the $H_i$. But we showed this was impossible as the $H_i$ intersect trivially. Thus the permutation representation is injective.

As $G$ contains eight distinct elements of order $3$ each including the identity and the two other elements of one of the $H_i$, its image in $S_3$ must also have eight such elements. Thus its intersection with $A_4$ must contain these elements. But $A_4$ can't have a subgroup of more than $6$ elements, so the image must in fact be $A_4$.

Thus $G$ is isomorphic to $A_4$. $\square$
\end{proof}

\begin{theorem}
Let $G$ be a group of order $p^2q$ for distinct primes $p$ and $q$. Then $G$ contains a normal Sylow subgroup.
\end{theorem}

\begin{proof}
If $p > q$ then $n_p \equiv 1 \pmod{p}$ and $n_p \;|\; q$. Thus $n_p = 1$ and the Sylow $p$-subgroup is normal in $G$.

If $q > p$ then $n_q \equiv 1 \pmod{q}$ and $n_q \;|\; p^2$. If $n_q = 1$ we are done. Thus the only remaining case is $n_q = p^2$ and $n_q \equiv 1 \pmod{q}$.

But then $p^2 = 1 + kq$ for some $k \in \mathbb{Z}$. Thus $q \;|\; p + 1$. But $q > p$ and so $q = p + 1$. This implies $p = 2$ and $q = 3$ as both $p$ and $q$ are prime.

Thus $|G| = 12$. The result follows by the previous theorem. $\square$
\end{proof}

\subsubsection{Groups of Order $p^2$}

\begin{theorem}
Let $G$ be a group of order $p^2$ for some prime $p$. Then either $G$ is cyclic of order $p^2$ or $G \cong C_p\oplus C_p$.
\end{theorem}

\begin{proof}
Recall that a group of order $p^2$ is abelian.

Let $g \in G$ be an element other than the identity. If the order of $g$ is $p^2$ then $G$ is cyclic.

Otherwise, $g$ has order $p$. Now let $h \in G$ be a element that is not in $\langle g \rangle$. Clearly $\langle g \rangle$ and $\langle g \rangle$ intersect trivially.

It's also clear that $g$ and $h$ generate the whole of $G$. And as the group $G$ is abelian, this means that $G = \langle g \rangle \oplus \langle h \rangle$. $\square$
\end{proof}

\subsubsection{Groups of Order $30$}

\begin{theorem}
There are four isomorphism classes of groups $G$ of order $30$, of which one is abelian.
\end{theorem}

\begin{proof}
We note that $n_3$ is either $1$ or $10$ and $n_5$ is either $1$ or $6$. However we cannot have both $n_3 = 10$ and $n_5 = 6$. Thus either $n_3 = 1$ or $n_5 = 1$.

Therefore $G$ has a subgroup $P_3$ or $P_5$ of order $3$ or $5$ respectively which is normal in $G$. Thus $H = P_3P_5$ is a subgroup of $G$ of order $15$.

By the classification of groups of order $pq$ and the fact that $5 \not\equiv 1 \pmod{3}$ we have that $H$ is cyclic.

As $H$ has index $2$ in $G$ it must be normal. Thus $G \cong H\rtimes_{\theta} Q$ for some homomorphism $\theta : Q \to$ Aut$(H)$.

We have that Aut$(H) \cong C_4\times C_2$. Let $p$ be a generator of $C_4$ and $q$ a generator of $C_2$. Then there are only four possible images of the non-identity element of $Q$ under $\theta$, namely $(1, 1)$, $(p^2, 1)$, $(1, q)$ and $(p^2, q)$. The only one of these that results in an abelian group is the trivial homomorphism. $\square$
\end{proof}

\section{Simple Groups}

\subsection{Simple Groups}

\subsubsection{Simple Groups}

\begin{definition}
A group $G$ is said to be \emph{simple} if $G$ has no proper, nontrivial, normal subgroups.
\end{definition}

\subsubsection{Cyclic Groups of Prime Order are Simple}

\begin{theorem}
The cyclic group $C_p$ with $p$ prime is simple.
\end{theorem}

\begin{proof}
There are no subgroups other than the trivial and improper subgroup. $\square$
\end{proof}

\subsubsection{Finite Abelian Simple Groups}

\begin{theorem}
The only finite abelian simple groups are the cyclic groups of prime order.
\end{theorem}

\begin{proof}
A subgroup of an abelian group is always normal. Therefore if $G$ is a finite, simple abelian group it has no subgroups.

If $g \in G$ has order $n$ for $n < |G|$ then $\langle g \rangle$ is a normal subgroup of $G$. Thus $G$ must be cyclic.

Similarly, any cyclic group of non-prime order has a proper, normal subgroup other than the trivial group. $\square$
\end{proof}

\subsection{Simple Alternating Groups}

\subsubsection{The Group $A_5$ is Simple}

\begin{theorem}
If $G$ is a group of order $60$ and $n_5 > 1$ then $G$ is simple.
\end{theorem}

\begin{proof}
If $n_5 > 1$ then by the Sylow theorems, $n_5 = 6$. This accounts for $25$ of the elements of $G$. Let Syl$_5(G) = \{P_1, P_2, \ldots, P_6\}$. We have $P_i\cap P_j = \{1\}$ for all $i \neq j$.

Suppose that $G$ has a nontrivial, normal proper subgroup. We claim that $|H|$ is not divisible by $5$. If it were, $H$ would have a subgroup of order $5$. It will be one of the Sylow $5$-subgroups of $G$, $P_1$ say.  

But $H \mathrel{\unlhd} G$ and so all the conjugates of $P_1$ must be in $H$. Thus $|H| \geq 25$.

As $|H| \;|\; |G|$ we must have $|H| = 30$. But we have shown above that all groups of order $30$ have a single Sylow $5$-subgroup.

Thus we must in fact have $|H| \;|\; 12$.

But any group $H$ of order $12$ has a normal subgroup $K$ of order $3$ or order $4$. As $H$ is normal in $G$, any conjugate of $K$ in $G$ must be a subgroup of $H$ and since $K$ is the only subgroup of that order, $K$ must be normal in $G$.

Similarly, if $H$ has order $6$, it contains a normal subgroup of order $3$, and by the same argument as above, this is also normal in $G$.

Thus we have shown that if $G$ has a nontrivial, proper normal subgroup, it has a normal subgroup $K$ of order $4$, $3$ or $2$.

Thus $G/K$ has order $15$, $20$ or $30$.

In a group of order $15$, $20$ or $30$ there is a normal Sylow subgroup of order $5$. But the preimage of this group in $G$ is a normal subgroup of order $20$, $15$ or $10$.

But each of these orders is divisible by $5$, which contradicts what we have proved above. $\square$
\end{proof}

\begin{theorem}
The group $A_5$ is simple.
\end{theorem}

\begin{proof}
Note that $|A_5| = 60$.

$A_5$ has one identity element, twenty $3$-cycles, fifteen double $2$-cycles and twenty four $5$-cycles.

The result then follows from the previous theorem by noting that $n_5 = 6$. $\square$
\end{proof}

\subsubsection{The Group $A_n$ is Simple for $n \geq 5$}

\begin{theorem}
The group $A_n$ is simple for $n \geq 5$.
\end{theorem}

\begin{proof}
We will prove the result by induction on $n$, beginning with the fact that $A_5$ is simple by the previous theorem.

Suppose $A_n$ is simple for $5 \leq n < k$, but also suppose that it is not simple for $n = k$, i.e. there exists $H \mathrel{\unlhd} A_k$.

Let $G_i$ be the stabiliser of the $i$-th element of the set $A_k$ is acting on. Each of the $G_i$ are isomorphic to $A_{k-1}$ and therefore simple.

Recall that if $\sigma = (a_1 a_2 \ldots a_s)$ is a cycle then $\alpha\sigma\alpha^{-1} = (\alpha(a_1) \alpha(a_2) \ldots \alpha(a_s))$. Thus it is clear that $\alpha G_i \alpha^{-1} = G_{\alpha_i}$.

We will show that if $\sigma \in H$ and $\sigma(i) = i$ for some $1 \leq i \leq k$ then $\sigma = 1$.

If a $\sigma \neq 1$ exists in $H$ then $\sigma \in H\cap G_i$. But $H\cap G_i \mathrel{\unlhd} G_i$ since $H$ is a normal subgroup of $A_k$ so any conjugate of it is in $H$, and conjugation by an element of $G_i$ must yield another element of $G_i$.

But $G_i$ is simple and so $H\cap G_i$ must be $G_i$ since it is not trivial as $\sigma \neq 1$ is an element. Thus $G_i \leq H$.

By conjugation we have that $G_{\alpha(i)} \leq H$ for each $\alpha \in A_k$. Thus $G_i \leq H$ for all $1 \leq i \leq k$.

But every double transposition fixes some $i$ and is an element of one of the $G_i$. Thus the whole of $A_k$ must be in $H$, i.e. $H = A_k$. But this is a contradiction as we assumed $H$ to be a proper subgroup of $A_k$. Thus $\sigma = 1$ as claimed.

In other words, no non-identity elements of $H$ fix an $i$.

But this implies that there do not exist $\sigma_1, \sigma_2 \in H$ such that $\sigma_1(i) = \sigma_2(i)$ for some $1 \leq i \leq k$.

We will now show that no element of $H$ has a cycle decomposition containing a cycle of length $3$ or greater.

Suppose that such a $\sigma = (a_1 a_2 a_3 \ldots)$ exists. Let $\alpha \in A_k$ fix $a_1$ and $a_2$ but not fix $a_3$.

Now $\alpha\sigma\alpha^{-1} = (a_1 a_2 \alpha(a_3) \ldots) \in H$ as $H$ is normal. But now have that $\sigma(a_1) = \alpha\sigma\alpha^{-1}(a_1)$. These are distinct permutations because $\sigma(a_2) \neq \alpha\sigma\alpha^{-1}(a_2)$. But this contradicts what we have proved above.

Thus $H$ consists entirely of elements composed of pairs of pairwise disjoint transpositions.

Suppose that $H$ contains $\sigma = (a_1 a_2)(a_3 a_4)(a_5 a_6)\ldots$. Let $\alpha = (a_1 a_2)(a_3 a_5)$. Then $\alpha\sigma\alpha^{-1} = (a_2 a_1)(a_5 a_4)(a_3 a_6)$. Again we have a contradiction to what we proved above as we have distinct permutations in $H$ that send $a_1$ to the same place.

But now the only element that can be in $H$ is the identity and $A_k$ must be simple. $\square$
\end{proof}

\subsubsection{$A_3$ is Simple, $A_4$ is not Simple}

\begin{theorem}
$A_3$ is simple.
\end{theorem}

\begin{proof}
$|A_3| = 3$ and is  cyclic group of order $3$ and is therefore simple. $\square$
\end{proof}

\begin{theorem}
$A_4$ is not simple.
\end{theorem}

\begin{proof}
$|A_4| = 12$ and we have already shown that either $n_2 = 1$. $\square$
\end{proof}

\section{Normal Series}

\subsection{Composition Series}

\subsubsection{Composition Series}

\begin{definition}
Let $G$ be a group. A \emph{composition series} for $G$ of length $n$ is a finite chain of subgroups
$$1 = G_0 \leq G_1 \leq \cdots \leq G_n = G$$
satisfying the following:
\begin{enumerate}
\item $G_i \mathrel{\unlhd} G_{i + 1}$
\item $G_{i+1}/G_i$ is simple
\end{enumerate}
The quotients $G_{i+1}/G_i$ are called the \emph{composition factors} of the series.
\end{definition}

\begin{theorem}
Any finite group $G$ has a composition series.
\end{theorem}

\begin{proof}
Start with the group $G_n = G$ and at each step take a maximal, normal, proper subgroup, $G_i$ of $G_{i+1}$, which exists because $G$ is finite.

The quotient $G_{i+1}/G_i$ is simple by the correspondence theorem and the maximality of $G_i$ in $G_{i+1}$.

The process must terminate as $G$ is finite. $\square$
\end{proof}

\subsubsection{The Jordan-H\"{o}lder Theorem}

\begin{theorem}
Let $G$ be a group. The composition factors of any two composition series for $G$ are isomorphic in pairs, though the order of the factors may be different.
\end{theorem}

\begin{proof}
We proceed by induction on $|G|$. The theorem is true for $|G| = 2$.

Assume the result is true for all groups of order less than $n$ and suppose $G$ is a group of order $n$.

If $G$ is simple, the theorem is trivially true. Otherwise, suppose that
$$1 \mathrel{\unlhd} A_r \mathrel{\unlhd} \cdots \mathrel{\unlhd} A_1 \mathrel{\unlhd} A \mathrel{\unlhd} G$$
and
$$1 \mathrel{\unlhd} B_s \mathrel{\unlhd} \cdots \mathrel{\unlhd} B_1 \mathrel{\unlhd} B \mathrel{\unlhd} G$$
are two composition series for $G$.

If $A = B$ then $G/A = G/B$ and by the inductive hypothesis all the other quotients are the same in pairs, up to order, and we are done.

If $A \neq B$, consider the group $AB$ of $G$. Now $A$ and $B$ are distinct and maximal in $G$, and thus $AB = G$.

Now let $D = A\cap B$. By the second isomorphism theorem 
\begin{equation}\label{gabd}
G/A \cong B/D\;\;\mbox{and}\;\;G/B \cong A/D.
\end{equation}
But $G/A$ and $G/B$ are simple, thus so are $B/D$ and $A/D$. In other words, $D$ is a maximal normal subgroup of $A$ and $B$.

Let $1 \mathrel{\unlhd} D_t \mathrel{\unlhd} \cdots \mathrel{\unlhd} D_1 \mathrel{\unlhd} D$ be a composition series for $D$.

Consider the quotient groups
\begin{equation}\label{comp1}G/A, A/D, D/D_1, \ldots, D_t, 1,\end{equation}
and
$$G/A, A/A_1, \ldots, A_r, 1.$$
By the induction hypothesis, the theorem is true for $A$ and so the sequences above are isomorphic in pairs. In particular, $t = r$.

Similarly, the quotients
\begin{equation}\label{comp2}G/B, B/D, D/D_1, \ldots, D_t, 1\end{equation}
is isomorphic in pairs with
$$G/B, B/B_1, \ldots, B_s, 1,$$
and $s = r$.

But the sequences \eqref{comp1} and \eqref{comp2} are clearly isomorphic in pairs by \eqref{gabd} and so we are done. $\square$
\end{proof}

% fundamental theorem of finitely generated abelian groups
% transfer map
% solvable groups
% central series
% nilpotent groups

\end{document}

