\documentclass[a4paper,10pt]{amsart}
\usepackage{amsfonts}
\usepackage{amsmath}
\usepackage{eucal}
\usepackage{amscd}
\usepackage{url}
\usepackage[named]{algo}

\newcommand{\Z}{\mathbb{Z}}
\newcommand{\N}{\mathbb{N}}
\newcommand{\Q}{\mathbb{Q}}
\newcommand{\I}{\mathbb{I}}
\newcommand{\C}{\mathbb{C}}
\newcommand{\R}{\mathbb{R}}
\newcommand{\Pee}{\mathbb{P}}
\newcommand{\EuO}{\mathcal{O}}
\newcommand{\Qbar}{\overline{\mathbb{Q}}}

\newcommand{\ljk}[2]{\left(\frac{#1}{#2}\right)}
\newcommand{\modulo}[1]{\;\left(\mbox{mod}\;#1\right)}
\newcommand{\fr}{\mathfrak}

\def\notdivides{\mathrel{\kern-3pt\not\!\kern4.5pt\bigm|}}
\def\nmid{\notdivides}
\def\nsubseteq{\mathrel{\kern-3pt\not\!\kern2.5pt\subseteq}}

\newtheorem{theorem}{Theorem}[section]
\newtheorem{lemma}[theorem]{Lemma}
\newtheorem{proposition}[theorem]{Proposition}
\newtheorem{corollary}[theorem]{Corollary}
\newtheorem{definition}[theorem]{Definition}

\newenvironment{example}[1][Example]{\begin{trivlist}
\item[\hskip \labelsep {\bfseries #1}]}{\end{trivlist}}

\parindent=0pt
\parskip 4pt plus 2pt minus 2pt 

\email{W.B.Hart@warwick.ac.uk}

\title{Computation of quadratic forms}
\author{William Hart}

\begin{document}
\maketitle

\section{Introduction}

\begin{definition}
Given an integer $n \geq 1$ an \emph{$n$-ary quadratic form} is an expression
\begin{equation}\label{qf}
Q(x_1, x_2, \ldots, x_n) = \sum_{i,j=1}^{\infty} a_{ij}x_ix_j,
\end{equation}
for coefficients $a_{ij}$ in some commutative ring $R$ and variables $x_1, x_2, \ldots, x_n$.
\end{definition}

We say that a quadratic form is \emph{integral} if the coefficients $a_{ij}$ are rational integers.

If we write $A = (a_{ij})$ for the matrix of coefficients and $X = (x_1, x_2, \ldots, x_n)^T$ for the column vector of variables, then we can write (\ref{qf}) as
\begin{equation}
Q = X^TAX.
\end{equation}

A quadratic form in two variables is called a \emph{binary quadratic form}. If we write $x, y$ for the variables, it has the form
\begin{equation}\label{abc}
Q(x, y) = ax^2 + bxy + cy^2.
\end{equation}

We will often use the notation $Q := (a, b, c)$ to denote the binary quadratic form (\ref{abc}).

\begin{definition}
The \emph{discriminant} of a binary quadratic form $(a, b, c)$ is the expression $d = b^2 - 4ac$.
\end{definition}

We observe that 
\begin{equation}
4aQ(x, y) = 4a^2x^2 + 4abxy + 4acy^2 = (2ax + by)^2 - dy^2,
\end{equation}
where $d$ is the discriminant of $Q := (a, b, c)$.

From this we see that $(a, b, c)$ is always nonnegative if $d < 0$ and $a > 0$. In this case we call the quadratic form a \emph{positive definite} binary quadratic form. If $a < 0$ and $d < 0$ we say that it is a \emph{negative definite} binary quadratic form. If $d > 0$ the form is called an \emph{indefinite} binary quadratic form.

\section{Positive definite binary quadratic forms}

In this section we restrict our attention to integral binary quadratic forms. We also restrict to forms which are \emph{primitive}, i.e. forms $Q := (a, b, c)$ such that $a, b, c$ do not share a common factor.

We now introduce the concept of equivalent forms.

\begin{definition}
We say that binary quadratic forms $P(x, y)$ and $Q(x, y)$ are \emph{equivalent} if there is an $M \in$ GL$_2(\mathbb{Z})$ such that 
\begin{equation}\label{equiv}
P(x, y) = Q(M(x, y)^T)
\end{equation}
\end{definition}

Matrices in GL$_2(\mathbb{Z})$ have determinant $\pm 1$. We say that forms are \emph{properly equivalent} if the matrix $M$ in (\ref{equiv}) has determinant $1$, otherwise they are \emph{improperly equivalent}.

We see immediately that equivalent forms $P(x, y)$ and $Q(x, y)$ represent precisely the same set of integers as $x, y$ vary over $\mathbb{Z}$ because the matrix $M$, and hence the equivalence of forms, is invertible.

It will turn out that there are only finitely many GL$_2(\mathbb{Z})$ orbits of positive definite binary quadratic forms. 

\begin{lemma}
If two binary quadratic forms $P(x, y) := (a, b, c)$ and $Q(x, y) := (a', b', c')$ are properly equivalent they have the same discriminant.
\end{lemma}

\textbf{Proof:} 
If two forms are properly equivalent then $Q(x, y) = P(M(x, y)^T)$ for some $M \in$ SL$_2(\mathbb{Z})$. But this group is generated by $T = \left(\begin{array}{cc}1 & 1 \\ 0 & 1 \end{array}\right)$ and $S = \left(\begin{array}{cc}0 & -1 \\ 1 & 0 \end{array}\right)$.

The transformation $T$ sends $(a, b, c)$ to $(a, b + 2a, a + b + c)$ and its inverse sends $(a, b, c)$ to $(a, b - 2a, a - b + c)$, whereas the transformation $S$ sends $(a, b, c)$ to $(c, -b, a)$ (note that $S^4 = 1$ so that $S^{-1} = S^3$). In both cases we see that the discriminant is preserved. $\qed$

Legendre introduced a concept of reduced binary quadratic form which we now explore. This allows us to find a ``smallest'' such form in each GL$_2(\mathbb{Z})$ orbit by a process known as reduction.

\begin{definition}A positive definite binary quadratic form is said to be \emph{reduced} if 
\begin{equation}\label{reduced}
|b| \leq a \leq c, \;\mbox{and}\; b \geq 0 \;\mbox{if either}\; |b| = a \;\mbox{or}\; a = c.
\end{equation}
\end{definition}

We now proceed to show that there is a unique reduced form in each GL$_2(\mathbb{Z})$ orbit. We require some intermediate results.

\begin{lemma}
Let $Q(x, y) := (a, b, c)$ be a reduced positive definite binary quadratic form with discriminant $d \neq -3$. The minimum value taken by $Q(x, y)$ as $x, y$ not both zero range over the integers is $a$. If $a = c$ this value is taken at $(x, y) = (\pm 1, 0)$ and $(0, \pm 1)$, otherwise it is only taken at $(\pm 1, 0)$.
\end{lemma}

\textbf{Proof:} As $Q$ is reduced we have that $|b| \leq a \leq c$. Thus 
\begin{equation}
Q(x, y) \geq (a - |b| + c)\,\mbox{min}(x^2, y^2).
\end{equation}

If neither $x$ or $y$ are zero then $Q(x, y) \geq a - |b| + c \geq a$. Similarly if $x = 0$ or $y = 0$ then $Q(x, y) \geq a$. 

If $a = c$ then $|b| \neq a$ otherwise $Q$ is not primitive (if $a = b = c = 1$ then $d = -3$).

Thus if $xy \neq 0$ then $Q(x, y) > a$. So the only time $Q(x, y) = a$ is if $x = 0$ or $y = 0$. In the first case we have $(x, y) = (0, \pm 1)$ and in the second case $(x, y) = (\pm 1, 0)$. 

The other possibility is that $a < c$. In this case, if $xy \neq 0$ then $Q(x, y) \geq (a - |b|) + c > a$. Clearly the only remaining possibility is $(x, y) = (\pm 1, 0)$. $\qed$

In the case where $d = -3$ it is clear that the form $Q(x, y) = x^2 + xy + y^2$ is an exception. In this case we see that the minimum value is $1$, taken at $(x, y) = (\pm 1, 0)$, $(0, \pm 1)$, $(1, -1)$ and $(-1, 1)$. 
 
\begin{theorem}
If $P(x, y) := (a, b, c)$, $Q(x, y) = (a', b', c')$ are properly equivalent reduced positive definite binary quadratic forms then $P = Q$.
\end{theorem}

\textbf{Proof:}
We assume for the moment that $d \neq -3$. 

As $P$ and $Q$ are properly equivalent there exist integers $\alpha, \beta, \gamma, \delta$ with $\alpha\delta - \beta\gamma = 1$ such that $Q(x, y) = P(\alpha x + \beta y, \gamma x + \delta y)$. 

By the previous theorem, $P$ and $Q$ have the same minimum value, thus $a' = a$. 

If $a = c$ then $P$ achieves its minimum value at $(\pm 1, 0)$ and $(0, \pm 1)$. Therefore $a = Q(1, 0) = P(\alpha, \gamma)$. Thus $(\alpha, \gamma) = (\pm 1, 0)$, or $(0, \pm 1)$.

In the first case, $(\beta, \delta) = (r, \pm 1)$ for some integer $r$. Thus $Q(x, y) = P(\pm x + ry, \pm y)$. Thus $b' = b \pm 2ar$. By the fact that $Q$ is reduced, we see that $r = 0$, $b' = b$ and $c = c'$. Thus the forms are equal.

In the second case $(\beta, \delta) = (\mp 1, r)$. Thus $Q(x, y) = P(\mp y, \pm x + ry)$. Thus $b' = -b \pm 2cr$. Again we see that $r = 0$ and $b' = -b$. Here we must have $a = c = c'$ and so $b, b' \geq 0$. This means that $b' = 0$ and the forms are equal.

When $a < c$ we have that $P$ achieves its minimum at $(\pm 1, 0)$. The argument here is the same as for the first case above.

Finally, when $d = -3$ we have $-3 = b^2 - 4ac \leq a^2 - 4a^2 = -3a^2$. Thus $a = 1$. But this implies that $b = \pm 1$ and thus $c = 1$. But the form is reduced, thus $b = 1$. So we see there is only one reduced form of discriminant $d = -3$ and the result follows. $\qed$

\begin{theorem}
Every positive definite binary quadratic form is properly equivalent to a unique reduced binary quadratic form.
\end{theorem}

\textbf{Proof:} We have already shown uniqueness, thus it suffices to show that every form $P(x, y) := (a, b, c)$ is properly equivalent to a reduced one.

Let $Q := (a', b', c')$ be a form properly equivalent to $P$ with minimum positive $a'$. As the discriminants of $P$ and $Q$ are the same we must have $c' > 0$. 

The transformation $U = \left(\begin{array}{cc}1 & u \\ 0 & 1 \end{array}\right)$ sends $Q$ to $Q' = (a'', b'', c'') = (a', b' + 2ua', c'')$. We can therefore choose $u$ so that $|b''| \leq a''$. 

But the transformation $S$ above sends $(a'', b'', c'')$ to $(c'', -b'', a'')$ and therefore $a'' = a' \leq c''$ by the choice of $a'$. 

Thus every class contains a form $(a, b, c)$ with $|b| \leq a \leq c$.

This form is already reduced unless $b < 0$ and either $a = -b$ or $a = c$. In both cases $(a, -b, c)$ is reduced. 

In the first case $(a, -a, c)$ is sent to $(a, a, c)$ by the transformation $T$ above and in the second case $(a, b, a)$ is sent to $(a, -b, a)$ by the transformation $S$ above. 

Therefore in both cases $(a, b, c)$ is properly equivalent to $(a, -b, c)$ and the theorem is proved. $\qed$

\begin{theorem}\label{ineq}
If $P(x, y) := (a, b, c)$ is a positive definite reduced form of discriminant $d$ then $|b| \leq a \leq \sqrt{|d|/3}$ and $c \leq (1-d)/4$. 
\end{theorem}

\textbf{Proof:}
As the form is reduced, we have $-d = 4ac - b^2 \geq 4a^2 - a^2 = 3a^2$, and the first part of the result follows. 

For the second part we note that
$$c = \frac{b^2 - d}{4a} \leq \frac{a^2 - d}{4a}.$$

For constant $d$ this expression has negative derivative with respect to $a$ in the interval $[1, \sqrt{|d|}]$ and is therefore decreasing. It therefore achieves its maximum value at $a = 1$ which gives the required result.$\qed$

The average value of $h(d)$, the number of reduced forms of discriminant $d$, is known to be asymptotic to $C\pi^{-1}\sqrt{|d|}$ where $C \approx 0.8815$ and in fact $h(d)$ is unconditionally bounded above by $O(\sqrt{|d|}\log{|d|})$ (see \cite{cohen}).

Our first algorithm for computing all the primitive reduced binary quadratic forms for a disciminant $d < 0$ is essentially that of \cite{buchmann}. 

\begin{algorithm}{ReducedForms1}{
   \qcomment{Compute primitive reduced binary quadratic forms for $d < 0$}}
A $\qlet \sqrt{-d/3}$ \\
$h \qlet 0$ \\
$L \qlet \emptyset$ \\
\qfor $a \qlet 1$ \qto A \\
\qfor $b \in \Z$ such that $-a < b \leq a$ and $b^2 \equiv d \pmod{4a}$ \\
$c \qlet (b^2 - d)/4a$ \\
\qif $(a, b, c)$ is primitive and reduced \\
\qthen $L \qlet L \cup \{(a, b, c)\}$ \\
$h$ \qlet $h + 1$ \qfi \qrof \qrof \\
\qreturn $L$, $h$
\end{algorithm}
 
The algorithm makes use of the following observation:
if $P := (a, b, c)$ is a reduced form, then as $d = b^2 - 4ac$
we have $b^2 \equiv d \pmod{4a}$.

On the other hand, if $b$ is a square root of $d$ modulo $4a$ in the range $-a < b \leq a$ then $b^2 = d + 4ac$ for some integer $c$, i.e. $d = b^2 - 4ac$. 

\textbf{Performance improvements}

\begin{itemize}
\item Most of the run time is in finding square roots modulo $4a$. This can be optimised by using the Tonelli-Shanks algorithm to compute square roots modulo each prime $p$ dividing $4a$ and Hensel lifting the roots to square roots modulo each $p^k \;|\; 4a$. These roots can then be combined using the Chinese Remainder Theorem.

\item In order to compute the square roots modulo $4a$, the factorisation of $4a$ must be known. It is more efficient to compute the factorisations of all the values $4a$ in advance by sieving. This can be done by sieving by all primes $p$ up to $\sqrt{A}$ and by every power $p^k \leq A$ of such such a prime.

\item One can precompute square roots of $d$ modulo all powers of primes $p^k \leq A$ for odd primes $p$ and modulo all $2^k \leq 4A$. Having been precomputed, these roots can be combined using chinese remaindering to obtain square roots modulo $4a$ for each $a$ in turn.

\item In order to compute $c$ we must first compute $b^2 - d$. This can have the value $4|d|/3$ which may cause an overflow, depending on how integers are stored. However, if $d$ is stored as a C long and $b^2 - d$ and $4a$ as C unsigned longs, then no overflow occurs when computing $c = (b^2 - d)/4a$. 
\end{itemize}

\textbf{Analysis}

\begin{itemize}
\item Sieving requires computing primes up to $(-d/3)^{1/4}$. There can be at most one prime which exceeds this bound which divides any value $4a$, and it does so with multiplicity at most 1. Thus after dividing through by the highest power of each prime dividing $4a$, we are left with either 1 or a prime $p$. 

\item Clearly finding all the necessary primes can be completed in time $\tilde{O}(|d|^{1/4})$.

\item Sieving the interval $[1, A]$ for prime factors can be done in time $\tilde{O}(|d|^{1/2})$. 

\item Computing square roots modulo $p$ can be done in polynomial time (assuming we can find a quadratic non-residue modulo $p$). Similarly lifting these to roots modulo prime powers can be done in polynomial time, for $p \;\notdivides\; d$. 

\item If $p \;|\; d$ and $p \;|\; 4a$, then the number of square roots of $d$ modulo $4a$ is bounded by $4a/p$. Thus, computation of roots modulo values $4a$ divisible by primes dividing $d$ can also be done in time $\tilde{O}(|d|^{1/2})$.

\item Finally, searching for primitive, reduced forms from the roots that are computed takes time $\tilde{O}(|d|^{1/2})$.
\end{itemize}

We see that the following theorem holds:

\begin{theorem}Assuming the Generalised Riemann Hypothesis (required to guarantee efficiently finding a quadratic nonresidue in Tonelli-Shanks), we can find all primitive, reduced binary quadratic forms of discriminant $d < 0$ in time $\tilde{O}(|d|^{1/2})$. We must compute $\tilde{O}(|d|^{1/4})$ primes.
\end{theorem}

The algorithm above is quite complicated to implement. In FLINT we only implement the first two of the three performance improvements, and instead of sieving with all powers of the primes $p$, we simply sieve with the primes $p$ and then determine the highest power of $p$ dividing each value $4a$ that it divides.

There is an alternative algorithm which is slightly easier to implement and is slightly faster in practice, but it requires the computation of a greater number of primes. This makes it less practical for larger discriminants.

The formulation is suggested by Algorithm 5.3.5 of \cite{cohen}.

\begin{algorithm}{ReducedForms2}{
   \qcomment{Compute primitive reduced binary quadratic forms for $d < 0$}}
B $\qlet \sqrt{-d/3}$ \\
$h \qlet 0$ \\
$L \qlet \emptyset$ \\
\qfor $b \qlet 1$ \qto B \\
\qfor $a, c \in \N$ such that $ac = (b^2 - d)/4$ \\
\qif $(a, b, c)$ is primitive and reduced \\
\qthen $L \qlet L \cup \{(a, b, c)\}$ \\
$h$ \qlet $h + 1$ \qfi \\
\qif $(a, -b, c)$ is primitive and reduced \\
\qthen $L \qlet L \cup \{(a, -b, c)\}$ \\
$h$ \qlet $h + 1$ \qfi \qrof \qrof \\
\qreturn $L$, $h$
\end{algorithm}

This algorithm relies on being able to factor the expressions $(b^2 - d)/4$. 

\textbf{Performance improvements}

\begin{itemize}
\item Most of the run time is in factorisation of values $(b^2 - d)/4$. This can be optimised by computing all the prime power factors in advance and then iterating through all possible exponents to find factors $a$. We then compute $c = (b^2 - d)/(4a)$ for each such factor $a$.

\item We can use (quadratic) sieving to find all the prime factors of the values $(b^2 - d)/4$ in advance. Suppose that an odd prime $p$ divides $(b^2 - d)/4$. Then $b^2 \equiv d \pmod{p}$. We can use the Tonelli-Shanks algorithm to find square roots $r_0, r_1 \pmod{p}$. We can then mark off all values in the range $[1, B]$ congruent to $r_0$ or $r_1 \pmod{p}$.

\item Once we know for which values of $b$ the expression $(b^2 - d)/4$ is divisible by a given prime $p$, we can simply compute the maximum power of $p$ dividing each such value $(b^2 - d)/4$. This turns out to be faster in practice than sieving with higher powers of $p$. 

\item We are factoring values $(b^2 - d)/4$. In order to do so, we must first compute $b^2 - d$. Thus the same overflow considerations occur as for the first algorithm.
\end{itemize}

\textbf{Analysis}

\begin{itemize}
\item The largest value $(b^2 - d)/4$ can take is $-d/3$. Thus sieving requires computing primes up to $(-d/3)^{1/2}$. 

\item Clearly finding all the necessary primes can be completed in time $\tilde{O}(|d|^{1/2})$.

\item Quadratically sieving the interval $[1, B]$ for prime factors can be done in time $\tilde{O}(|d|^{1/2})$ once the square roots are known. 

\item Computing square roots modulo $p$ can be done in polynomial time (assuming we can find a quadratic non-residue modulo $p$). 

\item Finally, searching for primitive, reduced forms from the triples $(a, b, c)$ that are computed takes time $\tilde{O}(|d|^{1/2})$.
\end{itemize}

We clearly get the same time complexity as for the previous algorithm. However, we now need $\tilde{O}(|d|^{1/2})$ primes. The advantage is that the algorithm is slightly easier to implement (in FLINT we implement all the listed performance improvements).

The version of the second algorithm in FLINT is about 50\% faster than the first algorithm. Of course, this may be due to not implementing all the performance improvements listed for the first case. 

As we shall see later, computation of the class number $h$, only, can be completed much faster than computation of the list of reduced forms $L$. 

\begin{thebibliography}{99}

\bibitem{buchmann} J. Buchmann, U. Vollmer \emph{Binary Quadratic Forms: An Algorithmic Approach}, Springer, 2007.

\bibitem{cohen} H. Cohen, \emph{A Course in Computational Algebraic Number Theory}, Springer, 1996.

\end{thebibliography}
\end{document}
